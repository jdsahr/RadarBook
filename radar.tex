% $Id: radar.tex,v 1.5 2007/06/09 18:58:09 jdsahr Exp $

\documentclass[12pt]{book}

\nonstopmode

\usepackage{makeidx}
\usepackage{epsfig}

% \usepackage{layout}

% macros.tex.
%
% $Id$
%
% These are several LaTeX commands and environment modifications which
% allow improved typesetting of Cornell University dissertations.
% 
% Questions to John Sahr johns@alfven.spp.cornell.edu
%

\typeout{JDS MACROS}

\newcommand{\citey}[1]{\cite{#1}}

\newcommand{\sn}[2]{{#1} \times 10^{#2}}  % scientific notation a x 10^b
\newcommand{\cmto}[1]{{\rm cm}^{#1}}      % simple power macro for cm units
\newcommand{\tens}[1]{\stackrel{\rule{1ex}{0.5mm}}{#1}}
					  % like \vec{} for tensors
\newcommand{\nablaperp}{\nabla_{\!\perp}} % a better \nable_\perp
\newcommand{\pr}{\prime}                  % abbreviation for \prime
\newcommand{\ppr}{{\prime\prime}}         % abbreviation for \prime\prime

\newcommand{\llangle}{\left\langle}       % big left langle
\newcommand{\rrangle}{\right\rangle}      % big right rangle

\newcommand{\Ci}{\textrm{Ci}}
\newcommand{\Si}{\textrm{Si}}

\newcommand{\sinc}{\textrm{sinc}}
\newcommand{\cosech}{\textrm{cosech}}

\newcommand{\iint}{\int\!\!\!\int}
\newcommand{\curl}{\nabla\times}

\newcommand{\noop}[1]{}                   % handy for ``commenting out''
\newtheorem{definition}{definition}       % definition environment
\newtheorem{example}{example}             % example environment

\newcounter{deffc}
\newenvironment{deff}[1]{\noindent\stepcounter{deffc}\textit{def \arabic{chapter}.\arabic{deffc}}: \textbf{#1}\rule{0mm}{2em} ---}{\\}

\newlength{\defaultskip}
\setlength{\defaultskip}{\baselineskip}

% My labelling scheme works as follows:
%
%  figure   references all begin with ``f:''
%  table    references all begin with ``t:''
%  text     references all begin with ``p:''
%  equation references all begin with ``e:''

\newcommand{\eqref}[1]{(\ref{e:#1})}          % formatted equation
					      % reference, i.e. (3.31)
\newcommand{\figref}[1]{Fig.~\ref{f:#1}}      % formatted figure
					      % reference, i.e. Fig.~3.31
\newcommand{\tabref}[1]{Table~\ref{t:#1}}     % formatted table
					      % reference, i.e. Table~3.31
\newcommand{\coderef}[1]{Code Ex.~\ref{c:#1}} % formatted code example
					      % reference, i.e. Code Ex.~3.31

% the following function may be needed for my thesis
% \newtheorem{definition}{Definition}[chapter] % environment for
%					      % definitions 

% tcaption{} is the same as caption, except that you get a single
% spaced caption instead of regular text spacing.  tfigure{}
% (ttable{}) is the same as figure{} (table{}), forcing the [tb]
% placement, and  making sure the lof gets made right.  This should
% be used for both tcaption{} and ttable{} (below).

\newcommand{\tcaption}[1]{\caption{#1\SS}}

\newenvironment{tfigure}{\begin{figure}[tb]}{\addtocontents{lof}{\protect\addvspace{\lotlofextraspace}}\end{figure}} 

% ttable{} the considerations are the same as for tfigure{}.

\newenvironment{ttable}{\begin{table}[tb]}{\addtocontents{lot}{\protect\addvspace{\lotlofextraspace}}\end{table}}

% yacc{} environment with \item[] sets code for yacc somewhat nicely

\newcommand{\yitem}[1]{\item[{{\bf #1} \hfill}]}

\newenvironment{yacc}%
{\SS\begin{list}{}{\itemsep=1ex \leftmargin=1.25in \labelwidth=1.0in \parsep=0.0pt \labelsep=.13in}}{\end{list}\DS}

% from Kopka and Daly, page 192, for typesetting chemical formulas
\newlength{\fntxvi} \newlength{\fntxvii}
\newcommand{\chemical}[1]
{{\fontencoding{OMS}\fontfamily{cmsy}\selectfont
  \fntxvi\the\fontdimen16\font
  \fntxvii\the\fontdimen17\font
  $\mathrm{#1}$
  \fontencoding{OMS}\fontfamily{cmsys}\selectfont
  \fontdimen16\font=\fntxvi \fontdimen17\font=\fntxvii}}

% end of macros.tex


%\theoremstyle{plainn}
% \newtheorem{thm}{Theorem}[section]
\newtheorem{lemma}{Lemma}[section]
\newtheorem{defn}{Definitions}[section]

%\newcommand{\answer}[1]{}
\newcommand{\answer}[1]{\noindent \textbf{Answer}: {#1} }

% \newfont{\jds}{cmr12 at 24pt}

\title{Radar Remote Sensing of Deep Fluctuating Targets}
\author{John D. Sahr}

\makeindex
\typeout{hello}
\begin{document}

\maketitle

\rule{0mm}{0mm}
\vfill

\centerline{\copyright{1999--2017} by John D.~Sahr}

\vfill

\rule{0mm}{0mm}

\tableofcontents

\newpage
% \layout

\part{Basics}

\chapter{Introduction}

Radar is  a mature technology whose basic principles are very well 
understood. However, enormous potential remains, so research in 
radar technique remains very lively.  The innovators of radar\footnote{In this text we'll 
elevate "radar" to the status of a common noun from its origin as an acronym, 
RADAR, "RAdio Detection And Ranging."} technology have always 
been able to imagine instruments which are impossible to build in that era
but which have become possible in a subsequently.

By the beginning of the 21st Century several technological advances have changed a number
of the driving forces in radar design.  In particular, high speed, low
cost computing has permitted access to signal processing algorithms that
were effectively impossible only a decade earlier.  The availability
of the internet, and GPS time/frequency reference has enabled startling
new opportunities for multistatic radars and indeed an entirely new
topology of radar network.  Although advances in RF electronics
have increased performance in various ways (lower noise figure, higher
dynamic range, ever higher frequencies, semiconductor power amplifiers
replacing vacuum devices), the performance revolution has not been so
dramatic as in computation.  Furthermore, the cost of electricity has
not changed substantially --- so high power transmitters remain
expensive to acquire and to operate.  However, one of the parameters
above --- dynamic range --- does affect our ability to make full use
of modern digital methods. 
The tension between the possible in principle and the possible in
practice has created an enormous repertoire of tricks to permit the
performance of ``impossible'' measurements.  As technology improves,
these tricks' importance wanes, although the motivation for the tricks
remain --- and an understanding of the tricks can conceivably enable a
new generation of radars which could not have been approached at all.
We will describe how clever design of waveforms and receivers can 
\begin{itemize}
\item eliminate significant imperfections of the receivers
\item permit ``gigaflop'' processing with ``megaflop'' computers
\item enable fine resolution of targets whose range and velocity
spectrum cannot be studied by ``conventional'' means.
\end{itemize}

The myriad uses of radar instruments provide a constant demand for
more powerful transmitters, more sensitive receivers, greater receiver
dynamic range, more real-time signal processing capability, more
stable time and frequency references, and better (lower) component
cost.  As radar is applied to complex consumer tasks, such as air
traffic control, even human psychology becomes an issue as we struggle
to present complex fields of data to humans in a way which is natural
and comprehensible, permitting the optimal deployment of extremely
valuable resources.

The basic idea behind radar is simple: we measure the time of flight
of a radio wave pulse to deduce the distance to an object which
reflected the radio wave.  This is the origin of the acronym ``RADAR''
which stands for ``RAdio Detection And Ranging.'' The ubiquity of
RADAR has brought it wholly into the language as conventional word,
truly deserving the honor of lower case letters, as if it were of Greek or Latin origin.

Indeed, the idea of radar has become so common that it is used in very
different contexts.  A social, political, or economic issue becomes
important, and we hear expressions such as ``the discovery of taxol in
the bark of the yew has made the preservation of rare species show up
on the legislative radar screen.''  Or, just as interesting, we may
also read of entities attempting to avoid recognition, as in ``the
various industries which produce toxic waste are flying under the 
radar.''  Somehow the greater civilization has come to understand that 
``flying low'' has something to do with avoiding detection by radar, an 
idea which is basically correct, and about which we'll be able to say a 
few things in a subsequent chapter.

As it happens, not all radars are intended for range detection, and it
may be difficult to draw the line between ``radar'' and ``non radar''
instruments.  We will frequently give examples from our own radar,
which measures the complete range and Doppler spectrum of extended
targets --- and yet our radar has no transmitter.  Does it qualify as
a radar?  Judge for yourself.

The history of radar is fascinating.  Contrary to common belief, radar
was not suddenly discovered in the 1940s in service of World War II, but 
long before.  It is true that radar was of profound importance in WWII, 
because the Allies successfully developed and exploited it, while the
Axis did not\footnote{need a reference for this}.

\section{some applications --- needs fixing}

RADAR\index{RADAR} --- an acronym which means ``Radio Detection and Ranging''

basic idea: emit a pulse, wait for echoes, factor in $c$ to get
distance.

But radar is broad:
\begin{itemize}
\item ``picket fence'' (NAVSPASUR), CW ``burglar alarms,'' "tripwires"
\item weather radar (deep, volume targets)
\item bistatic, multistatic, forward scatter geometries
\item Synthetic Aperture
\item Planetary radar astronomy
\item passive radar
\item aerospace applications (point targets)
\end{itemize}
Incredibly broad set of applications, techniques, frequencies, costs,
complexities ... very difficult to give single set of design equations
that cover everything.

Need to acquire basic skills
\begin{itemize}
\item Analysis --- come to understand what a radar is doing, and why
\item Synthesis --- extend knowledge to design new systems.
\end{itemize}

compare to Forward Problems, Inverse Problems.

\section{Radar before World War 2}

It's possible to debate the origins of radar.  One possible origin
would be with Hertz and Maxwell at the end of the Nineteenth Century, 
who performed experiments in which a transmitter produced a wave, a 
receiver detected the wave, and output of the receiver was affected by the presence of a third object.

There are some earlier experiments worthy of mention, associated with
attempts (mostly unsuccessful) to measure the speed of light\footnote{need 
references; lantern unveiling, michelson-morley}.  Electromagnetic waves travel 
stupendously fast in human terms.  However light is almost sluggish in ways 
which limit high speed electronics.  Modern desktop CPUs can execute several 
instructions in the time it takes a radio wave to span the container in which the 
computer resides. However, our scientific forebears did not have access to equipment
which could easily split the second into billionths.  Even with the best
portable clocks using some variant of spring and mass, it's clear that
efforts to time the uncovering of lanterns on mountaintops (i.e. Ben
Franklin et al.) were doomed to failure.

\subsection{Well Before Maxwell}

These failures were not due to fundamental limitations in the clocks,
but the short distance over which the measurement was attempted.  If
you can resolve 1 microsecond, then 1 km is about the minimum size of
an experiment which can measure the speed of light.  If your time
resolution is one second, then the experiment should occupy more than
300,000 km, roughly the distance to the Moon.  If you wish to resolve
four decimal places with a clock which ticks seconds, the experiment
must be a thousand times larger, about the distance to the Sun.

Dutch astronomer Ole Roemer (years) was able to perform this experiment
with breathtaking precision.  With the invention of the telescope,
astronomers were able to measure the orbital period of the moons of
Jupiter, and that these orbital periods were very well defined --- in
other words, they are a very stable clock.  Roemer noticed that the
time of ``rise'' of these moons depended upon whether Jupiter was near
or far from Earth.  Since Jupiter's orbit has 5.5 times the diameter
of Earth, it is sometimes as near as 4.5 $R_e$ and as far as 6.5
$R_e$, a differential distance of about 300 million km, about a thousand
seconds of light travel time.  Unfortunately, in Roemer's day the
distance from the Earth to the Sun was not known\footnote{It is interesting
to contemplate: how would you measure the distance to the Sun?}, although the ratio
of Jupiter's to Earth's orbital radius (5.5) was known.  So, Roemer
was able to express the speed of light (simplifying greatly) as 
\begin{displaymath}
c = \frac{2R_e}{1000}
\end{displaymath}
The symbol $c$ is derived from ``celeritas'' which is Latin for
``velocity,'' and which symbol was first used by Newton.

\subsection{Michelson-Morley}

The Michelson-Morley experiment (1880s) was developed to measure
the speed of light with sufficient precision to formally test a
hypothesis of the ``ether'' theory of electromagnetic wave
propagation.  The experiment resembled Roemer's observation except
that the path and time scales were much shorter.

Michelson and Morley developed a ``coincidence detector'' which in
which a pulse of light appeared at a detector only when it traversed a
fixed (known) path and managed to reflect from two rotating mirrors at
exactly the right moment in their rotation.  By varying the rotation
speed of the two mirrors, different candidate speeds of light could be
tested (stating the experiment this way foreshadows the ``matched
filter and ambiguity function'' topics to come.).  The two mirrors
were actually different faces of an octagonal cylinder, and indeed
there are several solutions to the problem, but the least rotational
speed which yields a stable image unambiguously provides the speed of
light.  The timescale of the experiment was under a millisecond, and
the time resolution less than a microsecond, so the experiment fit
onto an Earthly laboratory (a hundred meters or less).

\subsection{James Clerk Maxwell}

Maxwell was a prolific physicist.  Among his major accomplishments was
a unification of the electric and magnetic fields through what are now
known as ``Maxwell's Equations.''  In modern (differential) form, and MKSA units,
these equations are as follows
\begin{eqnarray}
\nabla\times \vec{E} &=& -\frac{\partial \vec{B}}{\partial t} \\
\nabla\times \vec{H} &=&  \frac{\partial \vec{D}}{\partial t} + \vec{J} \\
\nabla\cdot \vec{D}  &=& \rho \\
\nabla\cdot \vec{B}  &=& 0
\end{eqnarray}
Maxwell's particular contribution was the addition of the $\partial
\vec{D}/\partial t$ term.  Again, in modern notation, if we apply the
constitutive relations
\begin{eqnarray}
\vec{D}    &=& \epsilon \vec{E} \\
\epsilon_0 &=& 8.854...\times 10^{-12} \textrm{F/m} \\
\vec{B}    &=& \mu \vec{H} \\
\mu_0      &=& 4\pi \times 10^{-7} \textrm{H/m}
\end{eqnarray}
and develop a wave equation for the electric or magnetic fields, we
find
\begin{equation}
\left(\nabla^2 - \mu \epsilon \frac{\partial^2}{\partial t^2}\right) \vec{E}  = 0
\end{equation}
Here $\mu \epsilon$ is recognized to have the units of inverse
velocity (squared) and in free space that velocity is
\begin{equation}
c = \sqrt{\frac{1}{\mu_0 \epsilon_0}} \approx 3 \times 10^8 \;
\textrm{m/s}
\end{equation}
There are several important features of this development.
\begin{itemize}
\item the vacuum propagation speed $c$ is independent of time and
length scales --- thus radio waves and light waves become different
aspects of the same phenomenon, as well as infrared light, x rays, and
gamma rays.
\item the wave equation is not dispersive (in vacuum) so that
well-formed pulses retain their shape as they propagate.  
\item $\mu$ and $\epsilon$ can in principle be measured in
electrostatic and magnetostatic experiments (the constitutive
relations), so the that speed of propagation $c$ can be deduced from
very different experiments from ordinary ``speed trials.''
\end{itemize}
Michelson and Morley and others performed their experiment to address
troubling questions that arise if two observers with different
velocities observe the same electromagnetic experiment: when we say
``the speed of light is $3\times10^8$ m/s,'' what should the person
who is moving west at 10 m/s say?  The answer turns out to be
``Special Relativity,'' which means that Maxwell's Equations are
ultimately far more important than descriptors of electromagnetic
waves: they forced the development of the theories of Special and
General Relativity.  From the moment that Maxwell's Equations for
electromagnetics were offered, twenty five years passed before
Einstein published his theory of Special Relativity.  In retrospect it
needn't have taken this long, but theories of relativity require us to
think rather profoundly about the nature of space-time, as opposed to
the more immediate view that space and time are largely independent,
and certainly not related by the motion of observers.

\subsection{Early Studies of the Ionosphere}

Ionosondes --- 1918; Eckersley 1937.

\subsection{early aerospace}

Passive Radar in England

\section{From World War 2 to the Digital Age}

Although radar was not really invented in World War 2, the explosion
of research and development during and immediately after WW2
fundamentally colored the perception of the radar problem, its
nomenclature, and established its importance as a tool in modern
warfare and national defense.

\subsection{VHF, UHF, EHF, sources and detectors}

\subsection{algorithms: MTI, conical scan, monopulse}

WWII and the Cold War

\subsection{Big Radars}

\subsubsection{BMEWS/DEW Radars}

\subsubsection{Planetary Radar Astronomy}

\subsubsection{Thomson Scatter Radars}

\section{The Early Digital Age}

\subsection{SAR and ISAR}

\subsection{phased arrays}

\section{The Modern Era: since 1985}

\subsection{high speed elecronics}

\subsection{time and frequency reference}

\section{Why Study Radar?}

If you're going to be a radar technician, the answer is obvious.  But
suppose you are \textbf{not} going to be a radar jock.  Is radar still
worth looking at?

\subsection{Radar as dual of Communications}

In Communications, you try to send an unknown data signal through a fairly
well-known channel, in hopes of recovering the data

In Radar, you send a fairly well-known signal through a poorly-known
channel, in hopes of recovering the structure of the channel.

Note, in some communications systems there is a ``training'' sequence
in which a known signal is sent through the channel to discover the
approximate impulse response of the channel.

\subsection{Radar as an Inverse Problem}

If we know the strength of the transmitter, the distance to the target, the size of the target, 
then we can predict how much signal will be received.  This is a \textit{Forward Problem}, 
which is an important class of problems to understand.  

However, more interesting would be the following: if we transmit a known signal, and at some
later time receive an echo of that signal, what can we deduce about the nature of the target that 
caused the reflection?  This is an \textit{Inverse Problem}.

To cast this differently, on a billiards table, if you know the position of all the balls, and the frictional
forces that slow them, then upon striking the cue with a known vector impulse, it should be possible
to deduce the terminal position of all the balls --- a forward problem.

On the other hand, suppose you viewed a configuration of balls, and were asked to determine the initial 
configuration as well as  the impulse given to the cue
As presented, this problem is hopeless.   But suppose that one was able to know "how many ball-ball
collisions were there?" and "what were the instants in which they occurred?"  This problem would still
be very difficult, but the number of solutions could, conceivably be manageable.


\subsection{Radar as driver for Engineering Art}

Radar requires extremes of performance:
\begin{itemize}
\item RF systems
  \begin{itemize}
  \item low power/weak signals
  \item high power/transmitters
  \item very fast switches
  \item Antennas
  \item RFI/EMC issues
  \end{itemize}
\item Signal Processing
  \begin{itemize}
  \item Detection \& Estimation Theory
  \item Parameter Estimation --- Inverse Problems
  \item Tracking
  \item Filtering
  \item Statistics (lots!)
  \end{itemize}
\item Digital Systems/Mixed Signal Systems
  \begin{itemize}
  \item Data Acquisition
  \item Data Transport
  \item Data Storage
  \item ASIC design
  \item EMC/EMP robustness
  \end{itemize}
\item Software
  \begin{itemize}
  \item (Graphical) User Interface design
  \item Real Time Systems
  \item Reliability/Robustness
  \item Networking
  \item Documentation
  \end{itemize}
\end{itemize}



% $Id: chapter2.tex,v 1.2 2005/01/23 01:06:44 jdsahr Exp $

\chapter{Some Radar Basics}

\section{Range Estimation (first pass)}

The basic idea of radar: emit pulse at $t=0$, hear an echo at $t=T$. 	

Q: what is the distance to the target?

A: $T$ is the round trip time, and $cT$ is the total distance traveled
by the wave, out plus back.  thus

\begin{equation}
2R = cT \;\;\;\; \rightarrow \;\;\;\; R = \frac{c}{2}T
\end{equation}

The quantity $c/2$ appears frequently; it is half the speed of light,
or, colloquially, ``the speed of radar'':
\begin{eqnarray}
\frac{c}{2} &=& 1.5\times 10^8 \; \frac{\textrm{m}}{\textrm{s}} \\
            &=& 150 \; \frac{\textrm{m}}{\mu\textrm{s}} \\
            &=& 150 \; \frac{\textrm{km}}{\textrm{ms}}
\end{eqnarray}

\begin{example}
A police radar ``zaps'' a car 150 m away.  How long before
the echo returns?

\answer{$T = R/(c/2) = 1\;\mu s$}
\end{example}

\begin{example}
Radars studying the Moon receive echoes after about 2.5 seconds.  So,
how far away is the Moon?

\answer{$R = 1.5\times 10^8 \; \times \; 2.5 = 3.75\times 10^8 =
  375,000 \; \textrm{km}$}

\end{example}

\section{Doppler Shift (intro)}

Consider a transmitter located at $x_0=0$ which radiates a signal
$\exp(j\omega t)$,  The field in space is $\exp(j(\omega t - k x))$.

Now, a target at $x=L$ scatters some energy back to the transmitter,
\begin{eqnarray}
\textrm{signal at L} &=& \exp(j(\omega t - kL)) \\
\textrm{signal scattered back is} &=& \exp(j(\omega t - k [2L])) \\
 &=& \exp(j(\omega t - 2kL)) 
\end{eqnarray}

So, now suppose that the target is moving, so that $L \rightarrow L +
vt$.  Then 
\begin{eqnarray}
\textrm{the recieved signal is} &=& \exp(j(\omega t - 2k(L + vt))) \\
&=& \exp(j(\omega t - 2kVt - 2kL) \\
&=& \exp(j(\omega - 2kV)t - 2kL) \\
&=& \exp(j(\omega + \Delta \omega) t - 2kL) \\
\Delta \omega &=& -2kV = -2\left(\frac{2\pi}{\lambda}\right) V\\
\Delta f &=& -\frac{2V}{\lambda}
\end{eqnarray}

Thus, the Doppler shift is \textit{negative} for targets with
increasing distance (``opening targets'') and \textit{positive} for
targets with decreasing distance (``closing targets'').  

Note also that the velocity $V$ is better described as the ``range rate'' because it represents
the rate of change of the separation of the target and the radar; it is not the actual target velocity.  
More generally we have
\begin{equation}
\Delta f = -2 \frac{\hat{r}\cdot \vec{V}}{\lambda}
\end{equation}
where $\hat{r}$ is a unit vector pointing toward the target from the radar, and $\vec{V}$ is the full vector velocity of the target.

\begin{example} A 10 GHz radar observes an approaching speeding car
  travelling at 30 m/s.  What is the Doppler Shift?

\answer{The wavelength of 10 GHz is 3.33 cm = 0.0333 m, so the Doppler
  Shift $\Delta f$ = -2 (-30 m/s)/(0.0333) = 1.8 kHz.}
\end{example}

\begin{example} A 3 GHz radar looking due East detects an aircraft flying due North.  What is the Doppler shift? \\
\answer{Zero.  This is because $\hat{r}$ is perpendicular to the target velocity vector.}
\end{example}

\begin{example} The Jicamarca Radar Observatory (50 MHz) sees meteor
  ``head echoes'' travelling 50 km/s.  What is the Doppler Shift?

\answer{The wavelength at 50 MHz is 6 m.  Thus, the Doppler Shift is
  2(50,000)/6 = 16.7 kHz.}
\end{example}

The latter example indicates that the receiver had better have sufficient bandwidth at 17 kHz above the transmitter frequency in order to capture this signal.  Meteors, at speeds approaching 100 km/s, are probably the highest velocity high mass objects that will ever be studied with radar.

Note that the Doppler Shift increases with increasing radar frequency, 
and that, as presented, the formulas are not correct with respect to
Relativity.  On the other hand, consider that Jicamarca/meteor
example.  The Doppler shift was 17 kHz compared to a transmitter
frequency of 50 MHz --- a ratio of 1:3000.  This is because the speed 
of light is \textit{very fast} compared to the speed of even the
fastest objects with appreciable mass (i.e. meteors).
When one accounts for relativistic effects we have  \cite{peebles-1998,levanon+mozesan-2004}:
\begin{equation}
f_R = f_T \frac{1 - \hat{r}\cdot \vec{\beta}}{1 + \hat{r}\cdot \vec{\beta}}
\end{equation}
Here $\vec{\beta} = \vec{v}/c$.  In the limit that $v/c \ll 1$, $\Delta f$ reduces to the non relativistic case.  Referring to the meteor example above, $\beta \le \frac{1}{3000}$ for any possible target that we are ever likely to observe.

\section{The Radar Equation}

The so-called ``Radar Equation''\index{radar equation} is a large set
of equations which 
predict the power which is expected to wind up in the receiver after
transmission and scattering.  There are many variations, and no one
true radar equation.  We'll build up a common form that uses the
following assumptions:
\begin{itemize}
\item free space propagation
\item a small target (much smaller than the antenna pattern), which
  scatters all the power it receives isotropically, absorbing none of
  the power.
\item bistatic geometry; distance from transmitter to target is $R_T$,
  and distance from target to receiver is $R_R$
\item Transmitter peak power $P_T$
\end{itemize}

\begin{enumerate}
\item the Transmitter illuminates the target, producing a Poynting
  Flux on the target of 
\begin{displaymath}
S_T = \frac{P_T G_T}{4\pi R_T^2} \;\;\; \textrm{W/m}^2
\end{displaymath}
\item the Target has a scattering cross section of $\sigma$ m$^2$, and
  thus intercepts an amount of power
\begin{displaymath}
P_{target} = \sigma S_T = \frac{P_T G_T \sigma}{4\pi R_T^2} \;\;\;
\textrm{W}
\end{displaymath}
\item The target now reradiates its received power.  In general the power is not scattered uniformly (isotropically), however we will assume isotropic scattering for them moment.  This produces a scattered Poynting Flux at the receiver,
\begin{displaymath}
S_R = \frac{P_{target}}{4\pi R_R^2} = \frac{P_T G_T \sigma}{(4\pi)^2
  R_T^2 R_R^2} \;\;\; \textrm{W/m}^2
\end{displaymath}
\item The receiving antenna has an effective area $A_R$, and thus it
  intercepts a fraction of the scattered power,
\begin{displaymath}
P_R = S_R A_R =  \frac{P_T G_T A_R \sigma}{(4\pi)^2 R_T^2 R_R^2}
\;\;\; \textrm{W}
\end{displaymath}
\end{enumerate}

There is a relationship between the effective area of an antenna, and
the gain of an antenna, namely $G = 4\pi A/\lambda^2$, so this permits
us to write ``bistatic, point target'' form of the radar equation
\begin{equation}
P_R = \frac{P_T G_T G_R \lambda^2 \sigma}{(4\pi)^3 R_T^2 R_R^2}
\end{equation}

For many radar systems the antenna and the receiver are in the same
place ($R_R = R_T = R$) and share an antenna ($G_T = G_R = G$) which
leads to the simpler ``monostatic, point target'' form of the radar
equation,
\begin{equation}
P_R = \frac{P_T G^2 \sigma \lambda^2}{(4\pi)^3 R^4}
\end{equation}

\subsection{features of the radar equation}

The simplicity of the radar equation makes it very easy to analyze,
and there are important implications for radar systems in even this
most basic form.

\begin{itemize}
\item \textbf{Inverse fourth power range dependence.}  As a target of
constant size $\sigma$ changes range, the scattered power varies
rapidly.  A target just detectable (SNR $\approx$ 0 dB) at range $R$ is
easily detectable at range $R/2$ (SNR $\approx$ 12 dB), while being
undetectable at range $2R$ (SNR $\approx -12$ dB).
\item \textbf{Doubly effective antenna.}  For a monostatic radar in
which the same antenna is used for transmission and reception, the
gain appears \textit{twice}, thus a 20 dB antenna contributes 40 dB to
the overall signal strength.
\item \textbf{High dynamic range of required of receivers.}  A target
tracked from 10 km to 100 km has a signal strength change of 10,000.
When coupled with ``clutter'' from targets which are not interesting
but otherwise large (nearby trees, buildings, mountains, etc.), or
electronic countermeasures (ECM) such as jamming, receivers with
dynamic range exceeding 100 dB are desirable. 
\item \textbf{short wavelengths desirable.}  Assuming that engineering
considerations limit the physical size ($A$) of antennas, the signal
strength increases for shorter wavelengths (when the antenna area is
held constant).
\item \textbf{high power transmitters are desirable.}  The scattered
power is clearly proportional to the transmitter power.  At suitably
high power, the transmitted signal may actually modify the target ---
induced heating, for example.
\end{itemize}

It's important to realize that all of the items above can be followed,
by ``yes, but ...'' arguments.  For example, the argument about
preferring short wavelengths fails if the target scattering cross
section $\sigma \propto \lambda^{-n}$ for $n > 2$.  Other engineering
considerations have yet to be considered, such as signal-to-noise
ratio and receiver bandwidth.  In the case of ``Thomson scatter''
radar studies of the ionosphere, the Doppler spectrum of the scattered
signal contains useful information only if the radio wavelength is
considerably larger than the ``Debye Length'' (a centimeter or so)
\cite{nicholson-1983}.  The hostile environment 
encountered by military radars inspires the design of receivers with
features that are very different from ``high quality'' radio receivers.  For example
a poor noise figure may be acceptable if it produces a receiver that is less
likely to suffer overload.  Indeed, if the environment is very noisy, there is 
no point in building a low noise receiver.

The lesson is that it is important to consider the whole radar
\textit{system} to evaluate the performance.  Radar systems frequently
present very unusual constraints which may be very cleverly addressed
if one's engineering judgment is not clouded by conventional
electronic systems.

\begin{example}
A 3 GHz radar with a 30 dBi antenna observes an aircraft with $\sigma
= 100 \textrm{m}^2$ at a range of 100 km.  What is $P_R/P_T$?

\answer{Noting that 30 dBi is a factor of 1000 in gain, we have
\begin{displaymath}
\frac{P_R}{P_T} = \frac{G^2\lambda^2\sigma}{(4\pi)^3 R^4} = 5\times
10^{-20}
\end{displaymath}
or about -193 dB.}
\end{example}

\begin{example} In the previous example, suppose the TX power is 100
  kW.  What is the receiver voltage (into $50 \Omega$)?

\answer{Continuing, we find that the actual received power is $5\times
  10^{-15} = \frac{V^2}{2\cdot 50}$, thus the RMS voltage is $0.707
  \mu$V. }
\end{example}



\subsection{non-isotropic Targets}

We assumed that the target scattered isotropically, but this turns out
to be a poor approximation --- there are literally (and provably) no
isotropic scatters at all.

In general $\sigma$ is therefor not a constant, but rather a function
of 
\begin{itemize}
\item the direction of the illuminating radiation $\hat{\imath}$
\item the direction of the scattered radiation $\hat{o}$
\item the wavelength of the illumination
\item the polarization of the illuminating radiation,
\item the polarization of the scattered radiation
\end{itemize}
So, we should more generally write
\begin{displaymath}
\sigma \rightarrow \underline{\underline{\sigma}}(\hat{\imath},\hat{o},k)
\end{displaymath}
where the double underline indicates the tensor nature of $\sigma$
scattering between various polarization states.  In the radar equation
above one would have to replace the appearance of $\sigma$ with
something like
$\hat{p}_i\cdot\underline{\underline{\sigma}}(\hat{\imath},\hat{o},k)\cdot\hat{p}_o$,
where $\hat{p}_{i,o}$ is the polarization state of the incoming and
outgoing waves.

\subsection{Deep (wide) Targets}

The ``point target'' version of the Radar Equation is misleading for
``deep targets'' such as weather, which fill the antenna beam
pattern.  Such targets are described not by scattering cross section,
but rather by ``volume scattering cross section,'' given the symbol
$\sigma_v$ and having units ``area/volume''.  For such targets, the
total cross section $\sigma = \sigma_v V$ where $V$ is the illuminated
volume.  Thus
\begin{eqnarray}
\sigma &=& \sigma_v V \\
V &=& R^2 \Omega \Delta R \\
\Delta R &=& \textrm{range resolution} \\
\Omega &=& \frac{4\pi}{G} = \textrm{Antenna Beam Solid Angle}
\end{eqnarray}
We'll see that the range resolution $\Delta R$ is related to the
bandwidth of the transmitter waveform and the receiver impulse
response.

Substituting the volume scatterer in the Radar Equation, we have the
monostatic deep target formula,
\begin{equation}
P_R = \frac{P_T G \sigma_v \Delta R}{(4\pi)^2 R^2}
\end{equation}
Notice that this has different dependence upon gain $G$ (appears once)
and range $R$ (appears squared, not quarticly).

\subsection{Line Targets}

There are also targets which fill the antenna beam in one direction,
but are small in the other.  Examples include radars looking obliquely
at ocean surface, and nearly all synthetic aperture radars (although
in the latter case the actual resolution needs more careful
description).

Without belaboring the point, we expect that the signal decreases like
$1/R^3$ for such ``line'' targets, and the antenna pattern interacts
in a somewhat more complicated way.

\subsection{Absorption}
So far we have assumed that the targets don't absorb any of the
radiation.  There are some subtle jargon issues about the definition
of scattering cross section when absorption is significant.

\subsection{Stealthy Vehicles}

Suppose a target with $\sigma = 100$ m$^2$ is ``just detectable'' at
$R = 100$ km.  By how much must the target reduce its scattering cross
section to avoid detection until $R = 10 $km?  Since there is a $R^4$
in the Radar Equation, decreasing the range by a factor of ten
increases the sensitivity by a factor of 10,000!  So the cross section
would have to be reduced to 0.01 m$^2$ --- an enormous engineering
challenge.

Two general techniques are available for reducing the scattering cross
section.
\begin{itemize}
\item Radar Absorbing Materials (RAM) which literally ``soak up'' the
  incident radiation.  Such materials tend to be bandwidth specific,
  and typically for microwave frequencies.
\item Minimization of backscatter cross section through shape
  control.  As most radars are monostatic, one can achieve stealth by
  controlling the shape so that the radiation scatters in directions
  other than the ``back'' direction.  This feature explains the
  peculiar shape of the so-called ``Stealth Fighter'' jet.
\end{itemize}
When vehicles wish to elude detection, an additional tactic is trajectory planning.


%%%%%%%%%%%%%%%%%%%


\section{Noise, Clutter, and Interference}

Although we now are able to estimate the size of the scattered signal
through the Radar Equation, we do not yet have a means to know how
detectable such signals are, which is to say, ``Is the power $P_R$ large
or small compared to other signals which are distracting, such as
noise, interference, jamming, and clutter?''

\subsection{Noise}

Very briefly, noise is a random signal that arises from finite
temperatures and stochastic processes in Nature.  There are many kinds
of noise processes, but for the immediate task at hand, we'll limit
our attention to ``white noise,'' which is characterized by a power
spectral density,
\begin{eqnarray}
\frac{dP_n}{df} &=& k_B T_{sys} \\
P_n &=& k_B T_{sys} B
\end{eqnarray}
where $k_B$ is Boltzmann's constant, $1.38\times 10^{-23}$
W/Hz/$^\circ$K; and $T_{sys}$ is the receiver system temperature,
$^\circ$K; and $B$ is the receiver bandwidth, Hz.  In the vicinity of Earth, the thermal noise
power spectral density as shown is essentially constant below 10 THz.   Thus the
spectrum is flat, hence the name ``white noise.''  

The system temperature can take some care to define and measure.  Very
loosely speaking, modern receivers can be built without difficulty
which achieve a $T_{sys} = 1000 ^\circ$K.

The receiver bandwidth is often set equal to the transmitter bandwidth
in a fairly precise way that will be discussed in the section on
matched filters\footnote{add ref to matched filters section}.  It can
often be well approximated by the inverse of the transmitter pulse
width.  However, for aerospace radar systems, dynamic range can be improved by making the
receiver bandwidth large compared to the transmitter bandwidth.

Given the system temperature and the receiver bandwidth, we can now
provide a formula for the Signal To Noise Ratio (SNR) which
\textit{does} provide a good measure of detectability, and in the
monostatic, point target case we have
\begin{equation}
\textrm{SNR} = \frac{P_r}{P_n} = \frac{P_t G^2 \lambda^2
  \sigma}{(4\pi)^3 R^4 \; k_B T_{sys} B}
\end{equation}

\subsection{Clutter} \index{clutter}

It would be a wonderful situation if we could arrange for the
transmitter energy to be deposited only upon the targets that we
wished to detect.  However, both uninteresting and interesting targets 
return power to the receiver.  It is very often the case that the
uninteresting signal power greatly exceeds the interesting signal
power.  Fortunately it is also often the case that the clutter has
Doppler content that is different from that of the desired targets
(e.g. airplanes) so that it may be possible to distinguish relatively
small target signals amongst otherwise enormous clutter signals.

Some clutter has Doppler content, however.  Weather signals include
wind, turbulence, and rain.  Long range radars looking for aircraft
may see turbulence in the ionosphere, and troublesome multipath
scattering from ocean surfaces, etc.

\subsection{Interference and Jamming}

Detection of target signals may be impeded by signals radiated from
electronic equipment or other transmitters.  The ``interference
power'' is that due to accidentally received signal; if the
interference is purposeful it is called ``jamming'' or Electronic Counter Measures (ECM).

\subsection{Chaff and Decoys} \index{chaff}

It is sometimes possible to create objects with low mass but large
radar cross section whose purpose is to befuddle a radar.  For
example, an airplane may drop small lengths of wire, or metallized
mylar film to create a large and distracting ``cloud'' from which the
aircraft can escape.  Similarly, one of the arguments against
Ballistic Missile Defense is that it is relatively easy to release
decoys from the ballistic missiles which distract from the task of
sensing the ``important'' missile payloads.

\subsection{ECM and ECCM} 

\index{EW} \index{EW!electronic warfare}
\index{EW!ECM} \index{EW!electronic countermeasures}

ECM and ECCM (``electronic counter-countermeasures'') encompass a large
progression of tactics for foiling radars and subsequently making
radars tougher to fool.  These techniques lie under the general rubric
of Electronic Warfare (EW) for which this text will have little to
say in detail.

%%%%%%%%%%%%%%

\section{Aliasing}

Radars sample the scattering volume in some fashion and pass their
data to digital signal processors.  Because the targets are sampled at
a finite rate, there is a possibility that they will be undersampled
in the Nyquist sense, which will impact the ability of the radar
operator to correctly estimate the Doppler shift.  In general one
compensates for this by increasing the sample rate.  However, radars
present an additional antagonistic kind of aliasing which may prevent
this.

\subsection{Range-Time Diagram}

The range-time diagram provides a simple means to figure out where the
radio pulses are in space time.  The idea is to plot time
horizontally, and range vertically; lines with positive slope are
outgoing waves, and those with negative slope are incoming radio
waves.  ``Events'' are located at points in space time, and include
the emission of a particular radio pulse, or the scattering of a
particular pulse.

\centerline{OBVIOUS FIGURE GOES HERE; see page 15 of scanned notes}

\subsection{periodic pulses}

The most common transmitter operation involves sending pulses
periodically.  If a target is 100 km distant, then sending pulses with
150 km separation (upon return), or 1 ms provides enough room for the
target echo to return before the subsequent transmitter pulse exits.

\centerline{FIGURE}

\subsection{range aliasing}

However it is also possible to send the transmitter pulses more
frequently.  If the target remained at 100 km, but the pulse period
was reduced to 0.5 ms, then a second transmitter pulse would leave
before the first echo returned.  The target would still appear delayed
by 100 km from the pulse which illuminated it, however it would also
appear to be only 25 km from the second pulse.  If all the pulses are
identical then there would be no obvious way to determine that the
target was at 25 km, 100 km, or any integer multiple of 75 km from 100
km.  

Targets whose distance exceeds the pulse spacing are said to be range
aliased. 

\subsection{Doppler aliasing}

\subsection{Overspread and Underspread targets}


% $Id: chapter3.tex,v 1.6 2005/02/01 01:16:12 jdsahr Exp $

\chapter{Receivers}

A receiver is the sensor which converts the scattered radio wave into
an electic signal which is suitable for analysis.

\section{Basic Receiver topologies}

\subsection{noncoherent}

power only detector.  Extremely simple.  Not very sensitive.  Can't do
Doppler processing (probably).

\subsection{direct conversion}

set LO to transmitter frequency.
advantages:
\begin{itemize}
\item simple
\end{itemize}

disadvantages
\begin{itemize}
\item LO pollution in the RX spectrum
\end{itemize}

\subsection{down conversion}

Two (or more) frequency conversion operations.
advantages
\begin{itemize}
\item higher performance
\item no LO in RF spectrum
\item access to good filter technologies
\end{itemize}

disadvantages
\begin{itemize}
\item higher cost
\end{itemize}


\subsection{analog nonlinearities in mixers and amplifiers}

High performance receivers and measuring equipment (e.g. vector
network analyzers) are frequently are performance-limited by
nonlinearities lurking in various places in the system.

Nonlinear behavior in components is not only difficult to accomodate;
it is difficult to even describe, because it arises in many different
ways.  

A ``simple'' kind of nonlinearity arises because of finite limits on
the power supply voltage for components --- clipping will eventually
set in.  More subtle are the weak, intrinsic nonlinearities offered by
amplifying and switching devices.

In ``mixers'' we hope to achieve an ideal analog multiplication.
Given two input signals $a(t), b(t)$ we would like to produce the
product signal $c(t) = a(t) b(t)$.  For example, if $a(t), b(t)$ are
a pair of sinusoidal signals with frequencies $f_a, f_b$, then the
output signal $c(t)$ will contain a pair of sinusoids with
frequencies $|f_a \pm f_b|$.

\subsubsection{Square-Law systems}

There is a basic topological problem with achieving an analog
multiply operation: two inputs and one output.  However there is a
trick that can be used to achieve multiplication if you have access
to ``addition'' and ``squaring.''

Consider two signals $a(t), b(t)$.  If their sum $a(t) + b(t)$ is
passed through a squaring operation you have
\begin{equation}
(a(t) + b(t))^2 = a(t)^2 + b(t)^2 + 2 a(t) b(t)
\end{equation}
The first two terms are not desired, but the third is.  It may be
possible to sufficiently suppress the first two by filtering. 

However, it is useful to consider the square of the difference, too.
\begin{equation}
(a + b)^2 - (a - b)^2 = 4 ab
\end{equation}
This is a perfect multiply!  However, achieving this in analog
circuits requires careful matching of the summing and squaring
circuits to achieve the mathematical cancellation.

\subsubsection{Junction Diodes}

Junction Diodes are frequently used for mixers (multipliers) because
of their well characterized nonlinear (exponential) current-voltage
characteristic.
\begin{eqnarray}
I &=& I_0\left( e^{V/V_{th}} - 1 \right) \\
I &=& I_0\left(\frac{V}{V_{th}}
+ \frac{1}{2}\left(\frac{V}{V_{th}}\right)^2 + \cdots \right)
\end{eqnarray}

\subsubsection{Diode Rings}

concept of double balanced

\subsubsection{Gilbert Cell}

active mixer/diffamp gadget

\subsubsection{Parametric}

use a varactor ...

\section{In Phase and Quadrature receivers}

Usually wish to use the slowest possible sampler to minimize data flow
through processor.  Lowest data rate is at baseband; achieve with some
sort of single or multi conversion receiver.

Basic problem --- can't tell sign of Doppler shift (spectrum of
real-valued signals are symmetric).

\begin{eqnarray}
s(t) &=& A \cos(2\pi(f_0 + \Delta f)t) \\
l(t) &=& \cos(2\pi f_0 t) \\
s(t)l(t) &=& A \cos(2\pi(f_0 + \Delta f)t)\cos(2\pi f_0 t) \\
&=& \frac{A}{2}\cos(2\pi \Delta f \,t) + \frac{A}{2}\cos(2\pi (2f_0)
t) \\
\textrm{LP}[s(t)l(t)] &=& \frac{A}{2}\cos(2\pi \Delta f \,t)
\end{eqnarray}
(LP indicates a ``low pass filter'' operation.)  In the last line,
notice that you wouldn't be able to notice if $\Delta f$ changed sign.
Same is true if $l(t) = \sin(2\pi f_0 t)$ --- the sign would change,
but that could be absorbed into a phase shift.

Idea: use both?


\subsection{The IQ signal}

If you use both cos and sin for the last detection, you get
\begin{eqnarray}
s(t) &=& A \cos(2\pi(f_0 + \Delta f)t) \nonumber \\
 &=& \frac{A}{2}\left( \exp(2\pi j(f_0 + \Delta f)t) + \exp(-2\pi
 j(f_0 + \Delta f)t) \right) \nonumber \\
l(t) &=& \exp(-2\pi j f_0 t) \\
s(t)l(t) &=& \frac{A}{2} \exp(2\pi j\Delta f)t) + 
\frac{A}{2} \exp(2\pi j(-2f_0 - \Delta f)t) \\
\textrm{LP}[s(t)l(t)] &=& \frac{A}{2}\exp(2\pi j \Delta f \,t)
\end{eqnarray}

Now the low pass signal has the unambiguous Doppler shift apparent.
This is a very convenient signal.  Although it may be puzzling to
think of at first, it really is as simple as having two wires, one
with the real part and one with the imaginary part.

\subsection{The Analytic Signal}

In many treatments of radar and radio communications systems the
theory of ``analytic signals'' and the Hilbert Transform are
introduced at this point.  

Although fascinating theory, knowing or not knowing that theory does
not preclude one from using or understanding the IQ receiver.  We
direct your attention to \cite{whalen,vantrees} if you would like to
learn more about this.  For the moment, suffice it to say that one can
formally create an ``analytic signal'' $\tilde{u}(t)$ from a real
signal $u(t)$ as follows:
\begin{eqnarray}
\tilde{u}(t) &=& u(t) + \jmath u_H(t) \\
u_H(t) &=& {\cal H}[u(t)] = u(t) \ast \frac{1}{\pi t} \\
       &=& \frac{1}{\pi} P \int_{-\infty}^\infty \frac{u(s)}{s - t}\; ds
\end{eqnarray}
Here $P\!\!\int$ means that the ``principle value'' of the integral is
taken, avoiding the singularity at $s=0$.  The (complex valued)
analytic signal $\tilde{u}(t)$ has the interesting property that it
has identically zero spectral content for negative frequencies. 

\subsection{analog IQ receivers}

In an analog IQ receiver, near the end of the downconversion chain,
the local oscillator will be split and phase shifted, with a cosine()
signal on one wire, and a sine() signal on the other.  The original
signal is then mixed against each of these oscillators and lowpass
filtered to produce the desired IQ baseband signal.

problems
\begin{itemize}
\item such mixers often (always?) have a DC offset, which leads to
  excess power in the zero frequency spectral channel.
\item different gain through I and Q channels leads to L-R spectrum
  pollution.
\item imperfect 90 degree phase shift between cosine and sine leads to
  L-R spectrum pollution
\item if the two LP filters in the I and Q channels are not perfectly
  matched, results in (complicated) L-R spectrum pollution.
\item Different ADCs could have different gains, nonlinearities ...
\end{itemize}

\section{Digital Receivers}

Modern digital receivers eliminate many of the limitations of
conventional analog IQ detectors, by implementing the last down
(IQ) conversion and subsequent low pass filtering entirely as digital
computations. 

advantages
\begin{itemize}
\item A single digitizer eliminates differential performance
\item $\exp(2\pi j f_0 t)$ is generated digitally --- ``perfect''
  quadrature is available
\item mixing/down conversion operations are digital, hence the I and Q
  channels are essentially perfectly matched.
\end{itemize}

disadvantages
\begin{itemize}
\item probably too expensive for very cheap receivers
\item ...?
\end{itemize}

Basically they are wonderful.


\section{The Matched Filter}

 When a transmitter emits a signal $g(t)$, it scatters from a target
and is detected by a receiver with an impulse response $h(t)$, so that
                   the detector output is $s_o(t)$,
\begin{eqnarray}
s_o(t) &=& h(t)\ast g(t) \\
       &=& \int h(\tau) g(t-\tau) d \tau
\end{eqnarray}
Of course, noise will be entering the receiver, too.  Letting $n(t)$
be the input noise, the output noise $n_o(t)$ is just 
\begin{eqnarray}
n_o(t) &=& h(t)\ast n(t) \\
       &=& \int h(\tau) n(t - \tau) d \tau
\end{eqnarray}
This is a mathematically handy step, because $n(t)$ has infinite
bandwidth.  This issue lead to the so-called ``ultraviolet
catastrophe'' early in the Twentieth Century.  If the power spectral
density is constant, and the noise has infinite bandwidth, then that
means that noise has infinite power!  The solution was provided by
Quantum Mechanics and the Planck Radiation Law, which showed that,
above a certain, temperature-dependent frequency, the noise power
spectral density falls of quite dramatically --- yielding finite noise
power at finite temperature, for infinite
bandwidth\footnote{buzzwords: Planck Radiation Law, Rayleigh-Jeans
  Law, Blackbody Radiation}.

So, how much signal power and noise power do we get?  Start with noise
power. 
\begin{eqnarray}
P_n &=& \langle n_o(t)^2 \rangle \\
    &=& \left\langle \left[ \int h(\tau) n(t-\tau) d\tau
    \right]^2\right\rangle \\
    &=& \left\langle \left[ 
    \int h(\tau) n(t-\tau) d\tau \int h(\tau') n(t-\tau') d\tau'
    \right]\right\rangle \\
    &=& 
    \int \int h(\tau) h(\tau') \langle n(t-\tau) n(t-\tau') \rangle
    d\tau' d\tau \\
    &=& \int \int h(\tau) h(\tau') N_0 \delta(\tau - \tau') d\tau' d\tau \\
    &=& N_0 \int \left| h(\tau)\right|^2 d\tau \\
P_n &=& \frac{N_0}{2} \int_{-\infty}^\infty \left| H(f) \right|^2 df
\end{eqnarray}

Now, what is the signal power $P_s$?
\begin{eqnarray}
P_s &=& (s_o(t))^2 \\
P_s &=& \left| \int h(\tau) g(t-\tau) \; d\tau \right|^2 \\
    &=& \left| \int H(f) G(f) e^{2\pi j f t} \; df \right|^2
\end{eqnarray}
Now, let's look at the Signal to Noise Ratio\index{signal to noise
ratio} (SNR\index{signal to noise ratio!SNR}),
\begin{equation}
\textrm{SNR} = \frac{P_s}{P_n} = 
\frac{\left| \int_{-\infty}^\infty H(f) G(f) e^{2\pi j f t} \; df \right|^2}
      {(N_0/2)  \int_{-\infty}^\infty \left| H(f) \right|^2 df}
\end{equation}
We would like to maximize this ration, but choosing the right time
$t$ to observe the signal, as well as the best impulse response
$h(t) \leftrightarrow H(f)$ given the transmitter waveform $g(t)$ to
maximize the ratio.  

There are a variety of ways to go about this, but the most
straightforward approach makes use of the Cauchy-Schwartz
Inequality\index{Cauchy-Schwartz Inequality}, which states that, for a
pair of (possibly complex valued) functions $A(p), B(p)$, we have
\begin{equation}
\left|\int A(p) B(p) \; dp \right|^2 \le \int |A(p)|^2 \; dp \int |B(p)|^2\; dp
\end{equation}
and, furthermore equality is achieved if and only if $A(p) =
kB^\ast(p)$ for some (possibly complex) scalar constant $k$.

In this case, by identifying $H(f) = A(f)$ and
$G(f)\exp(2\pi j f t) = B(f)$ we see immediately that the maximum SNR
is achieved when
\begin{eqnarray}
G(f) e^{2\pi j f t} &=& kH^\ast(f) \\
g(\tau) &=& kh^\ast(\tau)
\end{eqnarray}
for some constant $k$.  Also, the value of the maximum SNR can be
computed:
\begin{eqnarray}
\textrm{SNR}_{\textrm{max}} &=& 
  \frac{\int |H(f)|^2\; df  \; \int |G(f) e^{2\pi j f t}|^2 \; df}
      {(N_0/2)  \int\left| H(f) \right|^2 df} \\
  &=& \frac{\int |G(f) e^{2\pi j f t}|^2 \; df}{(N_0/2)} \\
  &=& \left. \frac{P_o(t)}{N_0/2} \right|_{\textrm{t=t}_m}
\end{eqnarray}
Here, $t_m$ is the time which maximizes the ratio ---
i.e. appropriate to the time-of-flight --- and $N_0$ is the one-sided
noise power spectral density $k_B T_{sys}$.

It is worth commenting on some limitations of the matched filter.
\begin{itemize}
\item The development above was predicated upon the whiteness of the
noise (flat spectrum).  However, a non-white noise spectrum can be
accomodated with a straightforward extension, which is to
``pre-whiten'' the detector.  See \cite{whalen-1971} for details.
\item The development above ignored the possibility of clutter.  The
matched filter is not necessarily optimal in the presence of
clutter.  However, in practice the matched filter is likely to be
near-optimal.
\item The development above ignored the possibility of several
targets.  This is a restatement of the problem of clutter.  Although
the matched is not necessarily optimal, it is very likely to be near
optimal.
\end{itemize}

\subsection{Square Pulse}

Transmit a square pulse $g(t)$ of duration $\tau$, matched filter
detect, get a triangle function $s_o(t) = g(t)\ast g(-t)$.

\centerline{figure}

This gives us a more sophisticated way of viewing range resolution
... 

\section{The Ambiguity Function}

\newcommand{\infint}{\int_{-\infty}^\infty}

Response of a (possibly matched) filter to a doppler shifted signal.

Suppose transmitter waveform $u(t)$, then matched impulse response is
$u^\ast(-t)$.  The output of matched filter is $u(t) \ast u^\ast(-t)$
evaluated at time $t$,
\begin{eqnarray}
u_o(t) &=& \infint u(t-s) u^\ast(-s) \; ds \\
&=& \infint u(t+s) u^\ast(s) \; ds \\
&=& \infint u(s) u^\ast(s-t) \; ds %\\
%&=& \infint u\left(s+\frac{t}{2}\right) u^\ast\left(s-\frac{t}{2}\right) \; ds 
\end{eqnarray}
Suppose we also Doppler shift the transmitter signal $u(t) \rightarrow
u(t) \exp(2\pi j \nu t)$, then response is
\begin{eqnarray}
\chi(\tau,\nu) &=& \infint u(s) e^{2\pi j \nu s} u^\ast(s-\tau)\;
ds  \label{e:AF-def}
\end{eqnarray}
Here $\chi(\tau,\nu)$ is the ``Ambiguity Function''.\index{ambiguity
function}  Note that several approximately equivalent forms are
defined by different authors.

The ambiguity function indicates the sensitivity of a matched filter
to point targets which are delayed by $\tau$ and Doppler shifted by
$\nu$ from the peak response at $\tau = 0$ and $\nu = 0$.

\subsection{Properties of the Ambiguity Function}

After \cite{levanon}

\begin{enumerate}
\item Normalization.  Using the Cauchy-Schartz inequality, we can
  rapidly show that 
\begin{equation} 
|\chi(\tau,\nu)| \le |\chi(0,0)| = 1
\end{equation}
where we scale the amplitude of $u(t)$ so that $\chi(0,0) = 1$.  The
implication is that the matched filter has its greatest sensitivity
when the targets are not relatively delayed, and have no Doppler
shift.
\item Conservation of Ambiguity.  
\begin{equation}
\int\!\!\infint |\chi(\tau,\nu)|^2 \; d\tau d\nu = 1
\end{equation}
This implies that, although the shape of the ambiguity surface can
change, the volume beneath that surface is preserved.  This has
tremendous ramifications for waveform design.
\item Symmetry.  
\begin{equation}
|\chi(\tau,\nu)| = |\chi(-\tau,-\nu)|
\end{equation}
\item Linear FM.  Application of a linear FM chirp distorts the
  ambiguity plane in a simple way:
\begin{equation}
\textrm{if}\;\; u(t) \rightarrow |\chi(\tau,\nu)| \;\;\; \textrm{then} \;\;\;
                u(t) \exp(\pi j \kappa t^2) \rightarrow |\chi(\tau,
                \nu + \kappa \tau)|
\end{equation}

\end{enumerate}

\subsection{Other Forms of the Ambiguity Function}

\subsubsection{Wigner Distribution}

A slightly different version of the ambiguity function is known as the
Wigner or Wigner-Ville Distribution. \cite{something}.
\begin{equation}
\chi_w(\tau,\nu) = \infint u(t + \tau/2) u^\ast(t-\tau/2) e^{2\pi j\nu t} \; dt 
\end{equation}
It is straightforward to show a simple relationship to $\chi$:
\begin{equation}
\chi_w(\tau,\nu) = e^{-\pi j \nu \tau} \chi(\tau, \nu)
\end{equation}
The Wigner Distribution has the nice property of complex symmetry, rather than
merely magnitude symmetry (property 3 above).
\begin{equation}
\chi_w(\tau,\nu) = \chi^\ast_w(-\tau,-\nu) 
\end{equation}
While $\chi_w$ is a little more awkward for computation, it may be more convenient for some proofs.
APPENDIX D elaborates on this topic.

\subsection{Fourier-Transformed form} 

Using the Fourier Transform relations found in appendix
\ref{fourier-transform}, page \pageref{fourier-transform}
we can show that
\begin{equation} \label{e:AF-def-alt}
\chi(\tau,\nu) = e^{-2\pi j \nu \tau} \infint \tilde{u}(f)\;
\tilde{u}^\ast(f-\nu) \;e^{2\pi j f \tau} \;df 
\end{equation}
If we look at the "wigner-form" we have
\begin{equation} \label{e:AF-def-alt}
\chi(\tau,\nu) = e^{-\pi j \nu \tau} \infint \tilde{u}(f + \nu/2)\;
\tilde{u}^\ast(f-\nu/2) \;e^{2\pi j f \tau} \;df  = e^{-\pi j \nu \tau} \chi_w(\tau,\nu)
\end{equation}
 


\subsection{Computation of the Ambiguity Function}

Depending upon whether one starts from \eqref{AF-def} or
\eqref{AF-def-alt} a few different approaches for computing the
Ambiguity Function are suggested.
\begin{itemize}
\item direct evaluation for a discrete set of $(\tau, \nu)$ pairs.
  The most efficient approach for just a few points. In general it
  takes a lot of computation to evaluate the AF, so this would be a
  naive approach to evaluate the entire $\tau, \nu$ plane.
\item Recognition of the the implied Fourier Transform in
  \eqref{AF-def}.  Requires recorrelation for each range $\tau$, but
  otherwise able to make use of the Fast Fourier Transform.  This is
  the basic approach used to calculate the cross ambiguity function
  for the Manastash Ridge Radar.  Sometimes referred to as a ``range
  first'' calculation.  A fast approach for computing a range-cut.
\item Recognition of the implied inverse Fourier Transform in
  \eqref{AF-def-alt}.  A completely valid approach which performs a
  correlation in the frquency domain.  A fast approach for computing a
  Doppler cut.
\end{itemize}


\subsection{Interpretation of the Ambiguity Function}

As defined, $|\chi(\tau,\nu)|$ is the magnitude of the response of the
matched filter to delayed and Doppler-shifted versions of the original
single.  It's a bit subtle, but $\chi$ has units of ``voltage''
(explain more), and thus the square, $|\chi(\tau,\nu)|^2$ (no relation
to ``chi-squared statistics,'' alas), has units of ``power'' or
``energy'' and is thus an ``observable.''

Since $|\chi|^2$ is strictly non-negative and integrable over the
$\tau, \nu$ plane, it can apparently serve as some sort of probability
density.  It represents a \textit{forward problem} \index{forward
  problem} in the following sense. 

\begin{quote}
If a (normalized) target is present at the range/Doppler coordinates
$\tau, \nu$, the output power from the undelayed, un-Doppler shifted
matched filter would be $|\chi(\tau,\nu)|^2$.
\end{quote}

Thus, if we specify a particular situation, we can predict, or
simulate what we would expect to happen.  However, with a radar, and
with most instruments, we actually wish to solve a different problem,
an \textit{inverse problem}:\index{inverse problem}  

\begin{quote}
Given a set of samples of the output of a matched filter, what
possible target positions and motions could have been responsible for
the measurements?
\end{quote}

In general, inverse problems are far more difficult to solve than
forward problesm.  Indeed it can be tremendously difficult to even
pose inverse problems in ways which permit their solution.  Also, the
most general form of solution of an inverse problem is not a set of
parameters, but rather a probability density for the desired
parameters\cite{tarantola}. 

Forward problem is a sequence of conditional probabilities:
\begin{equation}
p(voltage measured) = p(no-noise-output) * p(noise voltage)
no-noise-output-power = chi(tau,nu)*target_power(sigma,tau,nu)
\end{equation}

TO BE CONTINUED

\subsection{Ambiguity of some basic signals}

\subsubsection{Square Pulse} The ambiguity function for the basic
square pulse is readily evaluated, and given in the following formula:
\begin{equation}
|\chi(\tau,\nu)|^2 = \left\{\begin{array}{lcc}
\frac{1 - \cos(2\pi\nu(1-|\tau|))}{2\pi^2 \nu^2} &\textrm{for} & |\tau| < 1\\
0  & \textrm{for} & |\tau| > 0
\end{array}\right.
\end{equation}
For the cut $\nu = 0$ the expected autocorrelation function (ACF)
emerges, $|\chi(\tau,0)| = 1 - |\tau|$ for $|\tau| < 1$.  The
ambiguity function is plotted in \figref{square-pulse-ambig}.

\begin{figure}
\centerline{\epsfxsize=0.4\textwidth  \epsffile{ambig-square-pulse-wire.eps} \epsfxsize=0.4\textwidth \epsffile{ambig-square-pulse-cntr.eps}}
\caption{\label{f:square-pulse-ambig} The ambiguity function $|\chi|$
  for a unit duration square pulse, plotted as a surface (left) and
  with contours (right)}
\end{figure}

\subsubsection{Gaussian Pulse}

Although not used for radar applications (because of the amplitude
modulation), we can compute the ambiguity function of a gaussian
pulse. CHECK THIS!!!!
\begin{eqnarray}
u(t) &=& \sqrt{\frac{1}{\sqrt{2\pi \sigma^2}} e^{-t^2/(2\sigma^2)}} \\
|\chi(\tau,\nu)| &=& e^{-\tau^2/(8\sigma^2)} e^{-2 \pi^2 \sigma^2 \nu^2}
\end{eqnarray}
This is clearly the product of two gaussians, one in the delay
direction ($\tau$) and one in the Doppler direction ($\nu$).  The
product of the widths is $2\sigma/(2\pi\sigma) = 1/(2\pi)$, a constant.

\subsubsection{Periodic Pulse Train}

Consider a train of repeated pulses, each pulse with waveform $u_0(t)$
having ambiguity function $\chi_0(\tau,\nu)$.  We can describe the
sequence of $N$ pulses separated by interval $T$ as follows:
\begin{equation}
u(t) = \frac{1}{\sqrt{N}} \sum_{n=0}^{N-1} u_0(t - nT)
\end{equation}



\part{Radar Waveforms}

\chapter{Phase Codes for Underspread Targets}

Phase codes present a powerful means for modifying the ambiguity of
simple pulses.   The central idea is that the ``tent'' of a long
square pulse can be collapsed with an intricate phase coding, yielding
a more compact range resolution.  

For the most part, phase codes are considered from the viewpoint of
range clutter only, neglecting the Doppler component of the ambiguity
function.  These codes are very useful, but are most effective for
targets whose Doppler content evolves much less rapidly than the radar
waveform.

\section{Barker Codes}

In 1953 Barker\cite{barker-1953} developed a simple class of codes for
automatic synchronization of communications systems.  The basic idea
was to identify code sequences consisting of $+$ and $-$ phases which
produced a very sharp ACF peak with low delay sidelobes.  The best you
can do with binary phase codes is to accept sidelobes of $0$ (for even
lags) or $\pm 1$ for odd lags.  There are a finite number of such
codes, and all of them are given in \tabref{barker}.


\begin{table}
\caption{\label{t:barker} List of all Barker codes.  Note that if $c$
  is a Barker code, then so is $-c$, as well as the time reversal of $c$.}
\begin{center}
\begin{tabular}{rl}
length & pattern \\ \hline
2      & $++$, \rule{1em}{0mm} $+-$ \\
3      & $++-$ \\
4      & $+++-$,  \rule{1em}{0mm} $++-+$ \\
5      & $+++-+$ \\
7      & $+++--+-$ \\
11     & $+++---+--+-$ \\
13     & $+++++--++-+-+$
\end{tabular}
\end{center}
\end{table}

The limited number of such codes is an interesting mathematical
curiosity, but not really a practical one as we shall see later in the
discussion of random codes.  

The proof mechanisms for the non-existence of long Barker codes is
interesting, and (somewhat surprisingly) rests upon the periodic
correlation functions of these explicitly aperiodic codes.   Among
other things, it becomes clear that long codes of length $4n+1,
4n+2,4n+3$ are easily ruled out, and we learn that the odd length
codes must have range sidelobes of the same sign.  Codes of length
$4n$, however, require considerably more effort to prove their
existence (or nonexistence), partly because of their possible
alternating sidelobe signs.

It appears that the existence or non-existence of Barker codes of
length $4n$ remains open in the mathematical sense.  However, in the
practical sense the matter is closed.  It is known that there are no
additional Barker codes of length less than 200 or so.  A Barker code
of length 64 would be useful; a Barker code of length $2^{16}$ would
not be useful, because of transmitter bandwidth and other
considerations.

There are, however, longer codes with low range sidelobes that are
useful, such as the ``almost Barker Code'', of length 28, which has 9
range sidelobes of magnitude 2, the rest being 1 or 0
\cite{gray+farley-1973} \footnote{Radio Science, v8 no. 2, pp 123-131,
  feb 1973}
\begin{center}
$++-++-+--+---+---+---++++---$
\end{center}

% \clearpage

\section{Complementary Codes}

Consider the pair of 4-baud Barker Codes $+++-$ and $++-+$.  If you
compute and plot their autocorrelation functions, you'll notice that
the pattern of sidelobes has opposite sign (except for the central
peak).  Thus, if we interrogate a (constant) target first with one
pulse, then with the second, and add the two results point by point, a
``perfect'' pulse compression is obtained with zero amplitude
sidelobes.

Although 4-baud complementary codes are marginally useful, it is easy
to create much longer codes using the following theorem:

\begin{quote}
If $A$, $B$ are a complementary code pair, then the concatenated codes
$AB, A\bar{B}$ are also complementary (where the overbar indicates
changing the sign of each bit).
\end{quote}

Using this formula it is easy to create very long codes (and you can
observe that the 4-baud codes are the extension of the two 2-baud
codes).  However codes created in this fashion are usually suboptimal
in that their individual sidelobes can be fairly large, thus requiring
more precision in cancellation to achieve the perfect composite result.

Thus, in general some sort of search is required to find good, long
complementary codes.

\section{Polyphase Codes}

You can make polyphase versions of Barker Codes, too.  These can be
much longer than the biphase codes.  Six-phase codes are used in some
aerospace applications.

\section{M-sequences}

There is a large class of nearly perfect binary phase codes known as
M-sequences (``maximal length'') or PN-sequences (``pseudo noise'').
In contrast to the Barker Codes, the M-sequences exist for large
length (indeed for any length $M = 2^N - 1$, for positive integer
$N$).   The M-sequences share the nice autocorrelation property of the
Barker code in the cyclic as apposed to acyclic sense.  Thus when we
compute the range sidelobes via autocorrelation we have
\begin{equation}
\chi(\tau,0) = \sum_{k=0}^{k < N} c_k c_{k-\tau}
\end{equation}
and it is understood that either the subscripts of $c$ are computed
modulo $N$ or equivalently that the code is periodic with period $N$,
$c_k = c_{k+N}$.

The periodicity of the codes implies that they are 100 percent
duty-cycle --- this is a CW modulation scheme, implying that the radar
using these waveforms must be bistatic, although McEliece has
suggested that the waveforms actually work well aperiodically.

The construction of the codes is a direct, novel application of a
topic in abstract algebra, the Galois Fields arising from prime
polynomials modulo 2.  As a practical matter the codes are extremely
easy to generate in digital logic consisting of shift registers and
EXOR gates.  As an example, consider \figref{mseq1}, which shows a
shift register implementation associated with the polynomial $x^3 + x
+ 1 = 0$.

The shift register state stores the sequence $x^n$ which has been
reduced modulo the polynomial $x^3 + x + 1$, with coefficients
evaluated modulo two.  Thus we have
\begin{equation}
\begin{array}{lclclcl}
1 \\
x \\
x^2 \\
x^3 &\equiv& x+1 \\
x^4 &      & x^2 + x \\
x^5 &      & x^3 + x^2 &\equiv& x^2 + x + 1 \\
x^6 &      &           &      & x^3 + x^2 + x &\equiv&x^2 + 1 \\
x^7 &      &           &      &               &  &x^3 + x \equiv 1
\end{array}
\end{equation}
Thus we can see that $x^7 \equiv 1$ modulo $x^3 + x + 1$, which means
(of course) that $x^{7n} \equiv 1$.  The M-sequences are formed by
examining the coefficients of any bit; if we examine the 1s place, we
have $1001011$.  Mapping $1$ to $+$ and $0$ to $-$ we achieve a
biphase code with nearly ideal cyclic autocorrelation: the zero lag
has value $N$ and all other lags have value $-1$.

A little experimentation reveals that the sequence generated by
multiplying one M-sequence by a delayed copy is itself the same
M-sequence with a third shift.  Also, one can see that the M-sequences
have exactly one more $1$ than $0$.  Assembling these facts leads to
the strong correlation properties of the M sequences.

\begin{figure}
\centerline{\epsfxsize=0.6\textwidth \epsffile{mseq-sr.eps}}
\caption{\label{f:mseq1} This shift register circuit generates an M
sequence based on the polynmial $x^3 + x + 1 = 0$, which can be
rearranged (modulo $2$) to be $x^3 = 1 + x$.  The m-sequence arises
from the ``ones'' place of the shift register system, $\ldots 1001011\ldots$.}
\end{figure}

It is quite fortunate that prime polynomials modulo 2 exist for all
orders of polynomials.  Thus, unlike Barker codes arbitrarily long
M-sequences can be found (although not for all lengths).

The underlying polynomials are ``prime'' which means that they have no
factors except themselves and unity.  Of course, this factorization is
to be performed with coefficients modulo 2.  Thus, the polynomials $x$
and $x + 1$ are prime, although $x^2 + 1$ is not since it is the
square of $x+1$ (modulo 2).  A beginning table of prime polynomials
can be found in \tabref{ppolys}.  This table is not exhaustive, it's
just an indication of some primes.  None of the entries are really
large enough to use; in practice primes of order 10, giving sequences
of length 1023 begin to be useful.  A prime polynomial of order 48 is
embedded in the Global Positioning System algorithms; it is evaluated
each microsecond, and its period is thus (8.9 years).  The Qualcomm
CDMA protocol uses a generalization of m-sequences called ``Gold
codes'' \cite{uppala-1998,gold-1967,gold-1968} to permit several users
to share a single channel.

\begin{table}
\caption{\label{t:ppolys} A beginning table of prime polynomials
modulo 2.  The polynomials exist for all orders, and there are many of
them.  Polynomials of any particular order need only be tested by
primes of half order or smaller; also, a complete table can be built
using ``sieve'' methods.  As in the conventional natural primes, there
is no pattern which permits easy testing or generation of large primes.}
\centerline{\begin{tabular}{rl}
order & polynomials \\ \hline
1     & $x, x+1$ \\
2     & $x^2 + x + 1$ \\
3     & $x^3 + x + 1, x^3 + x^2 + 1$ \\
4     & $x^4 + x + 1, x^4 + x^3 + 1$ \\
5     & $x^5 + x^2 + 1, x^5 + x^3 + 1, x^5 + x^3 + x^2 + x + 1, \ldots$ \\
6     & $x^6 + x + 1, x^6 + x^3 + 1, \ldots$
\end{tabular}}
\end{table}

\section{Random Codes}

\subsection{Completely Random Codes}

If we contemplate the idea that good radar codes have an
autocorrelation function which resembles a delta function, we may
recall that white random processes also have a compact autocorrelation
function.  

A random sequence of $N$ $\pm 1$ bits will yield an autocorrelation
function with a zero lag height of $N$, with the typical sidelobe at
lag $n \ne 0$ having height $1/\sqrt{N-n}$.  Although this is not
particularly good, neither is it particularly bad.  Furthermore, such
codes will generally have good performance in the whole range-Doppler
ambiguity plane.

\subsection{Exhaustive Search}

In principle, one could simply examine all codes of length $N$ to find
which perform the best.  However, this method rapidly collapses in the
face of exponential reality, and the opportunity for finding long
codes through exhaustive search is probably vanishingly small.

\subsection{Optimized Random Codes}

One can get improved random codes by creating many of them, and
picking those which yield the best ambiguity properties.  Suppose we
invent a metric $M$ for goodness of a code $x_n$ with autocorrelation
$R_{xx}(n)$, such as 
\begin{equation}
M(x) = \sum_n |R_{xx}(n)|
\end{equation}
We can then create a population of codes and pick the best performer
from them.

Something a bit stronger can be done as well.  If we consider a metric
for code performance, then the shape of the manifold over all possible
codes is almost certainly a very bumpy, irregular surface.  Although
there is little structure locally, globally there will be some
noticable trends.  For example, the waveform which is all $+$ will
have a poor ambiguity, as will the waveform of alternating $+-$
phases.  If we can create a search process which preserves
``randomness'' but includes ``nearness'' then we may be able to find
good codes without wasting time in regions where only poor codes
exist.

Techniques which automate this process are widely used, and fall under
the general rubric of ``stochastic optimization.''  What one gives up
in analytic and definitive results can be regained through performance
specifications which would be difficult or impossible to express
analytically.  Sahr and Grannan\cite{sahr+grannan-1993} used the
technique of ``simulated annealing'' to design several long binary
phase codes with interesting correlation properties which would have
been extremely difficult to find by construction or exhaustive
search.  In particular they described a ``hurricane'' code for which
Barker-like performance was desired, but only for the innermost lags,
and ``quasi-orthogonal'' code pairs which individually behaved like
Barker codes, but whose cross-correlations were small.



\chapter{Frequency Modulation Codes for underspread targets}

\section{Linear FM}

\section{Costas Signals}



\include{chapter-CorrelationCodes}

\part{Estimation}

% $Id: chapter-Estimation.tex,v 1.4 2007/07/12 19:42:46 jdsahr Exp $

\chapter{Estimation and Detection}

{\Huge W}e constantly find that the radar systems are impacted by
signals which behave in a manner that we think of as random.  There
are several origins of ``randomness'' which are important to be aware
of.  The scatter of radio waves from macroscopic objects
(e.g.~aircraft) is a completely deterministic process; however our
state of knowledge about the location and pose of the aircraft is
limited.  The problem of radars, after all, is not to know where a 
target is, and to predict the electromagnetic energy which will be
detected.  Instead, the problem of radars is to observe that some
radiation has been detected, and to infer properties about the target
which caused the detectable scatter.

In some cases the scattering process itself has a stochastic nature.
For example, ion sound wave turbulence in the E region is
deterministic in the sense that it obeys the Navier-Stokes equations
augmented by Maxwell's Equations.  However, as a strongly turbulent
process, it is often conveniently described by statistical language
and measures.

Finally, at the statistical mechanics level, various sources of noise
compete with the signals that we wish to intercept, ultimately
providing implacable limits on detection and accuracy.

To that end, we will introduce some basic ideas about statistics and
estimation associated with random processes.


\section{Random Processes}

A random process is one whose behavior is random as opposed to
deterministic.  By ``process'' we frequently mean the electromagnetic
field intercepted by an antenna, or the voltage present at the
terminal of the antenna, or the modified voltage-like signals that
appear as the signal makes its way from the antenna through the digestive system
of the radar.  Eventually the continuous time voltage will be
converted to discrete time, sampled versions.  Those samples will be 
subject to various linear and nonlinear transformations, all of which,
as functions of random data, are themselves random.

However, the randomness of the signal is not without limit; signals may be correlated 
in time, significantly reducing the amount of randomness (in a way that can be
made precise) a central
goal of operation of radars is to constrain the randomness, and to
assign meaning and likelihood to the results.

We describe random processes somewhat differently than we describe
other signals; although we use some of the same words, the meaning may
have some subtle technical differences.  It is important to realize
that all of our work is motivated by a desire to sensibly observe the
environment, which obeys sensible laws of physics, and is, therefore,
not ``random'' in any strict sense.  However, the ensemble of atoms
and molecules is so large that statistical arguments are very
appealing, and frequently accurate.  

Let's begin with a single continuous time random process $x(t)$.
Unless otherwise stated explicitly, such random processes will be
presumed to have field units (volts) as opposed to power or energy
units.  Also, these random processes will be presumed to be zero mean,
unless otherwise stated explicitly.  This is not a large limitation
for the case of radar, for which the received signal voltages are
inevitably zero mean.  Instrumentation may introduce DC offsets but
the scattered signals are inevitably zero mean.

We will frequently need to compute functions of data when handling
data drawn from random processes.  Often the function of the data will
be some form of average.  We define the average value of the random
process in two different ways with a subtle but extremely important
technical distinction.

\deff{Time Average}{
An average may be formed from actual data, in either a continuous or
discrete sense; as in
\begin{equation}
\bar{x} = \frac{1}{T}\int_T x(t) \; dt \equiv \frac{1}{N} \sum_N x[n]
\end{equation}}

\deff{Ensemble Average}{
An idealized or theoretical average, formed by making a single
independent measurement an infinite number of times; such averages may
be computed by taking an appropriate moment of the probability density
function of the random variable. }

For example, the mean $\bar{x}$, or expected value of a random
variable $x$ can be computed from the probability density function
$f(x)$ as follows:
\begin{equation}
\bar{x} = \int x f_x(x) \; dx
\end{equation}

As a practical matter, all parameter estimation proceeds through time
averages, because the ensemble average is never available.  It happens 
that statistical theorems are frequently more easily proven in terms
of ensemble average, however.  Thus it would be extremely convenient if
the ensemble average and the time average were numerically equivalent.
This turns out to be the usual case, although the issue is subtle.

\deff{Ergodicity}{
A random process is said to be ergodic if its time
  averages are equivalent to its ensemble averages.}

It is possible to imagine stochastic processes whose properties change
as a function of time --- getting more powerful, changing their spectral
shape, etc. --- such variation is straightforward to handle with
ensemble averages.  However, in the real world we have to work with
time averages, so if a process is changing its power during the time
we are collecting measurements, ... well, it's a problem.  Thus we
have the notion of \textit{stationarity}:

\deff{Stationarity}{
A stochastic process is said to be stationary if all of its moments
are constant in time.}

\deff{Mutual, or Joint Stationarity}{Several stochastic processes are
  said to be jointly stationary if their joint moments are constant in
  time.}

It is very important to distinguish the concept of ``stationarity''
from ``constancy.''  A coin toss is a non constant process yielding
heads or tails; it is completely unpredictable.  On the other hand, a
coin toss is ``stationary'' since the odds of head or tail does not
change with time ... unless someone does something drastic to the coin 
(such as bending it).

The property of stationarity is very powerful; in radar we frequently
need only a relaxed version of stationarity, namely ``wide sense
stationary''

\deff{Wide Sense Stationarity}{A process is said to be
wide sense stationary if its second order moments are constant in
time.}



\subsection{Expected Value, or Expectation} 

\label{s:expectedvalue}
Very closely related to the notion of Ensemble Average is the Expected
Value. 

\deff{Expected Value}{The expected value of a random process is
  its ensemble average.  We usually speak of the expected value
  of a function of one or more random variables, which may or may not
  be identically distributed.  The expected value is indicated by
  angle brackets, e.g.
\begin{displaymath}
\langle g(x_1, x_2, \ldots x_N) \rangle = \int \!\!\int \!\! \ldots \int g(x_1,
x_2, \ldots x_N) \; f_{x_1} f_{x_2} \cdots f_{x_N} dx_1 dx_2 \cdots dx_N
\end{displaymath}
}

There are some simple calculus rules that one may apply to expressions
with the expected value operator.  In the following rules $x, y$ are
random variables, and $a, b$ are scalar contants:
\begin{enumerate}
\item Constants may be pulled out of expectations:
  \begin{displaymath}
  \langle a x \rangle = a \langle x \rangle
  \end{displaymath}
\item The expectation is linear: 
  \begin{displaymath}
  \langle ax + by \rangle = a \langle x \rangle + b \langle y \rangle
  \end{displaymath}
\item The expectation distributes across products of independent
  random processes,
  \begin{displaymath}
  \langle x y \rangle = \langle x \rangle \langle y \rangle \;\;\;
  \iff f_{xy}(x, y) = f_x(x) f_y(y)
  \end{displaymath}
\item The expectation does not commute with most operators, e.g.
  \begin{eqnarray*}
  \left\langle \frac{1}{x} \right\rangle &\ne & \frac{1}{\langle x \rangle} \\
  \langle \sin(x) \rangle &\ne & \sin(\langle x \rangle)
  \end{eqnarray*}
\end{enumerate}

\subsubsection{Isserlis Theorems for zero mean real gaussian random variables}

Additional rules for manipulation of expectations are available when
encountering products of gaussian random variables.  We will restrict
our attention to zero mean gaussian random variables, which will
suffice for our needs.  The extensions needed for non-zero mean
gaussian processes are straightforward, but tedious.

In the case of real valued, zero mean, gaussian random variables, we have
\begin{equation}
\langle xyzw \rangle = 
  \langle xy \rangle \langle zw \rangle + 
  \langle xz \rangle \langle yw \rangle + 
  \langle xw \rangle \langle yz \rangle
\end{equation}
The rule generalizes to higher order products, ``The expected value of
the product of an even number of gaussian random variables is equal to
the sum over all permuations of the second order products.''  Thus for
the fourth order case there are $3 = \left({}^4_2\right)$ terms, for
the sixth order case there are $15 = \left({}^6_2\right)$ terms, etc.
\begin{equation}
\langle xyzwpq \rangle = \langle xy \rangle \langle zwpq \rangle +
 \langle xz \rangle \langle ywpq \rangle +
 \langle xw \rangle \langle yzpq \rangle +
 \langle xp \rangle \langle yzwq \rangle +
 \langle xq \rangle \langle yzwp \rangle
\end{equation}
Here each of the fourth order expectations can now be expanded using
the earlier theorem.  It is usually quite tedious to work out
statistics of order higher than 6.

Note that the expected value of a product of an odd number of zero
mean real gaussian random variables is necessarily zero.

It is worth noting a useful special case:
\begin{equation}
\left\langle x^4 \right\rangle = 3 \left\langle x^2 \right\rangle^2 =
3 \sigma^4
\end{equation}
$\sigma^2$ is the usual variance of the random variable $x$.

\subsubsection{Isserlis Theorems for zero mean complex gaussian random variables}



A similar set of theorems applies for complex valued, zero mean
gaussian random variables of a kind which arise naturally in
communications theory and radar remote sensing.  We will describe
those signals elsewhere [[[ WHERE? ]]] for the present observe that
the real and imaginary parts are independent and identically
distributed zero man gaussian random variables.  Although the Isserlis
Theorems are a bit more difficult to state, in practice they are
easier to apply.


\begin{equation}
\left\langle \prod_{n=1}^N x_n \;\prod_{m=1}^M y_m^\ast \right\rangle
= \left\{
 \begin{array}{ll} 
    0 & \textrm{if} \; N \ne M \\ 
    \sum_P \prod_{n=1}^N \langle x_p y^\ast_{P(n)} \rangle &
    \textrm{if} \; N = M
 \end{array} \right.
\end{equation}
The latter sum is over all unique ways of picking pairs from $x$ and
$y$; it's best to illustrate with an example.
\begin{equation}
\langle xy^\ast zw^\ast \rangle = 
  \langle xy^\ast \rangle \langle zw^\ast \rangle + 
  \langle xw^\ast \rangle \langle zy^\ast \rangle 
\end{equation}
Again, some special cases,
\begin{eqnarray}
\langle |x|^2 \rangle &=& \langle x\,x^\ast\rangle = \sigma^2 \\
\langle |x|^4 \rangle &=& 2\langle x\,x^\ast \rangle^2 = 2 \sigma^4
\\
\langle |x|^6 \rangle &=& 6\sigma^6 \\
\langle |x|^{2N} \rangle &=& N!\, \sigma^{2N}
\end{eqnarray}

An additional special case is worth mention for field quantities (i.e. voltage):
\begin{equation}
\left\langle xx \right\rangle = \left\langle x_r^2 - x_i^2 +j x_r x_i \right\rangle = 0 + j0
\end{equation}
In other words $\left\langle x_r^2\right\rangle = \left\langle x_i^2\right\rangle$ and $\left\langle x_r x_i\right\rangle = 0$.  This perhaps
surprising result can be derived from the Kramers-Kronig [[ CITE ]] relations which result from the causality property that we expect 
in the physics of this universe.


\section{Estimation}

Now we couple basic ideas from probability with the need to convert
data into useful estimates.  Again, several definitions are coming.

\deff{Estimator}{
An estimator is a function of data which produces an
estimate\footnote{a pure function of data is also known as a
  \textit{statistic}} of a parameter (or perhaps several parameters).
For example, the following is an example of an estimator for the
arithmetic mean:
\begin{displaymath}
\frac{1}{N}\sum_{n=0}^N x_n
\end{displaymath}}

\deff{Estimate}{An estimate\footnote{\textit{ibid}} is the numerical
  result of applying an estimator to data, thus $\hat{x}$ is an
  estimate of the arithmetic mean: 
\begin{displaymath}
\hat{x} = \frac{1}{N}\sum_{n=0}^N x_n
\end{displaymath}
An estimate is a random variable, since it is a mathematical
function of random variables.}

Note that in summations we will use the convention that the lower
limit is inclusive (typically starting at zero) and the upper limit is
exclusive (last term is $N-1$, not $N$).

\subsection{properties of estimators}

Estimators may be good or bad, they may give answers which are always
wrong, but for which the amount of wrongness improves (decreases) with
more data.

\deff{Unbiased Estimator}{
An unbiased estimator has the important property that the expected
value of its estimates is equal to the parameter being estimated; or
mathematically, 
\begin{displaymath}
\langle \hat{x} \rangle = \langle x \rangle
\end{displaymath}}

\begin{example}{The usual estimator of the arithmetic mean is
    unbiased.}  To show this, we use the properties in section
  \ref{s:expectedvalue} on page \pageref{s:expectedvalue}:
\begin{displaymath}
\langle \hat{x} \rangle = 
\left\langle\frac{1}{N}\sum_{n=0}^N x_n \right\rangle =
\frac{1}{N}\sum_{n=0}^N \langle x_n \rangle =
\frac{1}{N}\sum_{n=0}^N \langle x \rangle = 
\frac{N}{N}\langle x \rangle = \langle x \rangle
\end{displaymath}
\end{example}

A weaker but nevertheless useful property is ``consistency.''

\deff{Consistent Estimator}{A biased estimater is said to be
  consistent if the bias vanishes as the amount of data increases
  without limit.}

\begin{example}The ``obvious'' estimator of the standard deviation is
biased but consistent:
\begin{displaymath}
\hat{SD}_0 = \left(1 - \frac{1}{N}\right) \sigma^2 
\end{displaymath}
The expected value differs from the ideal result by the
factor $(1 - 1/N)$.  However, in the limit $N\rightarrow\infty$ the bias
vanishes.
\end{example}

Additional properties of estimators are expressed in terms of the
``goodness'' of estimators, which is typically expressed in terms of
the variance of the estimate:

\deff{Estimator Variance}{The variance of an estimator is defined as
  the expected value of the square of the data less the true mean: 
\begin{displaymath}
\textrm{Var}\; \hat{p} = \left\langle \left|\hat{p} - \langle\hat{p} \rangle 
\right|^2 \right\rangle
\end{displaymath}
When the estimator is real-valued, expanding the square and using the linearity of the
$\langle\cdot\rangle$ operator we find equivalently that
\begin{displaymath}
\textrm{Var} \;\hat{p} = \langle\hat{p}^2\rangle - \langle \hat{p} \rangle^2
\end{displaymath}
This latter form is sometimes more convenient for calculations.
When the estimator is complex-valued, the appropriate expressions is
\begin{displaymath}
\textrm{Var} \;\hat{p} = \langle\hat{p}\hat{p}^\ast \rangle - \langle \hat{p} \rangle \langle \hat{p}^\ast \rangle
\end{displaymath}
The latter expression works for both real and complex valued estimators.
}


By computing the estimator variance, we can compare the quality of two estimators, noting which
produces the smaller variance, and leading to the following additional 
definitions.

\deff{sufficient estimator}{An estimator is said to be sufficient if
  no other estimator produces a smaller variance.}

\deff{efficient estimator}{An estimator is said to be efficient if it
  produces the smallest possible variance for that parameter.  Every
  efficient estimator is sufficient, but not all sufficient estimators
  are efficient.}

Sufficiency and Efficiency may appear to be the same thing, but they
are different.  Efficiency is defined in terms of the ``theoretical
limit'' defined by the Cramer-Rao Lower Bound (see Appendix
\ref{s:cramer-rao}), which is a hard theoretic bound that no estimator
can surpass.  However, efficient estimators may not exist, in which
case there will still be many estimators that are ``pretty good'' in
the sense of being better than most.

It is important to note that estimator variance is neither the only, nor even necessarily the best
measure of estimator quality.  For example, if the data are known to contain outliers, estimators
using order statistics will generally have poorer (larger) variance, but will nevertheless 
behave (significantlly) better.

\subsection{Advice on manipulation of estimators}
\label{s:estimator-advice}

There is rarely any difficulty in evaluating first order estimators
(the mean) for bias.  There is, however, ample opportunity for
mischief.  In particular, consider the variance of an estimate of the
sum of squares of data, that is, the power in a signal:
\begin{eqnarray}
\hat{P}_1 &=& \frac{1}{N} \sum_{n=0}^N |x_n|^2 \\
\langle \hat{P}_1 \rangle &=& P \;\;\;\;\;\; \textrm{(unbiased)} \\
\textrm{var} \hat{P}_1 &=& \langle \hat{P}^2 \rangle - \langle\hat{P}\rangle^2
= \langle \hat{P}^2 \rangle - P^2 \\
&=& \left\langle \left(\frac{1}{N} \sum_{n=0}^N |x_n|^2 \right)^2 \right\rangle
- P^2 \\
&=& \left\langle \left(\frac{1}{N} \sum_{n=0}^N |x_n|^2 \right)
  \left(\frac{1}{N} \sum_{m=0}^N |x_m|^2 \right) \right\rangle - P^2 \\
&=& \left\langle \frac{1}{N^2} \sum_{n=0}^N \sum_{m=0}^N |x_n|^2
  |x_m|^2 \right\rangle - P^2 \\
&=& \frac{1}{N^2} \sum_{n=0}^N \sum_{m=0}^N \left\langle |x_n|^2
  |x_m|^2 \right\rangle - P^2 \label{e:varp}
\end{eqnarray}
The subtlety here is in the use of a second ``dummy'' index $m$ for
the repeated power of the data $x$; it would be tempting, and wrong,
to use the index $n$ twice.  A similar issue arises in the computation
of powers of integrals.  It is reasonable to assert that the standard
notation for sums and integrals contains the notational weakness of
attaching an excess of meaning to the index of the sums, or the
variable of integration, which is not truly warranted.

We'll complete the evaluation of \eqref{varp} below.

\section{Estimation of Signal Power}

Let us consider the usual estimator of signal power,
\begin{equation}
\hat{P}_1 = \frac{1}{N}\sum_{n=0}^N |x_n|^2
\end{equation}
Having declared an estimator, one should then determine whether the
estimator is biased:
\begin{eqnarray}
\langle\hat{P}_1\rangle &=& 
\left\langle \frac{1}{N}\sum_{n=0}^N |x_n|^2 \right\rangle = \frac{1}{N}\sum_{n=0}^N \langle |x_n|^2
  \rangle \\
&=& \frac{1}{N}\sum_{n=0}^N P = P 
\end{eqnarray}
Thus the usual estimator of power is unbiased, since
$\langle \hat{P}_1\rangle = P$.  

Having established the (absence of) bias, we should then analyze the
variance of the estimator,
\begin{eqnarray}
\textrm{var} \;\hat{P}_1 &=& \langle |\hat{P}_1|^2 \rangle - P^2
\label{e:P1var-a} \\
&=& \left\langle \left(\frac{1}{N}\sum_{n=0}^N |x_n|^2 \right) 
  \left( \frac{1}{N}\sum_{m=0}^N |x_m|^2 \right) \right\rangle -
P^2 \label{e:P1var-b} \\
&=& \frac{1}{N^2}\sum_{n=0}^N \sum_{m=0}^N \langle|x_n|^2  |x_m|^2
\rangle - P^2
\end{eqnarray}
In going from \eqref{P1var-a} to \eqref{P1var-b} we have used the
``advice'' in section \ref{s:estimator-advice} about properly
manipulating powers of sums.

In order to proceed, we need to distinguish between the cases of the
signal $x$ having real or complex value, and employ the appropriate
version of the Isserlis Theorems.

\subsection{real processes}

In the event that the signal $x_n$ is real-valued and gaussian, we have

\begin{eqnarray}
\textrm{var} \; \hat{P}_1 &=& \frac{1}{N^2}\sum_{n=0}^N \sum_{m=0}^N
\langle x_n x_n x_m x_m \rangle - P^2 \\
&=& \frac{1}{N^2}\sum_{n=0}^N \sum_{m=0}^N \left( 
  \langle x_n x_n \rangle \langle x_m x_m \rangle +
  \langle x_n x_m \rangle \langle x_m x_n \rangle +
  \langle x_n x_m \rangle \langle x_m x_n \rangle \right) \nonumber \\
& & \rule{0.6\textwidth}{0mm} - P^2 \\
&=&\frac{1}{N^2}\sum_{n=0}^N \sum_{m=0}^N \left( 
  P^2 + 2 R_{nm}R_{mn} \right) - P^2 \\
\textrm{var} \; \hat{P}_1 &=& \frac{2}{N^2}\sum_{n=0}^N \sum_{m=0}^N 
  \left( R_{nm}R_{mn} \right) = \frac{2}{N^2}\sum_{n=0}^N \sum_{m=0}^N 
  |R_{nm}|^2 \label{e:P1var-raw}
\end{eqnarray}
Here $R_{nm}$ is the expected value of the correlation $x_n x_m$.

In order to proceed, we need to have some model for the correlation
$R_{nm}$.  In two particular cases we can produce concrete
expressions, but in general one must evaluate the implied double sum.

\subsubsection{white samples}

In the simplest case where the samples are white, we have $\langle x_n
x_m \rangle = 0$ if $n \ne m$; that is
\begin{eqnarray}
R_{nm} &=& \delta_{nm} \langle x_n x_m \rangle \\
      &=& \delta_{nm} \langle x_n x_n \rangle \\
      &=& \delta_{nm} P
\end{eqnarray}
In this case, only the $n = m$ terms of the double sum in
\eqref{P1var-raw} survive, and we have
\begin{eqnarray}
\textrm{var}\;\hat{P}_1 &=&  \frac{2}{N^2}\sum_{n=0}^N \sum_{m=0}^N 
  \delta_{nm} P^2 \\
  &=& \frac{2}{N^2} \sum_{n=0}^N P^2 \\
\textrm{var}\;\hat{P}_1 &=& \frac{2}{N} P^2
\end{eqnarray}
This result expresses the expected result, that the variance decreases
with the number of observations.

\subsubsection{periodically sampled, stationary}

In the event that $x$ is a real, stationary process, we have $R_{nm} = 
R(n-m) = R(m-n)$.  Also, correlation functions of physical processes
nearly always decay in some fashion over a characteristic time
$\tau_{ac}$.  In the (frequent) occurance that we periodically observe
a stationary process for a length time which is large compared to the
correlation time, we can write a different expression for the
variance of $\hat{P}_1$; for convenience define $r(n) = R(n)R(-n)$:

\begin{eqnarray}
\textrm{var}\;\hat{P}_1 &=&  \frac{2}{N^2}\sum_{n=0}^N \sum_{m=0}^N  
R(n-m) R(m-n) \\
&=& \frac{2}{N^2} \sum \left[ \begin{array}{cccc}
r(0) & r(1) & \ldots & r(N-1)  \\
r(1) & r(0) & \ldots & r(N-2) \\
\vdots & \vdots & \ddots & \vdots \\
r(N-2) & r(N-3) &  & r(1) \\
r(N-1) & r(N-2) & \ldots & r(0) 
\end{array} \right]
\end{eqnarray}
\begin{eqnarray}
  &=& \frac{2}{N^2} \left(NP^2 + 2(N-1)r(1) + 2(N-2)r(2) + \cdots 2r(N-1)\right)\\
  &=& \frac{2P^2}{N} + \frac{4}{N^2}  \sum_{n = 1}^N (N-n)r(n) \\
  &=& \frac{2P^2}{N} + \frac{4}{N}  \sum_{n = 1}^N r(n) - \frac{4}{N^2}  \sum_{n = 0}^N n r(n)
\end{eqnarray}
This result collapses to the result for white samples when $R(m > 0) =
0$, as expected.  In general there we can't say much about that sum, other than to perform it.
However in the particular case where $|R(m)| = Pe^{-|n|\kappa}$ so that $|R(n)|^2 = r(n) = P^2
e^{-2|n|\kappa}$ the sum can be evaluated,
\begin{eqnarray}
\textrm{var}\;\hat{P}_1 &=& 
\frac{2P^2}{N}\left(2\sum_{n=0}^N e^{-2\kappa |n|} - 2\sum_{n=0}^N ne^{-2\kappa |n|} - 1\right)
\end{eqnarray}
\begin{equation}
\textrm{var}\;\hat{P}_1 = \frac{2P^2}{N} \left(2\frac{1 - e^{-2\kappa N}}{1 - e^{-2\kappa}} + 
2\frac{(N-1) e^{-2\kappa(N+1)} - Ne^{-2\kappa N} - e^{-2\kappa}}{N(1 - e^{-2\kappa})^2} - 1\right) 
\end{equation}
In the event that the process decorrelates rapidly, namely $\kappa \gg 1$, all the exponential terms vanish, and we have
\begin{equation}
\textrm{var}\;\hat{P}_1 = \frac{2P^2}{N} 
\end{equation}
which is the expected result for white data.

It is often the case that the signal is observed for a time that is significantly longer than the correlation 
time $1/\kappa$, such that 
\begin{equation}
\kappa N \gg 1 \;\;\;\; \textrm{yet} \;\;\;\; \kappa \sim 1
\end{equation}  
When this is the case, we can make the approximation
\begin{equation}
e^{-2\kappa N} \rightarrow 0
\end{equation}
while retaining the $\exp(-2\kappa)$ terms:
\begin{eqnarray}
\textrm{var}\;\hat{P}_1 &=& \frac{2P^2}{N} \left(\frac{2}{1 - e^{-2\kappa}} - \frac{  2e^{-2\kappa}}{N(1 - e^{-2\kappa})^2} - 1\right) \\
&=& \frac{2P^2}{N} \left(\frac{1 + e^{-2\kappa}}{1 - e^{-2\kappa}} - \frac{  2e^{-2\kappa}}{N(1 - e^{-2\kappa})^2}\right)
\end{eqnarray}

If we let $\kappa \rightarrow 0$, implying long correlation time, the result simplifies further,

\begin{equation}
\textrm{var}\;\hat{P}_1 = 2\frac{P^2}{\kappa N} \left(1 - \frac{1}{2\kappa N}\right) 
\end{equation}
or more simply,
\begin{equation}
\textrm{var}\;\hat{P}_1 = 2\frac{P^2}{\kappa N} 
\end{equation}

recalling that $\kappa N$ is still large, even if $\kappa$ is small.  In this case $\kappa N < N$  becomes the effective number of samples.

\subsection{white, complex processes}

Returning to \eqref{P1var-raw} we now consider the case of white
complex processes, using the appropriate complex Isserlis Theorem,
\begin{eqnarray}
\langle x_n x_n^\ast x_m x_m^\ast \rangle &=& 
\langle x_n x_n^\ast \rangle\langle x_m x_m^\ast \rangle +
\langle x_n x_m^\ast \rangle \langle x_m x_n^\ast \rangle \\
&=& P^2 + R_{nm}R_{mn} \\
&=& P^2 + |R_{nm}|^2
\end{eqnarray}
The last line is subtle.  When we construct an I/Q representation of a real
process, the result is necessarily complex valued, but because it is a representation of
a real process, its power spectrum is strictly non-negative definite.  This means that the 
autocorrelation function of a real process represented in the I/Q sense has Hermite 
symmetry, namely $R(\tau) = R^\ast(-\tau)$. 

Therefore, for complex white samples we have
\begin{equation}
\textrm{var}\;\hat{P}_1 = \frac{1}{N} P^2
\end{equation}

The variance of the complex form appears to be half that of the real
form; however, this can be explained by the observation that complex
samples contain twice as much data; viewed in that sense, the
variances are equivalent.  However, in the case of radar, because both
real and imaginary (I and Q) samples can be made at the same time, and
are statistically independent, there is an advantage to
working with complex samples.

\subsubsection{periodically sampled, stationary, and long observed}

By similar reasoning in the real case, we can immediately conclude
that the variance of the $\hat{P}_1$ estimator applied to complex
data, periodically sampled, and observed for a long time, is
\begin{equation}
\textrm{var}\;\hat{P}_1 = \frac{1}{N} \sum_{m=0}^{\infty} |R(m)|^2
\end{equation}

If we again model the correlation function as $|R(n)| = Pe^{-\kappa|n|}$,
then we have the elegant result for $\kappa N \gg 1$,
\begin{eqnarray}
\textrm{var}\;\hat{P}_1 \approx \frac{P^2}{N} \;\;\; \textrm{for} \;\;\; \kappa \gg 1 \\[1em]
\textrm{var}\;\hat{P}_1 \approx \frac{P^2}{\kappa N} \;\;\; \textrm{for} \;\;\; \kappa \ll 1
\end{eqnarray}
Again, when $\kappa < 1$ then $\kappa N < N$ represents the number of independent samples of the random process.  For $\kappa = 1$ the variance
formula becomes
\begin{equation}
\textrm{var}\;\hat{P}_1 \approx 1.037 \;\frac{P^2}{N} \;\;\; \textrm{for} \;\;\; \kappa = 1 \\[1em]
\end{equation}
showing that the process has effectively decorrelated for $\kappa$ as small as unity.


\subsection{Signal and Noise power together}

Let us now consider the case in which a particular sample $x_m$ is
the sum of a desired signal $s_m$ and noise sample $n_m$.  Upon
applying $\hat{P}_1$ to this data, the expected value is given as
follows,
\begin{eqnarray}
\left\langle \hat{P}_1 \right\rangle &=& 
\left\langle \frac{1}{M} \sum_{m=0}^M |s_m + n_m|^2 \right\rangle \\
&=& \frac{1}{M} \sum_{m=0}^M \langle |s_m + n_m|^2 \rangle \\
&=& \frac{1}{M} \sum_{m=0}^M \langle (s_m + n_m)(s_m + n_m)^\ast \rangle \\
&=& \frac{1}{M} \sum_{m=0}^M \left( 
   \langle s_m s_m^\ast \rangle + 
   \langle s_m n_m^\ast \rangle + 
   \langle n_m s_m^\ast \rangle + 
   \langle n_m n_m^\ast \rangle \right) \label{e:psn-a} \\
&=& \frac{1}{M} \sum_{m=0}^M \left( 
   \langle s_m s_m^\ast \rangle + \langle n_m n_m^\ast \rangle
 \right) \label{e:psn-b} \\
&=& \frac{1}{M} \sum_{m=0}^M \left( P_s + P_n \right) \\
\left\langle \hat{P}_1 \right\rangle &=& P_s + P_n
\end{eqnarray}
In going from \eqref{psn-a} to \eqref{psn-b} we have used the fact
that signal and noise samples are uncorrelated.

Thus the estimator $\hat{P}_1$ applied to data containing signal and
noise power is an unbiased estimate of the sum of those powers (but a
biased estimate of either the noise power or signal power alone).

\subsubsection{variance of signal plus noise estimate}

We must now characterize the quality of the estimator by considering
its variance.
\begin{equation} 
\textrm{var}\;\hat{P}_1 = \frac{1}{M^2} \sum_p \sum_q 
\left\langle (s_p + n_p)(s_p + n_p)^\ast(s_q + n_q)(s_q + n_q)^\ast 
\right\rangle - \langle \hat{P}_1 \rangle^2
\end{equation}
In order to evaluate this expression, we need to consider all 16 terms
of the expanded product.  Doing so will illustrate the combinatoric
flavor which frequently emerges in such calculations.
\begin{itemize}
\item Products which include an odd number of signal terms $s$ (and
  necessarily an odd number of noise terms $n$) will have an expected
  value which is identically zero.  This is because application of
  Isserlis' Theorems will always include a product which is the
  correlation of a signal and noise, and such correlations always vanish.
  \item Products which include either products $\langle qq \rangle$ or
  $\langle q^\ast q^\ast \rangle$ are also necessarily null.
\end{itemize}
With these conditions in mind, the variance of $\hat{P}_1$ applied to
signal and noise is as follows:
\begin{eqnarray} 
\textrm{var}\;\hat{P}_1 &=& \frac{1}{M^2} \sum_p \sum_q 
  \langle s_p s_p^\ast s_q s_q^\ast \rangle
+ \langle n_p n_p^\ast n_q n_q^\ast \rangle + \nonumber \\
& & \langle s_p s_p^\ast n_q n_q^\ast \rangle
    + \langle n_p n_p^\ast s_q s_q^\ast \rangle
    + \langle s_p n_p^\ast n_q s_q^\ast \rangle
    + \langle n_p s_p^\ast s_q n_q^\ast \rangle \nonumber \\
& & - P_s^2 - P_n^2 - 2P_s P_n 
\end{eqnarray}
Tracing through the steps carefully we have
\begin{eqnarray}
\textrm{var}\;\hat{P}_1 &=& \frac{1}{M^2} \sum_p \sum_q 
  \langle s_p s_p^\ast s_q s_q^\ast \rangle + 
  \langle n_p n_p^\ast n_q n_q^\ast \rangle + \nonumber \\ 
& & P_s P_n + P_s P_n + P_s P_n \delta_{pq} + P_s P_n \delta_{pq} - \nonumber \\
& & P_s^2 - P_n^2 - 2 P_s P_n \label{e:complexsplusn} \\
\textrm{var}\;\hat{P}_1 &=& \frac{1}{M^2} \sum_p \sum_q 
  (1 + \delta_{pq}) (P_s^2 + P_n^2) + 2P_s P_n \delta_{pq} 
 - P_s^2 - P_n^2 \nonumber \\
\textrm{var}\;\hat{P}_1 &=& \frac{1}{M^2} \sum_p \sum_q 
  \delta_{pq} (P_s^2 + P_n^2 + 2P_s P_n) \\
\textrm{var}\;\hat{P}_1 &=& \frac{1}{M} (P_s^2 + P_n^2 + 2P_s P_n) \\
\textrm{var}\;\hat{P}_1 &=& \frac{1}{M} (P_s + P_n)^2 
\end{eqnarray}
We could have presented the last result immediately, however it was
perhaps a useful exercise to work the steps out in detail. 

When the signal is not white, we return to \eqref{complexsplusn}:
\begin{eqnarray}
\textrm{var}\;\hat{P}_1 &=& \frac{1}{M^2} \sum_p \sum_q 
  P_s^2 + |R_{pq}|^2 + (1 + \delta_{pq})P_n^2 + \nonumber \\
& & \;\;\; 2(1 + \delta_{pq}) P_s P_n - P_s^2 - P_n^2 - 2 P_s P_n \\
&=& \frac{1}{M^2} \sum_p \sum_q 
  |R_{pq}|^2 + \delta_{pq} P_n^2 + 2 \delta_{pq} P_s P_n 
\end{eqnarray}

It is useful to make the following definition.

\deff{correlation time}{A stationary process with
  autocorrelation function $R_{pq}$ (or $R(\tau)$) has a correlation
  time defined empirically as
\begin{eqnarray*}
  \kappa &=& \lim_{M \rightarrow \infty} \frac{1}{M} \frac{1}{P^2} \sum_{pq}
  |R_{pq}|^2 \;\;\;\; \textrm{discrete case} \\
  \tau_{ac}&=&  \frac{1}{P^2}
  \int_0^\infty |R(\tau)|^2 d\tau \;\;\;\; \textrm{continuous case}
\end{eqnarray*}
Note that $\kappa \ge 1$ for all discrete stochastic processes.}

With this definition, we can write a very general and useful
expression for the variance of $\hat{P}_1$:
\begin{eqnarray}
\textrm{var}\;\hat{P}_1 &=& \frac{1}{M} \left( \kappa P_s^2 + P_n^2 +
  2 P_s P_n\right) \\
 &=& \frac{1}{M} \left(P_s + P_n\right)^2 + \frac{\kappa - 1}{M} P_s^2
 \\
&=& \frac{P_t^2}{M}\left(1 + \frac{\kappa - 1}{1 +
    \frac{1}{\textrm{SNR}^2}}\right) \;\; \ge \frac{P_t^2}{M} \\
&=&  \frac{P_t^2}{M}\left(\frac{1 + \kappa \,\textrm{SNR}^2}{1 +
    \textrm{SNR}^2}\right) \;\; \ge \frac{P_t^2}{M}
\end{eqnarray}
In the last expression we have made use of the definition $P_t = P_s +
P_n$, as well as the signal to noise ratio SNR $=P_s/P_n$.  Note that
there is a hard lower bound on the variance.

Expressed in words, the variance of $\hat{P}_1$ is always at least as
great as $P_t^2/M$; it can be enhanced above this value if the signal
process has a large correlation time and a sufficiently large signal
to noise ratio. 

\section{Nonlinear Estimators}

Some parameters that we wish to estimate are intrinsically nonlinear,
and thus require care in their analysis.  We will introduce the
challenge with an artificial yet instructive example: the
``quietness'' estimator, which is intended to produce an estimate of
the inverse power in a signal. 

\subsection{Inverse Power}

Let us invent and analyze an estimator $\hat{Q}$ of the inverse power,
with the goal that it be unbiased, namely 
\begin{displaymath}
\langle \hat{Q} \rangle = \frac{1}{P}
\end{displaymath}

Two different estimators suggest themselves:
\begin{eqnarray}
\hat Q_1 &=& \frac{1}{N} \sum_n \frac{1}{|x_n|^2} \\
\hat Q_2 &=& \frac{1}{\hat P_1} = \frac{1}{\frac{1}{N} \sum_n |x_n|^2}
\end{eqnarray} 
We shall see that $\hat Q_1$ is unbiased, but terrible, in the sense
of having enormous variance.  On the other hand, $\hat Q_2$ is biased,
but consistent.

\subsubsection{evaluation of $Q_1$}

Let us first find (or at least approximate) the expected value of
$\hat{Q}_1$.   Notice that
\begin{displaymath}
y = \frac{1}{x} \;\;\; \rightarrow \;\;\; \frac{1}{|x|^2} = yy^\ast
\end{displaymath}
and $y$ is no longer a gaussian random variable.   We'll need to
compute
\begin{eqnarray}
\left\langle \left| \frac{1}{x} \right|^2 \right\rangle &=&\includegraphics[]{../../Desktop/D-R9CF7XUAAXrrs.jpg}

\frac{1}{\sqrt{2\pi} \sigma} \int_{-\infty}^\infty
\left|\frac{1}{x}\right|^2 \exp\left(-\frac{x^2}{2\sigma^2}\right) \;
dx \\
&=& \frac{2}{\sqrt{2\pi} \sigma} \int_{0}^\infty
\left|\frac{1}{x}\right|^2 \exp\left(-\frac{x^2}{2\sigma^2}\right) \;
dx\\
&=& \frac{2}{\sqrt{2\pi} \sigma^2} \frac{1}{\sqrt{8}} \int_{0}^\infty
s^{-3/2} e^{-s} \; ds \\
&=&\frac{2}{\sqrt{2\pi} \sigma^2} \frac{1}{\sqrt{8}}
\Gamma\left(-\frac{1}{2}\right) \\
&=&\frac{2}{\sqrt{2\pi} \sigma^2} \frac{1}{\sqrt{8}}
(-2\sqrt{\pi})\\
&=& \frac{-1}{\sigma^2}
\end{eqnarray}
Notice that the expected value of the power is negative.  This is (of
course) impossible.  It has to do with the formal definition of the
Gamma Function for negative argument values.  We could fool around
with this for a while, but essentially what is happening is that the
mean value is hard to find reliably.

More meaningfully we can do a quick numerical experiment (in
Matlab/Octave) to show that this is a hopeless estimator:
\begin{verbatim}
> x = randn(1,10000);
> z = 1./(x.*x);
> mean(z);
1264.1
> median(z);
2.2114
\end{verbatim}
This shows that the mean of the estimator is dramatically removed from
the value that we want, and even the median (which is known to be a
robust\footnote{see appendix XXX} estimator of central tendency) is
pretty far away from the unbiased value that we hope for, 1.0.

Thus $\hat{Q}_1$ is a terrible estimator of the inverse power.

\subsubsection{evaluation of $Q_2$}

On the other hand $\hat{Q}_2$ will show a bit more promise, although
we will have to wade through some algebra to figure out how it
behaves.  Our approach will require a perturbation expansion/Taylor
series approach.

Consider the following decomposition of the data,
\begin{eqnarray}
\langle x_n x_m^\ast \rangle &=& R_x(n-m) \\
x_n x_m^\ast &=& R_x(n-m) + \epsilon_{nm} \\
\textrm{where} \;\;\; \langle \epsilon_{nm} \rangle &=& 0 \\
\textrm{and} \;\;\; x_n x_n^\ast &=& P + \epsilon_{nn} 
\end{eqnarray}
Now we can start to evaluate $\hat{Q}_2$.
\begin{eqnarray}
\hat{Q}_2 &=& \frac{1}{\hat{P_1}} \\
&=& \frac{1}{\frac{1}{N}\sum_p x_p x_p^\ast} \\
&=& \frac{1}{\frac{1}{N}\sum_p (P + \epsilon_{pp})} \\
&=& \frac{1}{P + \frac{1}{N}\sum_p \epsilon_{pp}} \\
&=& \frac{1}{P}\left(\frac{1}{1 + \sum_p \frac{\epsilon_{pp}}{NP}}\right) \\
&=& \frac{1}{P}\left(1 -  \sum_p  \frac{\epsilon_{pp}}{NP} + 
\sum_p \sum_q  \frac{\epsilon_{pp}}{NP} \frac{\epsilon_{qq}}{NP} +
\cdots \right) \\
%
\left\langle \hat{Q}_2 \right\rangle &=& \frac{1}{P} \left( 1 - 
\sum_p  \frac{\langle \epsilon_{pp}\rangle}{NP} + 
\sum_p \sum_q  \frac{\langle \epsilon_{pp} \epsilon_{qq}\rangle}{(NP)^2} +
\cdots \right) \\
&=& \frac{1}{P} \left( 1 + \sum_p \sum_q  \frac{\langle \epsilon_{pp} \epsilon_{qq}\rangle}{(NP)^2} +
\cdots \right) 
\end{eqnarray}
Now we need to figure out that four corellation ...
\begin{eqnarray}
\langle \epsilon_{pp} \epsilon_{qq} \rangle &=& 
\langle (x_p x^\ast_p - P) (x_q x^\ast_q - P) \rangle \\
\langle \epsilon_{pp} \epsilon_{qq} \rangle &=& 
\langle (x_p x^\ast_p x_q x^\ast_q) - P x_p x^\ast_p - P x_q x^\ast_q + P^2 \rangle \\
&=& \langle (x_p x^\ast_p x_q x^\ast_q)\rangle - P^2 \\
&=& \langle x_p x^\ast_p \rangle \langle x_q x^\ast_q \rangle
+\langle x_p x^\ast_q\rangle \langle x_q x^\ast_p \rangle - P^2 \\
&=& P^2 + |R_{pq}|^2 - P^2 \\
&=& |R_{pq}|^2
\end{eqnarray}
So we can now evaluate the expected value of $\hat{Q}_2$:
\begin{eqnarray}
\hat{Q}_2 &=& \frac{1}{P} \left( 1 + \sum_p \sum_q \frac{\langle
    \epsilon_{pp} \epsilon_{qq}\rangle}{(NP)^2} + \cdots \right)
\nonumber \\
&=& \frac{1}{P} \left(1 + \sum_p \sum_q \frac{|R_{pq}|^2}{(NP)^2} +
  \cdots \right) \\
&=& \frac{1}{P} \left(1 + \sum\sum_{p=q} \frac{|R_{pq}|^2}{(NP)^2} +
+ 2\sum\sum_{p>q} \frac{|R_{pq}|^2}{(NP)^2} 
+ \cdots \right) \\
&=& \frac{1}{P} \left(1 
+ \frac{1}{N} 
+ 2\sum\sum_{p>q} \frac{|R_{pq}|^2}{(NP)^2} + \cdots \right) 
\end{eqnarray}

Let's use as a model for $|R_{pq}| \sim P\exp(-|p-q|/\kappa)$. 
\begin{eqnarray}
\hat{Q}_2 &=& \frac{1}{P} \left(1 + \frac{1}{N} 
+ \frac{2}{N} e^{-2/\kappa}\sum_{p=0}^\infty e^{-2p/\kappa} + \cdots
\right) \\
 &=& \frac{1}{P} \left(1 + \frac{1}{N} 
+ \frac{2}{N} \frac{e^{-2/\kappa}}{1 - e^{-2/\kappa}} + \cdots \right)
\\
&=& \frac{1}{P} \left(1 + \frac{1}{N} \frac{1 + e^{-2/\kappa}}{1 -
    e^{-2/\kappa}} + \cdots \right)\\
&=& \frac{1}{P} \left(1 + \frac{1}{N} \coth(1/\kappa) +  \cdots  \right) \\
&\sim& \frac{1}{P}\left\{\begin{array}{ll}
1 + \frac{\kappa}{N} & \kappa \gg 1 \\[0.5em]
1 + \frac{1}{N} & \kappa \ll 1
\end{array} \right. \\
&\sim& \frac{1}{P}\left(1 + \frac{1 + \kappa}{N}\right) \;\;\; \forall
\; \kappa
\end{eqnarray}
Thus, $\hat{Q}_2$ is biased, but at least well behaved.  Although it
it biased, it is \textit{consistent}, in that the amount of bias
vanishes as $N \rightarrow \infty$.

We haven't shown this, but the next correction in the expressions
above scales as $1/N^2$.

\vspace{1em}

\noindent \textbf{A rule of thumb: reasonable estimators will
work well if the target is observed for times much larger than the
target decorellation time.}


\subsection{Signal to Noise Ratio}

Now we can approach the Signal to Noise Ratio estimator.
\begin{eqnarray}
\widehat{\textrm{SNR}} &=& \frac{\hat{P}_1|_{s+n} - \hat{P}_1|_n}{\hat{P}_1|_n} \\
&=& \frac{\hat{P}_1|_{s+n}}{\hat{P}_1|_n} - 1
\end{eqnarray}
If the numerator and denominator use independent samples ($N$ for the
numerator, and $M$ for the denominator), then we can compute expected
value of the SNR estimator quickly,
\begin{eqnarray}
\left\langle \widehat{\textrm{SNR}} \right\rangle &=& \left\langle
  \hat{P}_1|_{s+n} \right\rangle \left\langle
  \frac{1}{\hat{P}_1|_n}\right\rangle - 1 \\
&=& (P_s + P_n) \frac{1}{P_n}(1 + \frac{1}{N}) - 1 \\
&=& \frac{P_s + P_n}{P_n}\left( 1 + \frac{1}{N}\right) - 1 \\
&=& \left(\frac{P_s}{P_n} + 1 \right)\left(1 + \frac{1}{N}\right) - 1
\\
&=& \frac{P_s}{P_n}\left(1 + \frac{1}{N}\right) +\frac{1}{N} \\
&=& \textrm{SNR}\left(1 + \frac{1}{N}\right) +\frac{1}{N} \\
&=& \textrm{SNR} + \frac{1}{N}(1 + \textrm{SNR}) 
\end{eqnarray}
Thus, the SNR estimator (as presented) is always biased; however it is
consistent because the bias vanishes as $N\rightarrow \infty$.

\subsubsection{variance of the SNR estimator}

We should still estimate the quality of the SNR estimator by computing
its variance:
\begin{eqnarray}
\textrm{var } \widehat{\textrm{SNR}} &=& \left\langle
  \widehat{\textrm{SNR}} \;\;\widehat{\textrm{SNR}}\right\rangle -\left (\textrm{SNR} + \frac{1}{N}(1 + \textrm{SNR})\right)^2
\end{eqnarray}

\section{Estimation of Correlation}

Estimates of correlation will prove to be extremely useful for radar
remote sensing of deep, overspread targets.  Let's begin with a basic
estimator of correlation of Inphase/Quadrature data $x[n]$ which is
periodically sampled with sample period $\tau$.
\begin{eqnarray}
\hat{R}_{xx}(m) &=& \frac{1}{N} \sum_{p=0}^{N-1} x[p] x^\ast[p-m] \\
\left\langle \hat{R}_{xx}(m) \right\rangle &=& \frac{1}{N}
\sum_{p=0}^{N-1} \left\langle x[p] x^\ast[p-m] \right\rangle \\
&=& \frac{1}{N} \sum_{p=0}^{N-1} R_{xx}(m) \\
&=& R_{xx}(m)
\end{eqnarray}
Thus, the autocorrelation estimator is unbiased.  Unlike the power
estimator $\hat{P}_1$ the autocorrelation estimator is also unbiased
in the presence of noise; let $x[p] = s[p] + n[p]$:
\begin{eqnarray}
\left\langle \hat{R}(m) \right\rangle &=& \frac{1}{N}
\sum_{p=0}^{N-1} \left\langle (s[p] + n[p]) (s[p-m] + n[p-m])^\ast \right\rangle \\
&=& \frac{1}{N} \sum_{p=0}^{N-1} \langle s(p)s^\ast(p-m)\rangle
+\langle s(p)n^\ast(p-m)\rangle +  \\ \nonumber 
& & \;\;\;\;\;\;\;\;\;\;\;\;\langle n(p)s^\ast(p-m)\rangle +\langle n(p)^\ast(p-m)\rangle\\
&=&  \frac{1}{N} \sum_{p=0}^{N-1} R(m) + 0 + 0 + 0 \\
&=& R_{xx}(m)
\end{eqnarray}
The latter result occurs because
\begin{itemize}
\item Signals are (always) uncorrelated with noise.
\item White noise is uncorrelated with delayed versions of itself.
\end{itemize}

Now we should estimate the variance of the correlation estimate.
\begin{eqnarray}
\textrm{var}\hat{R}(m) &=& \left\langle \hat{R}(m)\hat{R}^\ast(m) -
  R_{xx}(m) R^\ast_{xx}(m)\right\rangle \\
&=& \left\langle \hat{R}(m)\hat{R}^\ast(m) \right\rangle - R_{xx}(m)
R^\ast_{xx}(m) \\
&=& \frac{1}{N^2}
\sum_{p=0}^{N-1}\sum_{q=0}^{N-1} \left\langle x[p] x^\ast[p-m]
  x^\ast[q] x[q-m] \right\rangle - R_{xx}(m) R^\ast_{xx}(m) \\
&=& \frac{1}{N^2}\sum_{p=0}^{N-1}\sum_{q=0}^{N-1} \langle x[p]
x^\ast[p-m]\rangle\langle x^\ast[q] x[q-m] \rangle +\langle x[p]
x^\ast[q]\rangle\langle x^\ast[p-m] x[q-m] \rangle \nonumber \\
& &- R_{xx}(m) R^\ast_{xx}(m) \\
&=&  \frac{1}{N^2}\sum_{p=0}^{N-1}\sum_{q=0}^{N-1}  \left(R_{xx}(m)
R^\ast_{xx}(m) - \delta_{pq} |R(0)|^2\right) -  \left(R_{xx}(m)
R^\ast_{xx}(m) \\
&=& \frac{1}{N}P^2 = \frac{1}{N} \left(P_s^2 + P_n^2\right)
\end{eqnarray}

Thus, the variance of the correlation estimate depends upon the total
power in the signal plus the noise.  This estimate is correct for the
case in which there is a single isolated target.  However this is not
the correct result for the case of the ``double pulse'' waveform.
\typeout{BLHABAALSDKJAFSALKFJSAFLKJASLJFLJALKFJDLASFLJK}

\chapter{Detection}

Useful operation of a radar requires the detection of known signals.
These signals may be simple pulses, pure tones, or intricately coded,
or even random.  These signals are known, or, in the case of passive
radar, there should be some independent way of deducing what they were 

If the transmitted signal is $s(t)$, then we will be interested in
discovering the amplitude $a$, delay $\tau$, and Doppler Shift $v$ in
the presence of noise $n(t)$, in a received signal.  Of course
additional signal properties are of interest, such as the angle of
arrival, angular image, but these properties are derived from spatial
information --- a vector of signals --- and we'll begin by considering
a simple scalar signal.

\section{Narrowband Signal}

We will find it very convenient to work with a narrowband
representation of radar signals.  As a practical matter the actual
radiated signal will be a real-valued signal of the following form
\begin{equation} \label{e:realmodel0}
y(t) = s(t) \cos[2 \pi f_0 t + \phi(t)]
\end{equation}
where the signal has some amplitude modulation $a(t)$ and phase
modulation $\phi(t)$ and a carrier frequency $f_0$.  An alternative
formulation would be 
\begin{equation}
y(t) = \Re\left[\tilde{s}(t) \exp(2\pi j f_0 t)\right]
\end{equation}
where we now include the phase modulation into a complex amplitude
$s(t)$.  This permits very convenient incorporation of amplitude and
phase changes, delays, and Doppler shifts:
\begin{equation}
y(t) = \Re\left[a \tilde{s}(t - \tau) \exp(2\pi j f_0 (t-\tau)) \exp(2\pi j \nu
  (t-\tau))\right]
\end{equation}
In fact, we can let the signal $\tilde{y}(t)$ be complex,
\begin{equation} \label{e:analyticmodel0}
\tilde{y}(t) = a\;\tilde{s}(t - \tau) \exp(2\pi j f_0 (t-\tau)) \exp(2\pi j \nu  (t-\tau))
\end{equation}
One problem with this signal is that the imaginary part of
$\tilde{y}(t)$ has not been specified.  The usual response is to
constrain the imaginary part of $\tilde{y}(t)$ so that the whole
signal spectrum has identically zero content for negative frequencies.
Such a signal is denoted the ``analytic signal'' in communications
literature \cite{vantrees,whalen,helstrom}.  Given a real signal
$s(t)$, the Hilbert Transform creates the appropriate imaginary part
to create an analytic signal.

\begin{factoid}
The Matlab function {\protect\texttt{hilbert(x)}} returns the analytic signal
$\tilde{x}$ given a real signal $x$.
\end{factoid}

Suppose that the spectrum of $\tilde{S}$ is entirely bounded within
$\pm f_s$, and further suppose that $f_s \ll f_0$.  Then such a
signal is said to be ``narrowband'' because the spectral width is much
less than the center frequency.  The Nyquist Rate of the signal $s$ is
$2 f_s$, the full bandwidth of the signal, including the negative
spectral content.

\begin{factoid} All radio broadcast signals are narrowband signals.
   FM channels have (full) bandwidth 200 kHz at a center frequency of
   100 MHz; AM channels have a (full) bandwidth of about 20 kHz at 1
   MHz.  In both cases the signal bandwidth is only 2 percent of the
   center frequency.
\end{factoid}

In the discussion of bandwidth above, the Doppler frequency shift
$\nu$ was not mentioned.  In general, the maximum Doppler Shift
$\nu_{max}$ is usually smaller than the radar signal bandwidth $f_s$.
Exceptions do exist, such as pure CW illuminators with very low
bandwidth.

\section{The IQ Receiver}

When we wish to receive and process narrowband signals, a particularly
convenient receiver is available which produces a signal which is very
nearly ``analytic.''  It is possible to build finite impulse response
(FIR) processors which can produce an approximation of the Hermite
Transform, but a somewhat different method is used in practice.

First, we can rewrite \eqref{analyticmodel0} as follows:
\begin{equation}
y(t) = \Re\{ \tilde{y}(t) \} = 
            \frac{1}{2}\left[\tilde{y}(t) + \tilde{y}^\ast(t)\right] 
\end{equation}
Now, let us ``down convert'' by multiplying $y(t)$ by the conjugate
carrier signal $\exp(-2\pi j f_0 t)$,
\begin{eqnarray}
e^{-2\pi j f_0 t} y(t) &=&
\frac{1}{2} \left(
  a \tilde{s}(t - \tau) e^{2\pi j \nu t} e^{- 2\pi j (f_0 + \nu) \tau}
  \; + \right.  \nonumber
\\
& & \;\;\;\;\;\; \left.a^\ast \tilde{s}^\ast(t - \tau) e^{-2\pi j \nu t} 
   e^{2\pi j(f_0 + \nu) \tau} e^{-4\pi j f_0 t} \right)
\end{eqnarray}
Note that the first term has only low frequency content, while the
second term has been shifted to high (negative) frequency.  If we
apply a low pass filter which slices off the high (negative)
frequencies, we will be left with the ``baseband signal'' $y_b(t)$,
\begin{eqnarray}
y_b(t) &=& {\cal L}\{e^{-2\pi j f_0 t} \tilde{y}(t)\} \\
       &=& \frac{1}{2}a \tilde{s}(t - \tau) e^{2\pi j \nu (t-\tau)} e^{- 2\pi
       j f_0 \tau} 
\end{eqnarray}
Thus the signal $y_b(t)$ is directly proportional to the complex
modulation, scaled by the scattering amplitude $a$ and a phase factor
$\exp(-2\pi j f_0 \tau)$, which in general can not be determined but
will usually be unimportant anyway.

Now the baseband signal $y_b(t)$ is complex valued, and it would be
reasonable to object.  Nevertheless this signal is perfectly
realizable by simply permitting $y_b(t)$ to be decomposed into real
and imaginary parts, each part being real valued.

NEED FIGURE, NEED MORE TEXT

\section{The Matched Filter}

Let us initially consider the detection of a signal which is neither
delayed nor Doppler shifted, but whose amplitude $a$ is unknown.  The
received signal will be processed by conversion to baseband, and then
passed through a filter with some impulse response $h(t)$.  The result
is a new signal $r(t)$,
\begin{eqnarray}
r(t) &=& h(t) \ast (y(t) + n(t)) \\
     &=& \int_0^\infty h^\ast(t') [y(t-t') + n(t-t')]\; dt'
\end{eqnarray}





\chapter{Introduction}

\noindent {\Huge P}hase codes present a powerful means for modifying
the ambiguity of simple pulses.  

\section{Barker Codes}

\newpage

\section{M-sequences}

There is a large class of nearly perfect binary phase codes known as
M-sequences (``maximal length'') or PN-sequences (``pseudo noise'').
In contrast to the Barker Codes, the M-sequences exist for large
length (indeed for any length $M = 2^N - 1$, for positive integer
$N$).   The M-sequences share the nice autocorrelation property of the
Barker code in the cyclic as apposed to acyclic sense.  Thus when we
compute the range sidelobes via autocorrelation we have
\begin{equation}
\chi(\tau,0) = \sum_{k=0}^{k < N} c_k c_{k-\tau}
\end{equation}
and it is understood that either the subscripts of $c$ are computed
modulo $N$ or equivalently that the code is periodic with period $N$,
$c_k = c_{k+N}$.

The periodicity of the codes implies that they are 100 percent
duty-cycle --- this is a CW modulation scheme, implying that the radar
using these waveforms must be bistatic, although McEliece has
suggested that the waveforms actually work well aperiodically.

The construction of the codes is a direct, novel application of a
topic in abstract algebra, the Galois Fields arising from prime
polynomials modulo 2.  As a practical matter the codes are extremely
easy to generate in digital logic consisting of shift registers and
EXOR gates.  As an example, consider \figref{mseq1}, which shows a
shift register implementation associated with the polynomial $x^3 + x
+ 1 = 0$.

The shift register state stores the sequence $x^n$ which has been
reduced modulo the polynomial $x^3 + x + 1$, with coefficients
evaluated modulo two.  Thus we have
\begin{equation}
\begin{array}{lclclcl}
1 \\
x \\
x^2 \\
x^3 &\equiv& x+1 \\
x^4 &      & x^2 + x \\
x^5 &      & x^3 + x^2 &\equiv& x^2 + x + 1 \\
x^6 &      &           &      & x^3 + x^2 + x &\equiv&x^2 + 1 \\
x^7 &      &           &      &               &  &x^3 + x \equiv 1
\end{array}
\end{equation}
Thus we can see that $x^7 \equiv 1$ modulo $x^3 + x + 1$, which means
(of course) that $x^{7n} \equiv 1$.  The M-sequences are formed by
examining the coefficients of any bit; if we examine the 1s place, we
have $1001011$.  Mapping $1$ to $+$ and $0$ to $-$ we achieve a
biphase code with nearly ideal cyclic autocorrelation: the zero lag
has value $N$ and all other lags have value $-1$.

A little experimentation reveals that the sequence generated by
multiplying one M-sequence by a delayed copy is itself the same
M-sequence with a third shift.  Also, one can see that the M-sequences
have exactly one more $1$ than $0$.  Assembling these facts leads to
the strong correlation properties of the M sequences.

\begin{figure}
\centerline{\epsfxsize=0.6\textwidth \epsffile{mseq-sr.eps}}
BLAH.
\caption{\label{f:mseq1} This shift register circuit generates an M
sequence based on the polynmial $x^3 + x + 1 = 0$, which can be
rearranged (modulo $2$) to be $x^3 = 1 + x$.  The m-sequence arises
from the ``ones'' place of the shift register system, $\ldots 1001011\ldots$.}
\end{figure}

It is quite fortunate that prime polynomials modulo 2 exist for all
orders of polynomials.  Thus, unlike Barker codes arbitrarily long
M-sequences can be found (although not for all lengths).

The underlying polynomials are ``prime'' which means that they have no
factors except themselves and unity.  Of course, this factorization is
to be performed with coefficients modulo 2.  Thus, the polynomials $x$
and $x + 1$ are prime, although $x^2 + 1$ is not since it is the
square of $x+1$ (modulo 2).  A beginning table of prime polynomials
can be found in \tabref{ppolys}.  This table is not exhaustive, it's
just an indication of some primes.  None of the entries are really
large enough to use; in practice primes of order 10, giving sequences
of length 1023 begin to be useful.  A prime polynomial of order 48 is
embedded in the Global Positioning System algorithms; it is evaluated
each microsecond, and its period is thus (8.9 years).  The Qualcomm
CDMA protocol uses a generalization of m-sequences called ``Gold
codes'' \cite{uppala-1998,gold-1967,gold-1968} to permit several users
to share a single channel.

\begin{table}
\caption{\label{t:ppolys} A beginning table of prime polynomials
modulo 2.  The polynomials exist for all orders, and there are many of
them.  Polynomials of any particular order need only be tested by
primes of half order or smaller; also, a complete table can be built
using ``sieve'' methods.  As in the conventional natural primes, there
is no pattern which permits easy testing or generation of large primes.}
\centerline{\begin{tabular}{rl}
order & polynomials \\ \hline
1     & $x, x+1$ \\
2     & $x^2 + x + 1$ \\
3     & $x^3 + x + 1, x^3 + x^2 + 1$ \\
4     & $x^4 + x + 1, x^4 + x^3 + 1$ \\
5     & $x^5 + x^2 + 1, x^5 + x^3 + 1, x^5 + x^3 + x^2 + x + 1, \ldots$ \\
6     & $x^6 + x + 1, x^6 + x^3 + 1, \ldots$
\end{tabular}}
\end{table}



\part{Appendices}

\appendix

\chapter{Propagation and Doppler Shift}

\setlength{\arraycolsep}{0.5ex}

Now we'll review propagation of radio waves.  The simplest model is an
empty Galilean (non relativistic) universe containing a single target
and a transmitter located at the origin.  Subsequently we'll permit
the target to move, and even allow a varying media through which the
wave travels to the ``target.''

\section{Propagation in a simple universe}

We've previously considered the amplitude of radar signals with the
radar equation; now we'll consider other ways in which the signal is
modified by delaying it and by Doppler shifting it.  We can simplify
the discussion by considering a one dimensional universe.

The transmitter emits some waveform $s(ct)$ which then propagates away
from the transmitter, so that space-time contains the signal $s(ct -
x)$ (consider $x > 0$ for now).  Notice that $ct$ has length units,
which is conceptually useful and also connects radar to the theory of
special relativity.

The target located at $x_1$ is thus bathed in the signal $s(ct - x_1)$
and it scatters a fraction of that signal, $f s((ct - x_1) + (x
- x_1))$.  Here $f$ indicates the scattering amplitude, not the
scattering cross section $\sigma$.  Letting $f$ be a simple scalar
hides the realistic possibility that the scattering process may not be
uniform for all frequencies --- thus the scattering amplitude should
be more generally represented as an impulse response convolved with
the illuminating signal; and even this does not admit the 
possibility of a nonlinear scattering process.  Thus a little more
generally, but not particularly usefully, we can write that
backscattered signal $u(t,x) = \mathcal{S}[s(ct - x_1) + (x -
x_1),x,t]$, where $\mathcal S$ is a nonlinear operator dependent upon
space and time.

When the scattered signal returns to the transmitter, it can be
described as $f s((ct - x_1) + (0 - x1)) = f s(ct - 2x_1)$,
the factor $2$ appearing to indicate the round trip.  If we compute
the ``phase velocity'' $x/t = c/2$ we have a quantity that arises
frequently in radar, the ``speed of radar.''  This speed is certainly
$c/2 = 1.5\times 10^8$ m/s, but in units more convenient for radar
technology it is also 150 km/ms and 150 m/$\mu$s.  Thus the echo of a
target at a range of 150 km returns to the transmitter one millisecond
after its transmission, and if the receiver is sampled at 1 $\mu$s
intervals, each sample represents 150 m of range increment.

\newenvironment{factoid}{}{}

\begin{factoid} The ``Speed of Radar.''  The factor $c/2$ arises
  frequently in radar, and has the value $1.5\times 10^8$ m/s.
  However, it is very conveniently expressed additionally as
\begin{displaymath}
\frac{c}{2} = 150 \;\textrm{m}/\mu\textrm{s} = 150 \;\textrm{km/ms}
\end{displaymath}
\end{factoid}


\section{Galilean Doppler Shift}

Let us suppose that the target at $x_1$ is not fixed, but moving with
some velocity, so that $x_1 \rightarrow x_1 + vt$.  If we illuminate the target
with a sinusoid, $s(ct) = \cos(\omega t)$ then the returning
scattered wave will be
\begin{eqnarray*}
\sigma \cos(\omega t - 2 k (x_1 + vt)) &\sim& 
  \cos((\omega - 2kv) t - 2kx_1) \\
&=&\cos((\omega + \Delta \omega)t - 2kx_1) \\
\Delta \omega &=& -2kv \\[0.5ex]
\Delta f      &=& -2\frac{v}{\lambda}
\end{eqnarray*}
Here $\Delta f$ is the Doppler Shift (in Hz) associated with motion of
the target away from the transmitter, in particular, negative Doppler
Shift is associated with increasing distance ($v > 0$) while positive
Doppler Shift indicates decreasing distance.

Both positive and negative frequencies are possible Doppler shifts;
so we need to accomodate positive and negative frequencies in spectrum
estimation of radar signals.  In conventional signal processing
negative frequency content is implicitly paired with positive
frequency content. In radar signal processing we will find that the
``natural'' time series for radar signal processing is complex valued,
not real valued, and that this provides a simple means to preserve the
sign of the Doppler shift.

\section{Relativistic Doppler Shift}

In the previous derivation of the Doppler Shift, one might be troubled
that a new frequency is generated without changing the wavenumber;
thus the speed of propagation must change.  One could fix this problem
by self consistently changing the wavenumber through a more
sophisticated argument.  However a much more powerful approach is to
rely directly upon Special Relativity.

Consider three events: $\mathcal{A}$, a pulse of frequency $\omega$ is
emitted from the transmitter; $\mathcal{O}$, the pulse encounters and
scatters from a target; $\mathcal{B}$, the scattered signal returns to
the transmitter.  For convenience, let us place the origin of all
coordinate systems at $\mathcal{O}$, aligning the spatial axes of all
coordinate systems; let the relative motion of the coordinate systems
lie along the $x$ axis.  Thus, Lorentz Transformations of 4-vectors
will affect only $x$ and $t$ axes.

In its inertial frame, let the transmitter be located at $(x,y,z) =
(x_t, y_t, 0)$.  The motion of the target is described in the
transmitter frame as $x(t) = Vt, y = z = 0$.  The times of events
$\mathcal{A}$ and $\mathcal{B}$ are easy to calculate,
\begin{equation} \label{e:relativity1}
ct_{\mathcal{A},\mathcal{B}} = \mp \sqrt{x_t^2 + y_t^2} = \mp r_t
\end{equation}
These can be easily converted to times in the target frame of
reference through the Lorentz Tranformation:
\begin{equation} \label{e:relativity2}
ct_{\mathcal{A},\mathcal{B}}' = 
  \gamma (ct_{\mathcal{A},\mathcal{B}} - \beta x_t) =
  \gamma (\mp r_t - \beta x_t)
\end{equation}
Note that although the times of flight inbound and outbound are
identical in the transmitter rest frame, but not (in general) in the
target frame.

In the rest frame of the target, the transmitter is moving, and its
motion is described as $(x', y', z') = (x'_t - Vt', y'_t, 0)$.  The
frequency of the incoming and outgoing waves is the same, say
$\omega'$.  The total phase accumulated along the world lines
$\mathcal{AO}$ and $\mathcal{OB}$ is
\begin{eqnarray}
\phi'_{\mathcal{AO}} &=& 
 \frac{\omega'}{c}\left(0 - ct'_\mathcal{A}\right) = 
 \frac{\omega'}{c}\left(\gamma(r_t - \beta x_t)\right) \\
\phi'_{\mathcal{OB}} &=& 
 \frac{\omega'}{c}\left(ct'_\mathcal{A} - 0\right) = 
 \frac{\omega'}{c}\left(\gamma(r_t + \beta x_t)\right)
\end{eqnarray}

Since the phase is invariant to Lorentz Transformation, $\phi =
\phi'$.  Thus, the frequencies in the frame of the transmitter are
given by
\begin{equation}
\omega_{\mathcal{AO}, \mathcal{OB}} = 
\frac{c\phi_{\mathcal{AO}, \mathcal{OB}}}
            {ct_{\mathcal{A},\mathcal{B}}} =
\frac{\gamma(r_t \mp \beta x_t)}{r_t}\omega'
\end{equation}
Thus, the change in frequency is easily expressed as a ratio,
\begin{equation}
\frac{f_r}{f_t} = \frac{\omega_{\mathcal{OB}}}{\omega_{\mathcal{AO}}}
= \frac{r_t - \beta x_t}{r_t + \beta x_t}
\end{equation}
With $\cos(\theta) = x_t/r_t$ this may be written simply as
\begin{eqnarray}
\frac{f_r}{f_t} &=& \frac{1 - \beta \cos(\theta)}{1 + \beta \cos(\theta)}\\
\frac{\Delta f}{f_t} = \frac{f_r - f_t}{f_t} &=& \frac{-2\beta\cos(\theta)}{1 + \beta\cos{\theta}}
\end{eqnarray}
The angle $\theta$ is determined by the radio wave propagation
direction and the target velocity vector, such that $\theta = 0$
corresponds to the target receding along the line of sight.

Expanding the relativistic Doppler shift expression in $\beta$ we have
\begin{equation}
\frac{f_r}{f_t} = 
1 - 2\beta\cos(\theta) + 2(\beta\cos(\theta))^2 + \cdots
\end{equation}
To first order in $\beta$ this agrees with the non relativistic
development.

Note that the target will (in general) see a Doppler shift of the
transmitter signal even if there is no net Doppler shift at the
receiver ($\cos(\theta) = 0$).  In fact, the target's frequency shift
and radar frequency shift need not have the same sign.

However, in the ordinary non relativistic case, the target Doppler
shift will be half the radar Doppler shift, and relativistic
corrections will be quite small.

\begin{example} radar observation of meteor ``head echoes.''  Among the
fastest targets commonly seen with radars are meteors.  Near Earth
meteors may have velocity up to abuot 75 km/sec.  Relativistic
corrections to Doppler phenomena would have size
\begin{displaymath}
\beta = \frac{v}{c} = \frac{75\times 10^3}{3\times 10^8} = 25\times
10^{-5} = 0.00025
\end{displaymath}
Orbital velocity is 7.5 km/sec ($\beta = 2.5\times 10^{-5}$) and
aircraft travelling at Mach 2 travel 750 m/s ($\beta = 2.5\times
10^{-6}$).
\end{example}

\begin{factoid} The speed of light is just about one million times faster
than the speed of sound in air; the speed of light is Mach One
Million.
\end{factoid}

\section{Doppler Dispersion}

The Doppler Shift expressions show that velocity applies a constant
rescaling to the transmitter frequency.  If the transmitter waveform
has nonzero bandwidth (described by a Fourier spectrum $P(f)\, df$),
then the differential Doppler shift of higher frequencies is greater
than that of lower frequencies.  Thus target motion causes the
modification of the spectrum not as $P(f + \Delta f) \, df$ but rather
as $P(\alpha f) \,df\!/\!\alpha$.  However, in the usual non
relativistic case, with narrowband transmitter waveforms, $\alpha = 1
+ \epsilon$, $|\epsilon| \ll 1$, so that $P(\alpha f) \approx P(f +
\epsilon f_t) = P(f + \Delta f)$, where $f_t$ is the center frequency
of the transmitter.

When the bandwidth of the scattered signal exceeds about one percent
of the transmitter center frequency, this spectral asymmetry may need
to be accounted for in precision measurements.  The circumstance
arises in VHF and UHF radar observations of the ``plasma line'' in the
ionosphere; the plasma line induces a shift (modulation) on the order
of 10 MHz at the peak of the ionosphere (about 350 km at noon).

\section{Variable Media}

When the medium through which the wave propagates is not ideal vacuum,
additional details must be accomodated.  If the medium is constant but
dispersive, a short pulse injected into the medium will be spread in
time.  If the medium has a time-varying index of refraction, scatter will
be generated from that change.  Let us consider that case, briefly.

Suppose that a transmitter is located at $x_1$ and a target at $x_2$
separated by a region of space whose scalar dielectric constant is
also a function of space, such that
\begin{eqnarray*}
\epsilon &=& \epsilon_0 + \epsilon_1(x,t) \\
0 &\le& \epsilon_1(x,t)  \ll \epsilon_0
\end{eqnarray*}
Let us further suppose that the time scale for variation of the medium
is slow compared to the transit time, namely
\begin{displaymath}
\epsilon_1(x,t) \left(\frac{\partial \epsilon_1(x,t)}{\partial t}\right)^{-1}
\ll \frac{|x_t - x_r|}{c}
\end{displaymath}
Then the elapsed phase is $2\int k dx = 2\int \omega/v_{ph} dx$.
\begin{displaymath}
\phi(t) = 4\pi \int_{x_t(t)}^{x_r(t)} \frac{f}{v_{ph}(x,t)} \;\; dx
\end{displaymath}
The Doppler Shift is simply $\partial \phi/\partial t$, which in this
fairly general case has three components due to the motion of the
transmitter, the motion of the receiver, and the changing of the
medium through which the radiowave travels.  Thus (frequently dropping
the slow time dependence)
\begin{displaymath}
\Delta \omega =  4\pi f\left(
\frac{\dot{x}_t}{v_{ph}(x_t)} - \frac{\dot{x}_r}{v_{ph}(x_r)} +
\int_{x_t}^{x_r} \frac{\partial}{\partial t}
\frac{dx}{v_{ph}(x,t)}
\right)
\end{displaymath}
The first two terms represent the ordinary Doppler Shift associated
with relative motion between the transmitter and scatterer.  The third
term shows that a Doppler shift (change in frequency of scattered wave)
can occur even if both the transmitter and target are fixed --- but
the medium between them fluctuates.

\chapter{Mathematical Reference}

\section{Texts}

Abramovitz and Stegun

Gradshteyn and Ryzhik

Morse and Feschback

Arfken

\section{Fourier Transform}

\subsection{a rambling discussion}

There are many definitions of the Fourier Transform.  While these definitions are mutually consistent, 
their variety may be disheartening.  In the definitions for Fourier Transforms of continuous 
functions there are features to be aware of.  

\begin{itemize}
\item does the reconstruction model depend upon $e^{j\omega p}$ or $e^{-j\omega p}$.
\item does the expression of frequency depend upon ``natural'' frequency (radians per unit) or 
cyclic frequency (cycles per unit).
\end{itemize}

In electrical engineering practice it is usually the case that sinusoidal excitation is expressed with the
dependence $e^{j\omega t} = e^{2\pi j f t}$, with $j$ itself understood to represent the imaginary unit in 
the positive imaginary half plane.

Ultimately this means that plane waves, to the Electrical Engineer, would be described as follows:
\begin{displaymath}
E(\vec{r},t) = E_0 e^{j(\omega t - \vec{k}\cdot\vec{r})}
\end{displaymath}
When studying such waves with Fourier Transform techniques, the underlying constructive basis 
has a ``$+$'' in front of the temporal component, and a ``$-$'' sign in front of the spatial component.  
For example, consider the following wave:
\begin{displaymath}
E(x,t) = e^{2\pi j(4t - 6x)}
\end{displaymath}
If the temporal variation was suppressed, one would be left with 
\begin{displaymath}
E(x,t) = e^{2\pi j(- 6x)}
\end{displaymath}
in which case it is ambiguous whether the wave is travelling in the $+x$ or $-x$ direction unless you
were certain of the Electrical Engineering convention.

On the other hand, one might prefer to describe waves with a ``$+$'' sign in front of the spatial 
component in order to make it slightly more obvious which direction the wave is propagating.  About
half of the Physics literature uses this convention, which is the opposite of the Electrical Engineering
convention.

Finally, it is worth mentioning that Ishimaru's electromagnetics textbook CITE CITE CITE uses 
the Electrical Engineering convention for the first 2/3, and the opposite convention for the final 1/3.
This was a reasonable choice by Ishimaru, but it serves as a warning that one must keep one's 
wits handy.

\vspace{1em}
The second issue has to do with the use of ``natural'' ($\omega$) vs ``cyclic'' frequency ($f$).  
Fortunately this issue is a little less foggy; it primarily has to do with the presence or absence of
factors of $2\pi$ in the transformation.  Here, the preference we express is pragmatic: the use of 
``cyclic frequency'' leads to simpler expressions, easier to remember, and easier to apply.

\label{fourier-transform}

\begin{eqnarray}
g(t)         &\leftrightarrow& G(f) \\
g(t)         &=& \infint G(f) e^{2\pi j f t} \; df \\
G(f)         &=& \infint g(t) e^{-2\pi j f t} \; dt 
\end{eqnarray}

Note that these transforms have no constants in front of the integrals.  It makes sense if you look 
at the symmetry of $f$ and $t$: if there was a constant, it should be the \textit{same} constant.

If you put the latter two equations together, you can deduce the following very useful theorem:
\begin{equation}
\delta(t) = \int_{-\infty}^\infty e^{2\pi j f t} \;df
\end{equation}
This is a pretty strong statement that ``cyclic frequency'' is a very good choice.

Some additional theorems:
\begin{eqnarray}
g(t - t_0)   &\leftrightarrow& e^{-2\pi j ft_0} G(f)  \\
e^{-2\pi f_0 t} g(t) &\leftrightarrow& G(f + f_0) \\
g(t) \;\textrm{is real} &\leftrightarrow& G(f) \;\textrm{has Hermite-symmetry} \\
S(f) \;\textrm{is real} &\leftrightarrow& R(\tau) \;\textrm{has Hermite-symmetry}
\end{eqnarray}

\section{Physical Processes}

The power spectrum of every physical process must be non-negative definite because
\begin{itemize}
\item It doesn't make sense to have a negative energy density for any frequency
\item ... and it certainly doesn't make sense to have a complex-valued energy density.
\end{itemize}

Skipping a lot of (important, but slightly tedious) math, the power spectrum $S_{xx}(f)$ can be defined as follows:
\begin{equation}
S_{xx}(f) = E (|X(f)|^2) = E (X(f)X^\ast(f))
\end{equation}
Here $X(f)$ is the Fourier Transform of a physical process $x(t)$.  Since $x(t)$ is real, it is necessarily the case
that $X(f)$ has Hermite symmetry, $X(f) = X^\ast(-f)$.

This means in turn that the Power spectrum must not only be real but even:
\begin{equation}
S_{xx}(f) = E (|X(f)|^2) = E (X(f)X^\ast(f)) = E (X(f)X(-f)) = S_{xx}(-f)
\end{equation}

And this in turn means that the Autocorrelation function $R(\tau)$ must have Hermite symmetry.

\chapter{Special Functions}

\section{Bessel Functions}

An engineering scientist's introduction to the Bessel Functions usually arises from the study of the wave 
equation in circular geometry, in which the two dimensional cartesian Laplacian is expressed in 
cylindrical coordinates:
\begin{equation}
\frac{\partial^2 f}{\partial r^2} + \frac{1}{r}\frac{\partial f}{\partial r} + \frac{1}{r^2}\frac{\partial^2 f}{\partial \phi^2} + \omega^2 \mu \epsilon \,f = 0
\end{equation}
In radar, the Bessel functions show up in quite different contexts, and most commonly as the 
\textit{modified} Bessel functions, associated with probabilities.
\section{The Rician density}
\begin{equation} f(x\mid \nu ,\sigma )={\frac  {x}{\sigma ^{2}}}\exp \left({\frac  {-(x^{2}+\nu ^{2})}{2\sigma ^{2}}}\right)I_{0}\left({\frac  {x\nu }{\sigma ^{2}}}\right)
\end{equation}
This is the pdf associated with observing a voltage of magnitude $x$ in the presence of gaussian noise 
with variance $\sigma^2$ and a sine wave of amplitude $\nu$.  This collapses to the usual Rayleigh 
distribution when the amplitude of the sine wave goes to zero,
\begin{equation} f(x\mid \nu\rightarrow 0,\sigma )={\frac  {x}{\sigma ^{2}}}\exp \left({-\frac {x^{2}}{2\sigma ^{2}}}\right)
\end{equation}

\section{The K distribution}

NEEDS MORE WORK.

The K-distribution arises in contemplation of the pdf of the product of two (complex) gaussian random 
variables.  The joint density of the magnitude of two complex gaussian RVs is given as follows:
\begin{eqnarray}
f(r, s \mid \sigma ) &=& \exp\left(-\frac{r^2}{2\sigma^2}\right)\exp\left(-\frac{s^2}{2\sigma^2}\right)\,rs\,drds \\
f(R,S \mid \sigma )&=&
\end{eqnarray}

If we let $R = rs$, then we can figure out the cumulative distribution of R as follows:
\begin{eqnarray}
F( R) &=& \nt_0^R f(R) dR \\
         &=& \int_0^\infty \int 0^s 
\end{eqnarray}





\chapter{Computation of the Ambiguity Function}
\label{s-ambiguity-function}

As pointed out in Chapter 3, the Ambiguity Function describes the resolution in range and Doppler 
provided by a radar waveform.  Unfortunately this function has a number of different but roughly 
equivalent forms.

The form used in this report is as follows, for a radar waveform $u(t)$:
\begin{equation}
\chi(\tau,\nu)  = \int u(t) u^\ast(t - \tau) e^{2 \pi j \nu t} \; dt
\end{equation}
Some authors prefer to take the absolute value of this expression; others take the absolute
value squared.  We prefer the complex-valued form above for  important but subtle reasons:
there is important information present in the phase, and the units of $\chi$ are effectively
field units (e.g. volts).

The importance of the phase was demonstrated by Meyer [2003]\footnote{@article{meyer2004passive,
  title={Passive coherent scatter radar interferometer implementation, observations, and analysis},
  author={Meyer, Melissa G and Sahr, John D},
  journal={Radio science},
  volume={39},
  number={3},
  year={2004},
  publisher={Wiley Online Library}
}}, who pointed out how to perform
multi-antenna interferometry by computing cross-ambiguity $\chi_{nm}$ on two or more antennas to 
image the angular distribution of scatter of a deep fluctuating targets.

\section{Self-Ambiguity}

In chapter 3 the Ambiguity function and its several properties were developed.  One of them was a 
symmetry property:
\begin{equation}
|\chi_{uu}(\tau, \nu)| = |\chi^\ast_{uu}(-\tau, -\nu)|
\end{equation}
With a slightly different (but completely equivalent) definition of the Ambiguity function, the
previous expression is true \textit{without} the absolute value operation:
\begin{equation}
\tilde{\chi}_{uu}(\tau, \nu) = \int u\!\left(t + \frac{\tau}{2}\right) u^\ast \!\left(t - \frac{\tau}{2}\right) e^{2 \pi j \nu t} \; dt \label{e:wigner}
\end{equation}
With this definition, the symmetry property is more simply (and powerfiully) stated:
\begin{equation}
\tilde{\chi}_{uu}(\tau, \nu) = \tilde{\chi}_{uu}(-\tau, -\nu) 
\end{equataion}
This form of the ambiguity function has some very nice properties, but it is not quite so convenient
for computation.  Fortunately it is very easy to compute from $\chi_{uu}(\tau,\nu)$:
\begin{equation}
\tilde{\chi}_{uu}(\tau, \nu) = e^{-\pi j \nu \tau} \chi_{uu}(\tau, \nu)
\end{equation}

\subsection{Wigner Distribution}
The definition of $\tilde{\chi}_{uu}(\tau, \nu)$ in \eqref{wigner} is also known as the Wigner 
Distribution\footnote{see \textt{https://en.wikipedia.org/wiki/Wigner\_distribution\_function}}, which
arises in some problems in signal analysis and quantum mechanics.  As such it has a number of 
properties beyond those mentioned in Chapter 3.
\begin{eqnarray}
|x(t)|^2 &=& \int \tilde{\chi}_{xx}(\tau, \nu) \; d\nu \\
|X(\nu)|^2 &=& \int \tilde{\chi}_{xx}(\tau, \nu) \; d\tau \\
\int \tilde{\chi}_{xx}(\tau/2, \nu) e^{2\pi j \nu \tau} \; d\nu &=& x(t) x^\ast(0)  \\
\int \tilde{\chi}_{xx}(\tau, \nu/2) e^{2\pi j \nu \tau} \; d\tau &=& X(\nu) X^\ast(0) \\
x(t) \rightarrow \tilde{\chi}_{xx}(\tau, \nu) &\leftrightarrow& 
   x(t-\eta) \rightarrow e^{2\pi j \nu \eta} \tilde{\chi}_{xx}(\tau, \nu) \\
x(t) \rightarrow \tilde{\chi}_{xx}(\tau, \nu) &\leftrightarrow& 
   x(t) e^{2 \pi j f t} \rightarrow e^{2\pi j f \tau} \tilde{\chi}_{xx}(\tau, \nu) 
 \end{eqnarray}

\section{Cross-ambiguity}

The cross ambiguity $\chi_{nm}$ is defined as follows, describing a signal that was received on
antenna $n$, with a reference to the transmitter signal provided on antenna $m$.:
\begin{equation}
\chi_{nm}(\tau,\nu)  = \int u_n(t) u_m^\ast(t - \tau) e^{2 \pi j \nu t} \; dt
\end{equation}

\section{Cross-ambiguity in the presence of zero doppler ground clutter}

Suppose that the field of view contains a variety of scattering sources of strength $\alpha_k$ at 
group delay $\tau_k$.

Then the detected signal (following application of the matched filter) is the cross-ambiguity:

\begin{eqnarray}
\chi'_{nm}(\tau,\nu) &=& \sum_k \alpha_k \int u_n(t - \tau_n) u_m^\ast(t - \tau) e^{2 \pi j \nu t} \; dt \\
&=& \sum_k \alpha_k \chi_{nm}(\tau - \tau_k, \nu)
\end{eqnarray}
Equation D.4 is essentially the starting point for the Manastash Ridge Radar (Sahr, Lind 1997), and 
applies when the target extends over group delays that are much larger than the correlation time
of the illumination $u_m(t)$.

In the case of close-in ground clutter, inside the correlation time of $u_m(t)$ this formula is not directly 
useful.

By using Equation D.10 above, we can rewrite D.14 as an interesting theorem:

\begin{equation}
\chi'_{nm}(\tau,\nu) = \chi_{nm} (\tau,\nu) \sum_k \alpha_k e^{-2\pi j \nu \tau_k}
\end{equation}


Alternatively we could create the ratio
\begin{equation}
\frac{\chi'_{nn}(\tau,\nu)}{\chi_{nn} (\tau,\nu)} = \sum_k \alpha_k e^{-2\pi j \nu \tau_k}
\end{equation}
The numerator on the left can be directly estimated from data; the denominator can be estimated if the
the uncorrupted signal $u_n(t)$ can be estimated.  Finally, the right hand side is a computable
linear combination of the $\alpha_k$ if the $\tau_k$ are known.  However, it can also be viewed as
a Fourier Transform of the $\alpha_k$ for a set $\tau_k = k\tau_0$.



\chapter{The Cram\'er-Rao Lower Bound}
\label{s:cramer-rao}

The Cram\'er-Rao Lower Bound (CRLB) provides an absolute lower bound
on the variance of estimators of functions of parameters in the pdf of
random variables, given a knowledge of the functional form of the pdf.

\section{single parameter case}

Let us first consider the case of an estimate of a single parameter of a
random variable\footnote{The proof given is based upon a proof
found in the Wikipedia on 14 July 2007
\texttt{http://en.wikipedia.org/wiki/Cram\'er-Rao\_bound\#Single-parameter\_proof}}.
Consider a set of data $x$ which has a pdf $f(x; \theta)$ that depends
upon a parameter $\theta$.  For example, $\theta$ could represent the
mean or standard deviation of the random variable.  Let us form an
estimator of the data $t(x)$ which produces the new random variable
$T$; we'll show all the intermediate steps, for clarity, as some of
the manipulations can appear magical upon first encounter:
\begin{eqnarray}
T &=& t(x) \\
\langle T \rangle &=& \int t(x) f(x; \theta) \; dx = \psi(\theta)
\end{eqnarray}
Since the pdf is assumed to be normalized, we also have the following
curious result:
\begin{eqnarray}
\int f(x; \theta) \; dx &=& 1 \\
\frac{\partial}{\partial \theta} \int f(x; \theta) \; dx &=&
\frac{\partial}{\partial \theta} \; (1) = 0 \\
\int \frac{\partial}{\partial \theta} f(x; \theta) \; dx &=& 0 \\
\int \frac{\partial \log f(x; \theta)}{\partial \theta} \; f(x; \theta)
\; dx &=& 0 \\
\left\langle \frac{\partial \log f(x; \theta)}{\partial \theta} \right\rangle&=& 0 \\
\textrm{define} \;\;\; V &=& \frac{\partial \log f(x ;\theta)}{\partial \theta} \\
\langle V \rangle &=& 0
\end{eqnarray}
Now, consider the covariance of $T$ and $V$:
\begin{eqnarray}
\textrm{cov}(V,T) &=& \langle (T - \langle T \rangle)(V - \langle V
\rangle)\rangle \\
&=& \langle (T - \langle T \rangle) V\rangle \\
&=& \langle T V \rangle - \langle \langle T \rangle V\rangle \\
&=& \langle T V \rangle - \langle T \rangle \langle V\rangle \\
&=& \langle T V \rangle \\
&=& \int t(x) \frac{\partial \log f(x; \theta)}{\partial \theta} f(x;
\theta)\; dx \\
&=& \int t(x) \frac{1}{f(x; \theta)} \frac{\partial f(x;
  \theta)}{\partial \theta} f(x; \theta) \; dx \\
&=& \int t(x) \frac{\partial f(x; \theta)}{\partial \theta} \; dx \\
&=& \frac{\partial}{\partial \theta} \int t(x) f(x; \theta) \; d\theta \\
\textrm{cov}(V,T) &=& \frac{\partial \psi(\theta)}{\partial \theta}
\end{eqnarray}
Now, we examine the covariance by contemplating the Cauchy-Schwartz
inequality \eqref{cauchy-schwarz-integral}; let $p(x) = T = t(x)$ and
$q(x) = V$.  Then, the C-S inequality can be expressed as
\begin{eqnarray}
\textrm{var}(T) \textrm{var}(V) &\ge& \left(\textrm{cov}(V,T)\right)^2
= \left(\frac{\partial \psi(\theta)}{\partial \theta} \right)^2 \\
\rightarrow \;\; \textrm{var}(T) &\ge& 
\frac{\left(\frac{\partial \psi(\theta)}{\partial \theta}\right)^2}
     {\left\langle \left(\frac{\partial \log f(x; \theta)}{\partial \theta}\right)^2\right\rangle}
\end{eqnarray}
Note that the quantity in the denominator of the r.h.s.~does not
depend upon the data $x$ because the expectation over $x$ has been taken.

\chapter{Helpful Identies and Theorems}

\section{Cauchy-Schwarz Inequality}

\begin{equation} \label{e:cauchy-schwarz-integral}
\left| \int p(x) q^\ast(x) f(x) \; dx \right|^2 \le 
\int |p(x)|^2 f(x) \; dx \int |q(x)|^2 f(x) \; dx 
\end{equation}
Here, we require that $p, q$ are square integrable over the measure
$f(x) \ge 0$.  The relation occurs in a discrete form as well:
\begin{equation}
\left| \sum_k p_k q_k^\ast f_k \right|^2 \le
  \sum_k |p_k|^2 f_k \sum_k |q_k|^2 f_k
\end{equation}
In many opportunities for application, $f(x) = 1$ and $f_k = 1$.  In
both the integral and discrete forms, equality arises if and only if
$p$ and $q$ are identical within a constant scalar multiple.

\section{Method of Lagrange Multipliers}

Suppose that have a score function of several variables, including some constraints, and we wish to seek an optimum value:
\begin{equation}
\textrm{minimize } f(x, y, \ldots) \textrm{ such that } g_n(x, y, \ldots) = 0
\end{equation}
The usual approach would be to use the constraints $g_n()$ to express all the variables $x, y, \ldots$ in terms of one of the 
variables (e.g. $x$), and then minimize the function $f(x, y(x), z(x), \ldots)$ with respect to $x$.

We can then create the augmented function $f'(x, y, \dots; \lambda_1, \lambda_2, \ldots)$ as follows:
\begin{equation}
f'(x, y, \ldots; \lambda_1, \lambda_2, \ldots) = f(x, y, \ldots) + \lambda_1 g_1(x, y, \ldots) + \lambda_2 g_2(x, y, \ldots) + \ldots
\end{equation}
Now we form the all the partial derivatives and set them equal to zero:
\begin{eqnarray}
\frac{\partial f'}{\partial x} &=& \frac{\partial f'}{\partial x} + \lambda_1 \frac{\partial g_1}{\partial x}+ \lambda_2 \frac{\partial g_2}{\partial x} + \ldots = 0 \\
\frac{\partial f'}{\partial y} &=& \frac{\partial f'}{\partial y} + \lambda_1 \frac{\partial g_1}{\partial y} + \lambda_2 \frac{\partial g_2}{\partial y} + \ldots = 0\\
&\vdots& \\
\frac{\partial f'}{\partial \lambda_1} &=& g_1(x, y, \ldots) = 0\\
\frac{\partial f'}{\partial \lambda_2} &=& g_2(x, y, \ldots) = 0\\
&\vdots&
\end{eqnarray}
You can see that the $\lambda$ derivatives simply return the constraints.   As presented this is not obviously an improvement, so let's illustrate with an example.

\subsection{example of the method of  Lagrange Multipliers}

Suppose that we have the following:

\begin{eqnarray}
\textrm{minimize } f(x, y) &=& 3 x + 5 y \\
\textrm{such that } xy^2 &=& 4 \\
\textrm{which can be written } xy^2 - 4 &=& 0
\end{eqnarray}
Then we form $f'$ and take partial derivatives.
\begin{eqnarray}
f'(x,y;\lambda) &=& 3 x + 5 y + \lambda(xy^2 - 4) \\
\frac{\partial f'}{\partial x} &=& 3 + \lambda y^2 = 0 \label{e:lgm01}\\
\frac{\partial f'}{\partial y} &=& 5 + 2\lambda xy = 0 \label{e:lgm02}\\
\frac{\partial f'}{\partial \lambda} &=& xy^2 - 4 = 0 \label{e:lgm03}
\end{eqnarray}
If we solve (\ref{e:lgm01}) and (\ref{e:lgm02}) for $\lambda$ can recover the following relationship for $x$ and $y$.
\begin{equation}
y = \frac{6}{5}x \label{e:lgm04}
\end{equation}
In concert with the constraint (\ref{e:lgm02}) we have
\begin{eqnarray}
x\left(\frac{6}{5}x\right)^2 &=& 4 \\
\frac{36}{25}x^3 &=& 4 \\
x &=& (25/9)^{1/3} = 1.40572 \\
y &=& \frac{2}{\sqrt{x}} = 2 \times (3/5)^{1/3} = 1.68686
\end{eqnarray}
Note that (\ref{e:lgm04}) provides a linear relationship between $x$ and $y$ which makes it easy to solve the constraint.

\include{chapter-digitizers}
\include{chapter-analytic}

\chapter{Hysell--Chau}
HC6 refers to the paper by Hysell and Chau, 2006\cite{hysell2006optimal}.
visibility from a baseline pair
\begin{eqnarray}
V(k(\vec{d}_n - \vec{d}_m)) &=& \int_{4\pi} A_n(\Omega) A^\ast_m(\Omega) B(\Omega) \exp(j k (\vec{d}_n - \vec{d}_m)\cdot \vec{\Omega}) \; d\Omega \\		
V(k\vec{d}_{nm}) &=& \int_{4\pi} B'_{nm}(\Omega) \exp(j k \vec{d}_{nm} \cdot \vec{\Omega}) \; d\Omega \label{e:vis0} \\
A_n(\Omega)      &=& n\textrm{th antenna response in direction } \Omega \\
B(\Omega)           &=& \textrm{image brightness in direction } \Omega \\
B'_{nm}(\Omega) &=& A_n(\Omega)A^\ast_m(\Omega) B(\Omega) \\
\vec{d}_n             &=& n\textrm{th antenna position} \\
\vec{d}_{nm}        &=&  \vec{d}_n - \vec{d}_m \;\;\; \textrm{antenna pair separation; baseline}
\end{eqnarray}
Note that the brightness $B$ properly has units of Watts/sr, so that the integral \eqref{vis0} can be interpreted as a dot 
product between $\exp(j k \vec{d}_{nm} \cdot \vec{\Omega})$ and $B'_{nm}(\Omega) \; d\Omega$ when both are
suitably sampled over the domain of integration.  In this case we may write
\begin{equation}
V(k\vec{d}_{nm}) = (B'(\Omega^i)d\Omega)\exp(j k \vec{d}_{nm} \cdot \vec{\Omega^i})  
\end{equation}
or more simply
\begin{eqnarray}
V(k\vec{d}_{nm}) &=& B^i h^i_{nm} \\
B^i       &=& B'(\Omega^i) d\Omega \\
h^i_{nm} &=&\exp(j k \vec{d}_{nm} \cdot \vec{\Omega}^i) \label{e:hnm}
\end{eqnarray}
In \eqref{hnm}, $\Omega$ can be thought of as both as a solid angle and as a unit vector.

Finally, we observe that carrying both $n$ and $m$ is a bit superfluous, and we'll enumerate the baselines $nm$ 
simply as $j$\footnote{However, we'll need the $nm$ form to deal with closure phase and amplitude later.  It is 
unfortunate that the integer subscript $j$ is overloaded with the imaginary unit.  However, since there are no 
complex quantities in MEM imaging the problem should be minor, and the context clear.}.
So, finally we write
\begin{eqnarray}
V(k\vec{d}_j) &=& B^i h^i_j \\
B^i       &=& B'(\Omega^i) \; d\Omega \\
h^i_j    &=&\exp(j k \vec{d}_j \cdot \vec{\Omega}^i)
\end{eqnarray}
HC6 refer to $h^i_j$ as the point spread function; it's important to note that this is the point spread function for
isotropic antennas only (HC6 later include the antenna gain patterns $A(\Omega)$.)

Measurements:
\begin{eqnarray}
g_j + e_j &=& f^i h^i_j \\
g_j         &=& \textrm{(imperfectly) measured visibility; the data} \\
e_j         &=& \textrm{error in true visibility measurement} \\
f^i           &=& \textrm{modeled image brightness per pointing direction, } \\
h^i_j      &=& \textrm{interferometric point spread function} \\
f^i h^i_j &=& \int_{4\pi} 
\underbrace{ \exp(j k \vec{d_j}\cdot \vec{\Omega}) }_{h^i_j}\; \underbrace{B'(\Omega) \; d\Omega}_{f^i} 
\end{eqnarray}
The last is a restatement of \eqref{vis0}.

Also note that notationally it is not possible for $h^i_j$ to remain complex valued; the real and imaginary 
parts have to be handled explicitly and separately in the Maximum Entropy formulation.  So, if there are 
$N$ antennas, then there are $N(N-1)/2$ baselines, and $N(N-1)$ real and imaginary components.  Thus,
algorithmically, $j$ runs from $0$ to $N(N-1) -1$.
\begin{eqnarray}
h_j^i &=& \cos(k\vec{d}_j \cdot \vec{\Omega}^i ) \;\;\;\;\; \textrm{for } j \textrm{ even} \\ 
h_j^i &=& \sin(k\vec{d}_j \cdot \vec{\Omega}^i) \;\;\;\;\; \textrm{for } j \textrm{ odd}  
\end{eqnarray}
It may be convenient to duplicate each baseline, algorithmically, so that $\vec{d}_{2n} = \vec{d}_{2n+1}$.

The total brightness is assumed to be a measured quantity; it occurs as a self correlation measurement rather
than a correlation between two antennas:
\begin{eqnarray}
F   &=& I^i f^i \\
     &\sim&  \int_{4\pi} B'(\Omega) \; d\Omega
\end{eqnarray}

Below we shall see an additional parameter, $Z$, which provides a normalization and Gibbs function for the MEM 
image.  A close reading of HC6 reveals what appears to be a tautology regarding $F$ and $Z$.  This tautology 
can be avoided by extending the interferometric baselines to include the $0$ baseline, in which the total power of 
the image scene arises naturally within the MEM model.  As an additional benefit, the zero 
baseline demonstrates its use a a "sharpness" or "focussing" parameter.   For small values of $\lambda_j$ ($
\lambda_j \le 1$) the MEM model is nearly linear in the Fourier sense, however for larger values of $\lambda_j$ 
the model can produce very sharp features.  This is one of the strengths of the MEM approach; it can rise to the 
occasion of super resolution given appropriate data.

Image entropy:
\begin{eqnarray}
S &=& -f^i \log(f^i/F) \\
&\sim& \int_{4\pi} B'(\Omega) \log( B'(\Omega)/F) \; d\Omega
\end{eqnarray}

The MaxEnt image model is
\begin{eqnarray}
f^i &=&\frac{F}{Z}\exp(-\lambda_j h_j^i) \\
Z &=& I^i \exp(-\lambda_j h_j^i)
\end{eqnarray}
for some $\lambda_j$ to be determined.  Note that
\begin{equation}
\log(f^i) = -\lambda_j h_j^i
\end{equation}
and we can write the entropy as
\begin{eqnarray}
S &=& -f^i\log((F/Z)\exp(-\lambda_j h_j^i)/F) \\
 &=& f^i \lambda_j h_j^i + f^i I^i\log(Z) \\
 &=& f^i \lambda_j h_j^i + F \log Z
\end{eqnarray}


Functional to be maximized including noise, 

\begin{eqnarray}
E(f(e_j, \lambda_j, \Lambda)) &=& S + \lambda_j(g_j + e_j - f^i h^i_j) + \Lambda(e_j^2 \sigma_j^{-2} - \Sigma) \\
&=& -f^i \log(f^i/F)+ \lambda_j(g_j + e_j - f^i h^i_j) + \Lambda(e_j^2 \sigma_j^{-2} - \Sigma) \nonumber \\
&=& f^i \lambda_j h_j^i + F \log(Z) + \lambda_j(g_j + e_j - f^i h^i_j) + \Lambda(e_j^2 \sigma_j^{-2} - \Sigma) \nonumber \\
&=& \lambda_n(g_j + e_j) + F \log Z + \Lambda(e_j^2 \sigma_j^{-2} - \Sigma) 
\end{eqnarray} 

Differentiating with respect to $\lambda_j, e_j, \Lambda$ we receive these equations to solve:
\begin{eqnarray}
g_j + e_j - f^i h^i_j &=& 0 \\[1em]
\lambda_j + \frac{2\Lambda}{\sigma_j^2} e_j &=& 0 \\[0.5em]
\Lambda^2 - \frac{\lambda_n^2 \sigma_n^2}{4\Sigma} &=& 0
\end{eqnarray}

In the case of no noise, we have
\begin{eqnarray}
g_j - f^i h^i_j &=& 0 
\end{eqnarray}

This could be useful for generating initial conditions.

\section{another whack}

The formulation of HC6 accommodates correlated errors among the $g_j$ through a similarity transform, and 
that's a good thing.

One of the surprising things about he solution of the coupled equations E.35-37 is that, given a set of $g_j$, and 
a solution $\lambda_j$ to them, the numerical solution $\lambda_j$ is extremely stable to variations of $g_j$.  
While initially satisfying, this leads to the puzzling circumstance that very few solutions are actually possible; or, to 
put it differently, in the noiseless case (perfect data $g_j$) the synthesized image is not a smooth function of the 
data.  A little investigation of this reveals that the MaxEnt image model is very stiff.  

\begin{eqnarray}
\Phi &=& \int \exp(-\lambda_k f_k(\Omega)) \;d\Omega \\
\Phi_n &=& \partial \Phi/\partial \lambda_n = \int -f_n(\Omega) \exp(-\lambda_k f_k(\Omega)) \;d\Omega = g_n \\
\Phi_{nm} &=& \partial^2 \Phi/\partial \lambda_n \partial \lambda_m = \int f_n(\Omega) f_m(\Omega) 
\exp(-\lambda_k f_k(\Omega)) \;d\Omega \\
\Phi_{nm} &=& \textrm{symmetric, positive definite} \nonumber \\
\Phi_{nm} \Delta \lambda_m &=& \Delta g_n \\
\Delta \lambda_m &=& [\Phi_{nm}]^{-1} \Delta g_n
\end{eqnarray}

The issue is that $\Phi_{nm}$ (and its inverse) is typically quite stiff --- the condition number is large, the range of
singular values/eigenvalues is large.  The error model is thus not so much estimating actual errors as it is 
compensating for the stiffness of $\Phi_{nm}$.  It would be interesting to find an alternative model which was less 
stiff, which would nevertheless preserve the friendly algebraic properties of the MaxEnt image model.

One possibility is the following:
\begin{eqnarray}
\Phi &=& \int \exp(-\lambda_k f_k(\Omega)) \;d\Omega + \int \exp(-\alpha \lambda_k f_k(\Omega)) \;d\Omega \\
&=&  \int \exp(-\lambda_k f_k(\Omega))\ + \exp(-\alpha \lambda_k f_k(\Omega)) \;d\Omega \\
\Phi_n &=& -\int f_n(\Omega) \exp(-\lambda_k f_k(\Omega)) \;d\Omega + \cdots \nonumber \\
 & & - \; \alpha \int f_n(\Omega) \exp(-\alpha \lambda_k f_k(\Omega)) \;d\Omega \\
\Phi_n &=& -\int f_n(\Omega) [ \exp(-\lambda_k f_k(\Omega)) + \alpha  \exp(-\alpha\lambda_k f_k(\Omega)) ] \;d
\Omega \\
\Phi_{nm} &=& \int f_n(\Omega) f_m(\Omega) \exp(-\lambda_k f_k(\Omega)) \;d\Omega + \cdots \nonumber \\
 & & + \; \alpha^2 \int f_n(\Omega) f_m(\Omega) \exp(-\alpha \lambda_k f_k(\Omega)) \;d\Omega \\
\Phi_{nm} &=& \!\int \! f_n(\Omega) f_m(\Omega) [ \exp(-\lambda_k f_k(\Omega)) + \alpha^2  \exp(-\alpha
\lambda_k 
f_k(\Omega)) ] \;d \Omega 
\end{eqnarray}
This is clearly extensible.  One possibility would be
\begin{equation}
\Phi = \int e^{-\lambda_k f_k(\Omega)} +  e^{-\alpha \lambda_k f_k(\Omega)} + 
e^{-\beta \lambda_k f_k(\Omega)} + e^{-\gamma \lambda_k f_k(\Omega)} \; d\Omega
\end{equation} 
\printindex

\cleardoublepage

\phantomsection

\addcontentsline{toc}{chapter}{Bibliography}
\bibliography{radar}
\bibliographystyle{plain}


\end{document}
