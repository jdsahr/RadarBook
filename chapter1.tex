\chapter{Introduction}

Radar is  a mature technology whose basic principles are very well 
understood. However, enormous potential remains, so research in 
radar technique remains very lively.  The innovators of radar\footnote{In this text we'll 
elevate "radar" to the status of a common noun from its origin as an acronym, 
RADAR, "RAdio Detection And Ranging."} technology have always 
been able to imagine instruments which are impossible to build in that era
but which have become possible in a subsequently.

By the beginning of the 21st Century several technological advances have changed a number
of the driving forces in radar design.  In particular, high speed, low
cost computing has permitted access to signal processing algorithms that
were effectively impossible only a decade earlier.  The availability
of the internet, and GPS time/frequency reference has enabled startling
new opportunities for multistatic radars and indeed an entirely new
topology of radar network.  Although advances in RF electronics
have increased performance in various ways (lower noise figure, higher
dynamic range, ever higher frequencies, semiconductor power amplifiers
replacing vacuum devices), the performance revolution has not been so
dramatic as in computation.  Furthermore, the cost of electricity has
not changed substantially --- so high power transmitters remain
expensive to acquire and to operate.  However, one of the parameters
above --- dynamic range --- does affect our ability to make full use
of modern digital methods. 
The tension between the possible in principle and the possible in
practice has created an enormous repertoire of tricks to permit the
performance of ``impossible'' measurements.  As technology improves,
these tricks' importance wanes, although the motivation for the tricks
remain --- and an understanding of the tricks can conceivably enable a
new generation of radars which could not have been approached at all.
We will describe how clever design of waveforms and receivers can 
\begin{itemize}
\item eliminate significant imperfections of the receivers
\item permit ``gigaflop'' processing with ``megaflop'' computers
\item enable fine resolution of targets whose range and velocity
spectrum cannot be studied by ``conventional'' means.
\end{itemize}

The myriad uses of radar instruments provide a constant demand for
more powerful transmitters, more sensitive receivers, greater receiver
dynamic range, more real-time signal processing capability, more
stable time and frequency references, and better (lower) component
cost.  As radar is applied to complex consumer tasks, such as air
traffic control, even human psychology becomes an issue as we struggle
to present complex fields of data to humans in a way which is natural
and comprehensible, permitting the optimal deployment of extremely
valuable resources.

The basic idea behind radar is simple: we measure the time of flight
of a radio wave pulse to deduce the distance to an object which
reflected the radio wave.  This is the origin of the acronym ``RADAR''
which stands for ``RAdio Detection And Ranging.'' The ubiquity of
RADAR has brought it wholly into the language as conventional word,
truly deserving the honor of lower case letters, as if it were of Greek or Latin origin.

Indeed, the idea of radar has become so common that it is used in very
different contexts.  A social, political, or economic issue becomes
important, and we hear expressions such as ``the discovery of taxol in
the bark of the yew has made the preservation of rare species show up
on the legislative radar screen.''  Or, just as interesting, we may
also read of entities attempting to avoid recognition, as in ``the
various industries which produce toxic waste are flying under the 
radar.''  Somehow the greater civilization has come to understand that 
``flying low'' has something to do with avoiding detection by radar, an 
idea which is basically correct, and about which we'll be able to say a 
few things in a subsequent chapter.

As it happens, not all radars are intended for range detection, and it
may be difficult to draw the line between ``radar'' and ``non radar''
instruments.  We will frequently give examples from our own radar,
which measures the complete range and Doppler spectrum of extended
targets --- and yet our radar has no transmitter.  Does it qualify as
a radar?  Judge for yourself.

The history of radar is fascinating.  Contrary to common belief, radar
was not suddenly discovered in the 1940s in service of World War II, but 
long before.  It is true that radar was of profound importance in WWII, 
because the Allies successfully developed and exploited it, while the
Axis did not\footnote{need a reference for this}.

\section{some applications --- needs fixing}

RADAR\index{RADAR} --- an acronym which means ``Radio Detection and Ranging''

basic idea: emit a pulse, wait for echoes, factor in $c$ to get
distance.

But radar is broad:
\begin{itemize}
\item ``picket fence'' (NAVSPASUR), CW ``burglar alarms,'' "tripwires"
\item weather radar (deep, volume targets)
\item bistatic, multistatic, forward scatter geometries
\item Synthetic Aperture
\item Planetary radar astronomy
\item passive radar
\item aerospace applications (point targets)
\end{itemize}
Incredibly broad set of applications, techniques, frequencies, costs,
complexities ... very difficult to give single set of design equations
that cover everything.

Need to acquire basic skills
\begin{itemize}
\item Analysis --- come to understand what a radar is doing, and why
\item Synthesis --- extend knowledge to design new systems.
\end{itemize}

compare to Forward Problems, Inverse Problems.

\section{Radar before World War 2}

It's possible to debate the origins of radar.  One possible origin
would be with Hertz and Maxwell at the end of the Nineteenth Century, 
who performed experiments in which a transmitter produced a wave, a 
receiver detected the wave, and output of the receiver was affected by the presence of a third object.

There are some earlier experiments worthy of mention, associated with
attempts (mostly unsuccessful) to measure the speed of light\footnote{need 
references; lantern unveiling, michelson-morley}.  Electromagnetic waves travel 
stupendously fast in human terms.  However light is almost sluggish in ways 
which limit high speed electronics.  Modern desktop CPUs can execute several 
instructions in the time it takes a radio wave to span the container in which the 
computer resides. However, our scientific forebears did not have access to equipment
which could easily split the second into billionths.  Even with the best
portable clocks using some variant of spring and mass, it's clear that
efforts to time the uncovering of lanterns on mountaintops (i.e. Ben
Franklin et al.) were doomed to failure.

\subsection{Well Before Maxwell}

These failures were not due to fundamental limitations in the clocks,
but the short distance over which the measurement was attempted.  If
you can resolve 1 microsecond, then 1 km is about the minimum size of
an experiment which can measure the speed of light.  If your time
resolution is one second, then the experiment should occupy more than
300,000 km, roughly the distance to the Moon.  If you wish to resolve
four decimal places with a clock which ticks seconds, the experiment
must be a thousand times larger, about the distance to the Sun.

Dutch astronomer Ole Roemer (years) was able to perform this experiment
with breathtaking precision.  With the invention of the telescope,
astronomers were able to measure the orbital period of the moons of
Jupiter, and that these orbital periods were very well defined --- in
other words, they are a very stable clock.  Roemer noticed that the
time of ``rise'' of these moons depended upon whether Jupiter was near
or far from Earth.  Since Jupiter's orbit has 5.5 times the diameter
of Earth, it is sometimes as near as 4.5 $R_e$ and as far as 6.5
$R_e$, a differential distance of about 300 million km, about a thousand
seconds of light travel time.  Unfortunately, in Roemer's day the
distance from the Earth to the Sun was not known\footnote{It is interesting
to contemplate: how would you measure the distance to the Sun?}, although the ratio
of Jupiter's to Earth's orbital radius (5.5) was known.  So, Roemer
was able to express the speed of light (simplifying greatly) as 
\begin{displaymath}
c = \frac{2R_e}{1000}
\end{displaymath}
The symbol $c$ is derived from ``celeritas'' which is Latin for
``velocity,'' and which symbol was first used by Newton.

\subsection{Michelson-Morley}

The Michelson-Morley experiment (1880s) was developed to measure
the speed of light with sufficient precision to formally test a
hypothesis of the ``ether'' theory of electromagnetic wave
propagation.  The experiment resembled Roemer's observation except
that the path and time scales were much shorter.

Michelson and Morley developed a ``coincidence detector'' which in
which a pulse of light appeared at a detector only when it traversed a
fixed (known) path and managed to reflect from two rotating mirrors at
exactly the right moment in their rotation.  By varying the rotation
speed of the two mirrors, different candidate speeds of light could be
tested (stating the experiment this way foreshadows the ``matched
filter and ambiguity function'' topics to come.).  The two mirrors
were actually different faces of an octagonal cylinder, and indeed
there are several solutions to the problem, but the least rotational
speed which yields a stable image unambiguously provides the speed of
light.  The timescale of the experiment was under a millisecond, and
the time resolution less than a microsecond, so the experiment fit
onto an Earthly laboratory (a hundred meters or less).

\subsection{James Clerk Maxwell}

Maxwell was a prolific physicist.  Among his major accomplishments was
a unification of the electric and magnetic fields through what are now
known as ``Maxwell's Equations.''  In modern (differential) form, and MKSA units,
these equations are as follows
\begin{eqnarray}
\nabla\times \vec{E} &=& -\frac{\partial \vec{B}}{\partial t} \\
\nabla\times \vec{H} &=&  \frac{\partial \vec{D}}{\partial t} + \vec{J} \\
\nabla\cdot \vec{D}  &=& \rho \\
\nabla\cdot \vec{B}  &=& 0
\end{eqnarray}
Maxwell's particular contribution was the addition of the $\partial
\vec{D}/\partial t$ term.  Again, in modern notation, if we apply the
constitutive relations
\begin{eqnarray}
\vec{D}    &=& \epsilon \vec{E} \\
\epsilon_0 &=& 8.854...\times 10^{-12} \textrm{F/m} \\
\vec{B}    &=& \mu \vec{H} \\
\mu_0      &=& 4\pi \times 10^{-7} \textrm{H/m}
\end{eqnarray}
and develop a wave equation for the electric or magnetic fields, we
find
\begin{equation}
\left(\nabla^2 - \mu \epsilon \frac{\partial^2}{\partial t^2}\right) \vec{E}  = 0
\end{equation}
Here $\mu \epsilon$ is recognized to have the units of inverse
velocity (squared) and in free space that velocity is
\begin{equation}
c = \sqrt{\frac{1}{\mu_0 \epsilon_0}} \approx 3 \times 10^8 \;
\textrm{m/s}
\end{equation}
There are several important features of this development.
\begin{itemize}
\item the vacuum propagation speed $c$ is independent of time and
length scales --- thus radio waves and light waves become different
aspects of the same phenomenon, as well as infrared light, x rays, and
gamma rays.
\item the wave equation is not dispersive (in vacuum) so that
well-formed pulses retain their shape as they propagate.  
\item $\mu$ and $\epsilon$ can in principle be measured in
electrostatic and magnetostatic experiments (the constitutive
relations), so the that speed of propagation $c$ can be deduced from
very different experiments from ordinary ``speed trials.''
\end{itemize}
Michelson and Morley and others performed their experiment to address
troubling questions that arise if two observers with different
velocities observe the same electromagnetic experiment: when we say
``the speed of light is $3\times10^8$ m/s,'' what should the person
who is moving west at 10 m/s say?  The answer turns out to be
``Special Relativity,'' which means that Maxwell's Equations are
ultimately far more important than descriptors of electromagnetic
waves: they forced the development of the theories of Special and
General Relativity.  From the moment that Maxwell's Equations for
electromagnetics were offered, twenty five years passed before
Einstein published his theory of Special Relativity.  In retrospect it
needn't have taken this long, but theories of relativity require us to
think rather profoundly about the nature of space-time, as opposed to
the more immediate view that space and time are largely independent,
and certainly not related by the motion of observers.

\subsection{Early Studies of the Ionosphere}

Ionosondes --- 1918; Eckersley 1937.

\subsection{early aerospace}

Passive Radar in England

\section{From World War 2 to the Digital Age}

Although radar was not really invented in World War 2, the explosion
of research and development during and immediately after WW2
fundamentally colored the perception of the radar problem, its
nomenclature, and established its importance as a tool in modern
warfare and national defense.

\subsection{VHF, UHF, EHF, sources and detectors}

\subsection{algorithms: MTI, conical scan, monopulse}

WWII and the Cold War

\subsection{Big Radars}

\subsubsection{BMEWS/DEW Radars}

\subsubsection{Planetary Radar Astronomy}

\subsubsection{Thomson Scatter Radars}

\section{The Early Digital Age}

\subsection{SAR and ISAR}

\subsection{phased arrays}

\section{The Modern Era: since 1985}

\subsection{high speed elecronics}

\subsection{time and frequency reference}

\section{Why Study Radar?}

If you're going to be a radar technician, the answer is obvious.  But
suppose you are \textbf{not} going to be a radar jock.  Is radar still
worth looking at?

\subsection{Radar as dual of Communications}

In Communications, you try to send an unknown data signal through a fairly
well-known channel, in hopes of recovering the data

In Radar, you send a fairly well-known signal through a poorly-known
channel, in hopes of recovering the structure of the channel.

Note, in some communications systems there is a ``training'' sequence
in which a known signal is sent through the channel to discover the
approximate impulse response of the channel.

\subsection{Radar as an Inverse Problem}

If we know the strength of the transmitter, the distance to the target, the size of the target, 
then we can predict how much signal will be received.  This is a \textit{Forward Problem}, 
which is an important class of problems to understand.  

However, more interesting would be the following: if we transmit a known signal, and at some
later time receive an echo of that signal, what can we deduce about the nature of the target that 
caused the reflection?  This is an \textit{Inverse Problem}.

To cast this differently, on a billiards table, if you know the position of all the balls, and the frictional
forces that slow them, then upon striking the cue with a known vector impulse, it should be possible
to deduce the terminal position of all the balls --- a forward problem.

On the other hand, suppose you viewed a configuration of balls, and were asked to determine the initial 
configuration as well as  the impulse given to the cue
As presented, this problem is hopeless.   But suppose that one was able to know "how many ball-ball
collisions were there?" and "what were the instants in which they occurred?"  This problem would still
be very difficult, but the number of solutions could, conceivably be manageable.


\subsection{Radar as driver for Engineering Art}

Radar requires extremes of performance:
\begin{itemize}
\item RF systems
  \begin{itemize}
  \item low power/weak signals
  \item high power/transmitters
  \item very fast switches
  \item Antennas
  \item RFI/EMC issues
  \end{itemize}
\item Signal Processing
  \begin{itemize}
  \item Detection \& Estimation Theory
  \item Parameter Estimation --- Inverse Problems
  \item Tracking
  \item Filtering
  \item Statistics (lots!)
  \end{itemize}
\item Digital Systems/Mixed Signal Systems
  \begin{itemize}
  \item Data Acquisition
  \item Data Transport
  \item Data Storage
  \item ASIC design
  \item EMC/EMP robustness
  \end{itemize}
\item Software
  \begin{itemize}
  \item (Graphical) User Interface design
  \item Real Time Systems
  \item Reliability/Robustness
  \item Networking
  \item Documentation
  \end{itemize}
\end{itemize}


