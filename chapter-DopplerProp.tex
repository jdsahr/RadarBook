\chapter{Propagation and Doppler Shift}

\setlength{\arraycolsep}{0.5ex}

Now we'll review propagation of radio waves.  The simplest model is an
empty Galilean (non relativistic) universe containing a single target
and a transmitter located at the origin.  Subsequently we'll permit
the target to move, and even allow a varying media through which the
wave travels to the ``target.''

\section{Propagation in a simple universe}

We've previously considered the amplitude of radar signals with the
radar equation; now we'll consider other ways in which the signal is
modified by delaying it and by Doppler shifting it.  We can simplify
the discussion by considering a one dimensional universe.

The transmitter emits some waveform $s(ct)$ which then propagates away
from the transmitter, so that space-time contains the signal $s(ct -
x)$ (consider $x > 0$ for now).  Notice that $ct$ has length units,
which is conceptually useful and also connects radar to the theory of
special relativity.

The target located at $x_1$ is thus bathed in the signal $s(ct - x_1)$
and it scatters a fraction of that signal, $f s((ct - x_1) + (x
- x_1))$.  Here $f$ indicates the scattering amplitude, not the
scattering cross section $\sigma$.  Letting $f$ be a simple scalar
hides the realistic possibility that the scattering process may not be
uniform for all frequencies --- thus the scattering amplitude should
be more generally represented as an impulse response convolved with
the illuminating signal; and even this does not admit the 
possibility of a nonlinear scattering process.  Thus a little more
generally, but not particularly usefully, we can write that
backscattered signal $u(t,x) = \mathcal{S}[s(ct - x_1) + (x -
x_1),x,t]$, where $\mathcal S$ is a nonlinear operator dependent upon
space and time.

When the scattered signal returns to the transmitter, it can be
described as $f s((ct - x_1) + (0 - x1)) = f s(ct - 2x_1)$,
the factor $2$ appearing to indicate the round trip.  If we compute
the ``phase velocity'' $x/t = c/2$ we have a quantity that arises
frequently in radar, the ``speed of radar.''  This speed is certainly
$c/2 = 1.5\times 10^8$ m/s, but in units more convenient for radar
technology it is also 150 km/ms and 150 m/$\mu$s.  Thus the echo of a
target at a range of 150 km returns to the transmitter one millisecond
after its transmission, and if the receiver is sampled at 1 $\mu$s
intervals, each sample represents 150 m of range increment.

\newenvironment{factoid}{}{}

\begin{factoid} The ``Speed of Radar.''  The factor $c/2$ arises
  frequently in radar, and has the value $1.5\times 10^8$ m/s.
  However, it is very conveniently expressed additionally as
\begin{displaymath}
\frac{c}{2} = 150 \;\textrm{m}/\mu\textrm{s} = 150 \;\textrm{km/ms}
\end{displaymath}
\end{factoid}


\section{Galilean Doppler Shift}

Let us suppose that the target at $x_1$ is not fixed, but moving with
some velocity, so that $x_1 \rightarrow x_1 + vt$.  If we illuminate the target
with a sinusoid, $s(ct) = \cos(\omega t)$ then the returning
scattered wave will be
\begin{eqnarray*}
\sigma \cos(\omega t - 2 k (x_1 + vt)) &\sim& 
  \cos((\omega - 2kv) t - 2kx_1) \\
&=&\cos((\omega + \Delta \omega)t - 2kx_1) \\
\Delta \omega &=& -2kv \\[0.5ex]
\Delta f      &=& -2\frac{v}{\lambda}
\end{eqnarray*}
Here $\Delta f$ is the Doppler Shift (in Hz) associated with motion of
the target away from the transmitter, in particular, negative Doppler
Shift is associated with increasing distance ($v > 0$) while positive
Doppler Shift indicates decreasing distance.

Both positive and negative frequencies are possible Doppler shifts;
so we need to accomodate positive and negative frequencies in spectrum
estimation of radar signals.  In conventional signal processing
negative frequency content is implicitly paired with positive
frequency content. In radar signal processing we will find that the
``natural'' time series for radar signal processing is complex valued,
not real valued, and that this provides a simple means to preserve the
sign of the Doppler shift.

\section{Relativistic Doppler Shift}

In the previous derivation of the Doppler Shift, one might be troubled
that a new frequency is generated without changing the wavenumber;
thus the speed of propagation must change.  One could fix this problem
by self consistently changing the wavenumber through a more
sophisticated argument.  However a much more powerful approach is to
rely directly upon Special Relativity.

Consider three events: $\mathcal{A}$, a pulse of frequency $\omega$ is
emitted from the transmitter; $\mathcal{O}$, the pulse encounters and
scatters from a target; $\mathcal{B}$, the scattered signal returns to
the transmitter.  For convenience, let us place the origin of all
coordinate systems at $\mathcal{O}$, aligning the spatial axes of all
coordinate systems; let the relative motion of the coordinate systems
lie along the $x$ axis.  Thus, Lorentz Transformations of 4-vectors
will affect only $x$ and $t$ axes.

In its inertial frame, let the transmitter be located at $(x,y,z) =
(x_t, y_t, 0)$.  The motion of the target is described in the
transmitter frame as $x(t) = Vt, y = z = 0$.  The times of events
$\mathcal{A}$ and $\mathcal{B}$ are easy to calculate,
\begin{equation} \label{e:relativity1}
ct_{\mathcal{A},\mathcal{B}} = \mp \sqrt{x_t^2 + y_t^2} = \mp r_t
\end{equation}
These can be easily converted to times in the target frame of
reference through the Lorentz Tranformation:
\begin{equation} \label{e:relativity2}
ct_{\mathcal{A},\mathcal{B}}' = 
  \gamma (ct_{\mathcal{A},\mathcal{B}} - \beta x_t) =
  \gamma (\mp r_t - \beta x_t)
\end{equation}
Note that although the times of flight inbound and outbound are
identical in the transmitter rest frame, but not (in general) in the
target frame.

In the rest frame of the target, the transmitter is moving, and its
motion is described as $(x', y', z') = (x'_t - Vt', y'_t, 0)$.  The
frequency of the incoming and outgoing waves is the same, say
$\omega'$.  The total phase accumulated along the world lines
$\mathcal{AO}$ and $\mathcal{OB}$ is
\begin{eqnarray}
\phi'_{\mathcal{AO}} &=& 
 \frac{\omega'}{c}\left(0 - ct'_\mathcal{A}\right) = 
 \frac{\omega'}{c}\left(\gamma(r_t - \beta x_t)\right) \\
\phi'_{\mathcal{OB}} &=& 
 \frac{\omega'}{c}\left(ct'_\mathcal{A} - 0\right) = 
 \frac{\omega'}{c}\left(\gamma(r_t + \beta x_t)\right)
\end{eqnarray}

Since the phase is invariant to Lorentz Transformation, $\phi =
\phi'$.  Thus, the frequencies in the frame of the transmitter are
given by
\begin{equation}
\omega_{\mathcal{AO}, \mathcal{OB}} = 
\frac{c\phi_{\mathcal{AO}, \mathcal{OB}}}
            {ct_{\mathcal{A},\mathcal{B}}} =
\frac{\gamma(r_t \mp \beta x_t)}{r_t}\omega'
\end{equation}
Thus, the change in frequency is easily expressed as a ratio,
\begin{equation}
\frac{f_r}{f_t} = \frac{\omega_{\mathcal{OB}}}{\omega_{\mathcal{AO}}}
= \frac{r_t - \beta x_t}{r_t + \beta x_t}
\end{equation}
With $\cos(\theta) = x_t/r_t$ this may be written simply as
\begin{eqnarray}
\frac{f_r}{f_t} &=& \frac{1 - \beta \cos(\theta)}{1 + \beta \cos(\theta)}\\
\frac{\Delta f}{f_t} = \frac{f_r - f_t}{f_t} &=& \frac{-2\beta\cos(\theta)}{1 + \beta\cos{\theta}}
\end{eqnarray}
The angle $\theta$ is determined by the radio wave propagation
direction and the target velocity vector, such that $\theta = 0$
corresponds to the target receding along the line of sight.

Expanding the relativistic Doppler shift expression in $\beta$ we have
\begin{equation}
\frac{f_r}{f_t} = 
1 - 2\beta\cos(\theta) + 2(\beta\cos(\theta))^2 + \cdots
\end{equation}
To first order in $\beta$ this agrees with the non relativistic
development.

Note that the target will (in general) see a Doppler shift of the
transmitter signal even if there is no net Doppler shift at the
receiver ($\cos(\theta) = 0$).  In fact, the target's frequency shift
and radar frequency shift need not have the same sign.

However, in the ordinary non relativistic case, the target Doppler
shift will be half the radar Doppler shift, and relativistic
corrections will be quite small.

\begin{example} radar observation of meteor ``head echoes.''  Among the
fastest targets commonly seen with radars are meteors.  Near Earth
meteors may have velocity up to abuot 75 km/sec.  Relativistic
corrections to Doppler phenomena would have size
\begin{displaymath}
\beta = \frac{v}{c} = \frac{75\times 10^3}{3\times 10^8} = 25\times
10^{-5} = 0.00025
\end{displaymath}
Orbital velocity is 7.5 km/sec ($\beta = 2.5\times 10^{-5}$) and
aircraft travelling at Mach 2 travel 750 m/s ($\beta = 2.5\times
10^{-6}$).
\end{example}

\begin{factoid} The speed of light is just about one million times faster
than the speed of sound in air; the speed of light is Mach One
Million.
\end{factoid}

\section{Doppler Dispersion}

The Doppler Shift expressions show that velocity applies a constant
rescaling to the transmitter frequency.  If the transmitter waveform
has nonzero bandwidth (described by a Fourier spectrum $P(f)\, df$),
then the differential Doppler shift of higher frequencies is greater
than that of lower frequencies.  Thus target motion causes the
modification of the spectrum not as $P(f + \Delta f) \, df$ but rather
as $P(\alpha f) \,df\!/\!\alpha$.  However, in the usual non
relativistic case, with narrowband transmitter waveforms, $\alpha = 1
+ \epsilon$, $|\epsilon| \ll 1$, so that $P(\alpha f) \approx P(f +
\epsilon f_t) = P(f + \Delta f)$, where $f_t$ is the center frequency
of the transmitter.

When the bandwidth of the scattered signal exceeds about one percent
of the transmitter center frequency, this spectral asymmetry may need
to be accounted for in precision measurements.  The circumstance
arises in VHF and UHF radar observations of the ``plasma line'' in the
ionosphere; the plasma line induces a shift (modulation) on the order
of 10 MHz at the peak of the ionosphere (about 350 km at noon).

\section{Variable Media}

When the medium through which the wave propagates is not ideal vacuum,
additional details must be accomodated.  If the medium is constant but
dispersive, a short pulse injected into the medium will be spread in
time.  If the medium has a time-varying index of refraction, scatter will
be generated from that change.  Let us consider that case, briefly.

Suppose that a transmitter is located at $x_1$ and a target at $x_2$
separated by a region of space whose scalar dielectric constant is
also a function of space, such that
\begin{eqnarray*}
\epsilon &=& \epsilon_0 + \epsilon_1(x,t) \\
0 &\le& \epsilon_1(x,t)  \ll \epsilon_0
\end{eqnarray*}
Let us further suppose that the time scale for variation of the medium
is slow compared to the transit time, namely
\begin{displaymath}
\epsilon_1(x,t) \left(\frac{\partial \epsilon_1(x,t)}{\partial t}\right)^{-1}
\ll \frac{|x_t - x_r|}{c}
\end{displaymath}
Then the elapsed phase is $2\int k dx = 2\int \omega/v_{ph} dx$.
\begin{displaymath}
\phi(t) = 4\pi \int_{x_t(t)}^{x_r(t)} \frac{f}{v_{ph}(x,t)} \;\; dx
\end{displaymath}
The Doppler Shift is simply $\partial \phi/\partial t$, which in this
fairly general case has three components due to the motion of the
transmitter, the motion of the receiver, and the changing of the
medium through which the radiowave travels.  Thus (frequently dropping
the slow time dependence)
\begin{displaymath}
\Delta \omega =  4\pi f\left(
\frac{\dot{x}_t}{v_{ph}(x_t)} - \frac{\dot{x}_r}{v_{ph}(x_r)} +
\int_{x_t}^{x_r} \frac{\partial}{\partial t}
\frac{dx}{v_{ph}(x,t)}
\right)
\end{displaymath}
The first two terms represent the ordinary Doppler Shift associated
with relative motion between the transmitter and scatterer.  The third
term shows that a Doppler shift (change in frequency of scattered wave)
can occur even if both the transmitter and target are fixed --- but
the medium between them fluctuates.
