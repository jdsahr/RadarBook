\chapter{Computation of the Ambiguity Function}
\label{s-ambiguity-function}

As pointed out in Chapter 3, the Ambiguity Function describes the resolution in range and Doppler 
provided by a radar waveform.  Unfortunately this function has a number of different but roughly 
equivalent forms.

The form used in this report is as follows, for a radar waveform $u(t)$:
\begin{equation}
\chi(\tau,\nu)  = \int u(t) u^\ast(t - \tau) e^{2 \pi j \nu t} \; dt
\end{equation}
Some authors prefer to take the absolute value of this expression; others take the absolute
value squared.  We prefer the complex-valued form above for  important but subtle reasons:
there is important information present in the phase, and the units of $\chi$ are effectively
field units (e.g. volts).

The importance of the phase was demonstrated by Meyer [2003]\footnote{@article{meyer2004passive,
  title={Passive coherent scatter radar interferometer implementation, observations, and analysis},
  author={Meyer, Melissa G and Sahr, John D},
  journal={Radio science},
  volume={39},
  number={3},
  year={2004},
  publisher={Wiley Online Library}
}}, who pointed out how to perform
multi-antenna interferometry by computing cross-ambiguity $\chi_{nm}$ on two or more antennas to 
image the angular distribution of scatter of a deep fluctuating targets.

\section{Self-Ambiguity}

In chapter 3 the Ambiguity function and its several properties were developed.  One of them was a 
symmetry property:
\begin{equation}
|\chi_{uu}(\tau, \nu)| = |\chi^\ast_{uu}(-\tau, -\nu)|
\end{equation}
With a slightly different (but completely equivalent) definition of the Ambiguity function, the
previous expression is true \textit{without} the absolute value operation:
\begin{equation}
\tilde{\chi}_{uu}(\tau, \nu) = \int u\!\left(t + \frac{\tau}{2}\right) u^\ast \!\left(t - \frac{\tau}{2}\right) e^{2 \pi j \nu t} \; dt \label{e:wigner}
\end{equation}
With this definition, the symmetry property is more simply (and powerfiully) stated:
\begin{equation}
\tilde{\chi}_{uu}(\tau, \nu) = \tilde{\chi}_{uu}(-\tau, -\nu) 
\end{equataion}
This form of the ambiguity function has some very nice properties, but it is not quite so convenient
for computation.  Fortunately it is very easy to compute from $\chi_{uu}(\tau,\nu)$:
\begin{equation}
\tilde{\chi}_{uu}(\tau, \nu) = e^{-\pi j \nu \tau} \chi_{uu}(\tau, \nu)
\end{equation}

\subsection{Wigner Distribution}
The definition of $\tilde{\chi}_{uu}(\tau, \nu)$ in \eqref{wigner} is also known as the Wigner 
Distribution\footnote{see \textt{https://en.wikipedia.org/wiki/Wigner\_distribution\_function}}, which
arises in some problems in signal analysis and quantum mechanics.  As such it has a number of 
properties beyond those mentioned in Chapter 3.
\begin{eqnarray}
|x(t)|^2 &=& \int \tilde{\chi}_{xx}(\tau, \nu) \; d\nu \\
|X(\nu)|^2 &=& \int \tilde{\chi}_{xx}(\tau, \nu) \; d\tau \\
\int \tilde{\chi}_{xx}(\tau/2, \nu) e^{2\pi j \nu \tau} \; d\nu &=& x(t) x^\ast(0)  \\
\int \tilde{\chi}_{xx}(\tau, \nu/2) e^{2\pi j \nu \tau} \; d\tau &=& X(\nu) X^\ast(0) \\
x(t) \rightarrow \tilde{\chi}_{xx}(\tau, \nu) &\leftrightarrow& 
   x(t-\eta) \rightarrow e^{2\pi j \nu \eta} \tilde{\chi}_{xx}(\tau, \nu) \\
x(t) \rightarrow \tilde{\chi}_{xx}(\tau, \nu) &\leftrightarrow& 
   x(t) e^{2 \pi j f t} \rightarrow e^{2\pi j f \tau} \tilde{\chi}_{xx}(\tau, \nu) 
 \end{eqnarray}

\section{Cross-ambiguity}

The cross ambiguity $\chi_{nm}$ is defined as follows, describing a signal that was received on
antenna $n$, with a reference to the transmitter signal provided on antenna $m$.:
\begin{equation}
\chi_{nm}(\tau,\nu)  = \int u_n(t) u_m^\ast(t - \tau) e^{2 \pi j \nu t} \; dt
\end{equation}

\section{Cross-ambiguity in the presence of zero doppler ground clutter}

Suppose that the field of view contains a variety of scattering sources of strength $\alpha_k$ at 
group delay $\tau_k$.

Then the detected signal (following application of the matched filter) is the cross-ambiguity:

\begin{eqnarray}
\chi'_{nm}(\tau,\nu) &=& \sum_k \alpha_k \int u_n(t - \tau_n) u_m^\ast(t - \tau) e^{2 \pi j \nu t} \; dt \\
&=& \sum_k \alpha_k \chi_{nm}(\tau - \tau_k, \nu)
\end{eqnarray}
Equation D.4 is essentially the starting point for the Manastash Ridge Radar (Sahr, Lind 1997), and 
applies when the target extends over group delays that are much larger than the correlation time
of the illumination $u_m(t)$.

In the case of close-in ground clutter, inside the correlation time of $u_m(t)$ this formula is not directly 
useful.

By using Equation D.10 above, we can rewrite D.14 as an interesting theorem:

\begin{equation}
\chi'_{nm}(\tau,\nu) = \chi_{nm} (\tau,\nu) \sum_k \alpha_k e^{-2\pi j \nu \tau_k}
\end{equation}


Alternatively we could create the ratio
\begin{equation}
\frac{\chi'_{nn}(\tau,\nu)}{\chi_{nn} (\tau,\nu)} = \sum_k \alpha_k e^{-2\pi j \nu \tau_k}
\end{equation}
The numerator on the left can be directly estimated from data; the denominator can be estimated if the
the uncorrupted signal $u_n(t)$ can be estimated.  Finally, the right hand side is a computable
linear combination of the $\alpha_k$ if the $\tau_k$ are known.  However, it can also be viewed as
a Fourier Transform of the $\alpha_k$ for a set $\tau_k = k\tau_0$.


