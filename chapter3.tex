% $Id: chapter3.tex,v 1.6 2005/02/01 01:16:12 jdsahr Exp $

\chapter{Receivers}

A receiver is the sensor which converts the scattered radio wave into
an electic signal which is suitable for analysis.

\section{Basic Receiver topologies}

\subsection{noncoherent}

power only detector.  Extremely simple.  Not very sensitive.  Can't do
Doppler processing (probably).

\subsection{direct conversion}

set LO to transmitter frequency.
advantages:
\begin{itemize}
\item simple
\end{itemize}

disadvantages
\begin{itemize}
\item LO pollution in the RX spectrum
\end{itemize}

\subsection{down conversion}

Two (or more) frequency conversion operations.
advantages
\begin{itemize}
\item higher performance
\item no LO in RF spectrum
\item access to good filter technologies
\end{itemize}

disadvantages
\begin{itemize}
\item higher cost
\end{itemize}


\subsection{analog nonlinearities in mixers and amplifiers}

High performance receivers and measuring equipment (e.g. vector
network analyzers) are frequently are performance-limited by
nonlinearities lurking in various places in the system.

Nonlinear behavior in components is not only difficult to accomodate;
it is difficult to even describe, because it arises in many different
ways.  

A ``simple'' kind of nonlinearity arises because of finite limits on
the power supply voltage for components --- clipping will eventually
set in.  More subtle are the weak, intrinsic nonlinearities offered by
amplifying and switching devices.

In ``mixers'' we hope to achieve an ideal analog multiplication.
Given two input signals $a(t), b(t)$ we would like to produce the
product signal $c(t) = a(t) b(t)$.  For example, if $a(t), b(t)$ are
a pair of sinusoidal signals with frequencies $f_a, f_b$, then the
output signal $c(t)$ will contain a pair of sinusoids with
frequencies $|f_a \pm f_b|$.

\subsubsection{Square-Law systems}

There is a basic topological problem with achieving an analog
multiply operation: two inputs and one output.  However there is a
trick that can be used to achieve multiplication if you have access
to ``addition'' and ``squaring.''

Consider two signals $a(t), b(t)$.  If their sum $a(t) + b(t)$ is
passed through a squaring operation you have
\begin{equation}
(a(t) + b(t))^2 = a(t)^2 + b(t)^2 + 2 a(t) b(t)
\end{equation}
The first two terms are not desired, but the third is.  It may be
possible to sufficiently suppress the first two by filtering. 

However, it is useful to consider the square of the difference, too.
\begin{equation}
(a + b)^2 - (a - b)^2 = 4 ab
\end{equation}
This is a perfect multiply!  However, achieving this in analog
circuits requires careful matching of the summing and squaring
circuits to achieve the mathematical cancellation.

\subsubsection{Junction Diodes}

Junction Diodes are frequently used for mixers (multipliers) because
of their well characterized nonlinear (exponential) current-voltage
characteristic.
\begin{eqnarray}
I &=& I_0\left( e^{V/V_{th}} - 1 \right) \\
I &=& I_0\left(\frac{V}{V_{th}}
+ \frac{1}{2}\left(\frac{V}{V_{th}}\right)^2 + \cdots \right)
\end{eqnarray}

\subsubsection{Diode Rings}

concept of double balanced

\subsubsection{Gilbert Cell}

active mixer/diffamp gadget

\subsubsection{Parametric}

use a varactor ...

\section{In Phase and Quadrature receivers}

Usually wish to use the slowest possible sampler to minimize data flow
through processor.  Lowest data rate is at baseband; achieve with some
sort of single or multi conversion receiver.

Basic problem --- can't tell sign of Doppler shift (spectrum of
real-valued signals are symmetric).

\begin{eqnarray}
s(t) &=& A \cos(2\pi(f_0 + \Delta f)t) \\
l(t) &=& \cos(2\pi f_0 t) \\
s(t)l(t) &=& A \cos(2\pi(f_0 + \Delta f)t)\cos(2\pi f_0 t) \\
&=& \frac{A}{2}\cos(2\pi \Delta f \,t) + \frac{A}{2}\cos(2\pi (2f_0)
t) \\
\textrm{LP}[s(t)l(t)] &=& \frac{A}{2}\cos(2\pi \Delta f \,t)
\end{eqnarray}
(LP indicates a ``low pass filter'' operation.)  In the last line,
notice that you wouldn't be able to notice if $\Delta f$ changed sign.
Same is true if $l(t) = \sin(2\pi f_0 t)$ --- the sign would change,
but that could be absorbed into a phase shift.

Idea: use both?


\subsection{The IQ signal}

If you use both cos and sin for the last detection, you get
\begin{eqnarray}
s(t) &=& A \cos(2\pi(f_0 + \Delta f)t) \nonumber \\
 &=& \frac{A}{2}\left( \exp(2\pi j(f_0 + \Delta f)t) + \exp(-2\pi
 j(f_0 + \Delta f)t) \right) \nonumber \\
l(t) &=& \exp(-2\pi j f_0 t) \\
s(t)l(t) &=& \frac{A}{2} \exp(2\pi j\Delta f)t) + 
\frac{A}{2} \exp(2\pi j(-2f_0 - \Delta f)t) \\
\textrm{LP}[s(t)l(t)] &=& \frac{A}{2}\exp(2\pi j \Delta f \,t)
\end{eqnarray}

Now the low pass signal has the unambiguous Doppler shift apparent.
This is a very convenient signal.  Although it may be puzzling to
think of at first, it really is as simple as having two wires, one
with the real part and one with the imaginary part.

\subsection{The Analytic Signal}

In many treatments of radar and radio communications systems the
theory of ``analytic signals'' and the Hilbert Transform are
introduced at this point.  

Although fascinating theory, knowing or not knowing that theory does
not preclude one from using or understanding the IQ receiver.  We
direct your attention to \cite{whalen,vantrees} if you would like to
learn more about this.  For the moment, suffice it to say that one can
formally create an ``analytic signal'' $\tilde{u}(t)$ from a real
signal $u(t)$ as follows:
\begin{eqnarray}
\tilde{u}(t) &=& u(t) + \jmath u_H(t) \\
u_H(t) &=& {\cal H}[u(t)] = u(t) \ast \frac{1}{\pi t} \\
       &=& \frac{1}{\pi} P \int_{-\infty}^\infty \frac{u(s)}{s - t}\; ds
\end{eqnarray}
Here $P\!\!\int$ means that the ``principle value'' of the integral is
taken, avoiding the singularity at $s=0$.  The (complex valued)
analytic signal $\tilde{u}(t)$ has the interesting property that it
has identically zero spectral content for negative frequencies. 

\subsection{analog IQ receivers}

In an analog IQ receiver, near the end of the downconversion chain,
the local oscillator will be split and phase shifted, with a cosine()
signal on one wire, and a sine() signal on the other.  The original
signal is then mixed against each of these oscillators and lowpass
filtered to produce the desired IQ baseband signal.

problems
\begin{itemize}
\item such mixers often (always?) have a DC offset, which leads to
  excess power in the zero frequency spectral channel.
\item different gain through I and Q channels leads to L-R spectrum
  pollution.
\item imperfect 90 degree phase shift between cosine and sine leads to
  L-R spectrum pollution
\item if the two LP filters in the I and Q channels are not perfectly
  matched, results in (complicated) L-R spectrum pollution.
\item Different ADCs could have different gains, nonlinearities ...
\end{itemize}

\section{Digital Receivers}

Modern digital receivers eliminate many of the limitations of
conventional analog IQ detectors, by implementing the last down
(IQ) conversion and subsequent low pass filtering entirely as digital
computations. 

advantages
\begin{itemize}
\item A single digitizer eliminates differential performance
\item $\exp(2\pi j f_0 t)$ is generated digitally --- ``perfect''
  quadrature is available
\item mixing/down conversion operations are digital, hence the I and Q
  channels are essentially perfectly matched.
\end{itemize}

disadvantages
\begin{itemize}
\item probably too expensive for very cheap receivers
\item ...?
\end{itemize}

Basically they are wonderful.


\section{The Matched Filter}

 When a transmitter emits a signal $g(t)$, it scatters from a target
and is detected by a receiver with an impulse response $h(t)$, so that
                   the detector output is $s_o(t)$,
\begin{eqnarray}
s_o(t) &=& h(t)\ast g(t) \\
       &=& \int h(\tau) g(t-\tau) d \tau
\end{eqnarray}
Of course, noise will be entering the receiver, too.  Letting $n(t)$
be the input noise, the output noise $n_o(t)$ is just 
\begin{eqnarray}
n_o(t) &=& h(t)\ast n(t) \\
       &=& \int h(\tau) n(t - \tau) d \tau
\end{eqnarray}
This is a mathematically handy step, because $n(t)$ has infinite
bandwidth.  This issue lead to the so-called ``ultraviolet
catastrophe'' early in the Twentieth Century.  If the power spectral
density is constant, and the noise has infinite bandwidth, then that
means that noise has infinite power!  The solution was provided by
Quantum Mechanics and the Planck Radiation Law, which showed that,
above a certain, temperature-dependent frequency, the noise power
spectral density falls of quite dramatically --- yielding finite noise
power at finite temperature, for infinite
bandwidth\footnote{buzzwords: Planck Radiation Law, Rayleigh-Jeans
  Law, Blackbody Radiation}.

So, how much signal power and noise power do we get?  Start with noise
power. 
\begin{eqnarray}
P_n &=& \langle n_o(t)^2 \rangle \\
    &=& \left\langle \left[ \int h(\tau) n(t-\tau) d\tau
    \right]^2\right\rangle \\
    &=& \left\langle \left[ 
    \int h(\tau) n(t-\tau) d\tau \int h(\tau') n(t-\tau') d\tau'
    \right]\right\rangle \\
    &=& 
    \int \int h(\tau) h(\tau') \langle n(t-\tau) n(t-\tau') \rangle
    d\tau' d\tau \\
    &=& \int \int h(\tau) h(\tau') N_0 \delta(\tau - \tau') d\tau' d\tau \\
    &=& N_0 \int \left| h(\tau)\right|^2 d\tau \\
P_n &=& \frac{N_0}{2} \int_{-\infty}^\infty \left| H(f) \right|^2 df
\end{eqnarray}

Now, what is the signal power $P_s$?
\begin{eqnarray}
P_s &=& (s_o(t))^2 \\
P_s &=& \left| \int h(\tau) g(t-\tau) \; d\tau \right|^2 \\
    &=& \left| \int H(f) G(f) e^{2\pi j f t} \; df \right|^2
\end{eqnarray}
Now, let's look at the Signal to Noise Ratio\index{signal to noise
ratio} (SNR\index{signal to noise ratio!SNR}),
\begin{equation}
\textrm{SNR} = \frac{P_s}{P_n} = 
\frac{\left| \int_{-\infty}^\infty H(f) G(f) e^{2\pi j f t} \; df \right|^2}
      {(N_0/2)  \int_{-\infty}^\infty \left| H(f) \right|^2 df}
\end{equation}
We would like to maximize this ration, but choosing the right time
$t$ to observe the signal, as well as the best impulse response
$h(t) \leftrightarrow H(f)$ given the transmitter waveform $g(t)$ to
maximize the ratio.  

There are a variety of ways to go about this, but the most
straightforward approach makes use of the Cauchy-Schwartz
Inequality\index{Cauchy-Schwartz Inequality}, which states that, for a
pair of (possibly complex valued) functions $A(p), B(p)$, we have
\begin{equation}
\left|\int A(p) B(p) \; dp \right|^2 \le \int |A(p)|^2 \; dp \int |B(p)|^2\; dp
\end{equation}
and, furthermore equality is achieved if and only if $A(p) =
kB^\ast(p)$ for some (possibly complex) scalar constant $k$.

In this case, by identifying $H(f) = A(f)$ and
$G(f)\exp(2\pi j f t) = B(f)$ we see immediately that the maximum SNR
is achieved when
\begin{eqnarray}
G(f) e^{2\pi j f t} &=& kH^\ast(f) \\
g(\tau) &=& kh^\ast(\tau)
\end{eqnarray}
for some constant $k$.  Also, the value of the maximum SNR can be
computed:
\begin{eqnarray}
\textrm{SNR}_{\textrm{max}} &=& 
  \frac{\int |H(f)|^2\; df  \; \int |G(f) e^{2\pi j f t}|^2 \; df}
      {(N_0/2)  \int\left| H(f) \right|^2 df} \\
  &=& \frac{\int |G(f) e^{2\pi j f t}|^2 \; df}{(N_0/2)} \\
  &=& \left. \frac{P_o(t)}{N_0/2} \right|_{\textrm{t=t}_m}
\end{eqnarray}
Here, $t_m$ is the time which maximizes the ratio ---
i.e. appropriate to the time-of-flight --- and $N_0$ is the one-sided
noise power spectral density $k_B T_{sys}$.

It is worth commenting on some limitations of the matched filter.
\begin{itemize}
\item The development above was predicated upon the whiteness of the
noise (flat spectrum).  However, a non-white noise spectrum can be
accomodated with a straightforward extension, which is to
``pre-whiten'' the detector.  See \cite{whalen-1971} for details.
\item The development above ignored the possibility of clutter.  The
matched filter is not necessarily optimal in the presence of
clutter.  However, in practice the matched filter is likely to be
near-optimal.
\item The development above ignored the possibility of several
targets.  This is a restatement of the problem of clutter.  Although
the matched is not necessarily optimal, it is very likely to be near
optimal.
\end{itemize}

\subsection{Square Pulse}

Transmit a square pulse $g(t)$ of duration $\tau$, matched filter
detect, get a triangle function $s_o(t) = g(t)\ast g(-t)$.

\centerline{figure}

This gives us a more sophisticated way of viewing range resolution
... 

\section{The Ambiguity Function}

\newcommand{\infint}{\int_{-\infty}^\infty}

Response of a (possibly matched) filter to a doppler shifted signal.

Suppose transmitter waveform $u(t)$, then matched impulse response is
$u^\ast(-t)$.  The output of matched filter is $u(t) \ast u^\ast(-t)$
evaluated at time $t$,
\begin{eqnarray}
u_o(t) &=& \infint u(t-s) u^\ast(-s) \; ds \\
&=& \infint u(t+s) u^\ast(s) \; ds \\
&=& \infint u(s) u^\ast(s-t) \; ds %\\
%&=& \infint u\left(s+\frac{t}{2}\right) u^\ast\left(s-\frac{t}{2}\right) \; ds 
\end{eqnarray}
Suppose we also Doppler shift the transmitter signal $u(t) \rightarrow
u(t) \exp(2\pi j \nu t)$, then response is
\begin{eqnarray}
\chi(\tau,\nu) &=& \infint u(s) e^{2\pi j \nu s} u^\ast(s-\tau)\;
ds  \label{e:AF-def}
\end{eqnarray}
Here $\chi(\tau,\nu)$ is the ``Ambiguity Function''.\index{ambiguity
function}  Note that several approximately equivalent forms are
defined by different authors.

The ambiguity function indicates the sensitivity of a matched filter
to point targets which are delayed by $\tau$ and Doppler shifted by
$\nu$ from the peak response at $\tau = 0$ and $\nu = 0$.

\subsection{Properties of the Ambiguity Function}

After \cite{levanon}

\begin{enumerate}
\item Normalization.  Using the Cauchy-Schartz inequality, we can
  rapidly show that 
\begin{equation} 
|\chi(\tau,\nu)| \le |\chi(0,0)| = 1
\end{equation}
where we scale the amplitude of $u(t)$ so that $\chi(0,0) = 1$.  The
implication is that the matched filter has its greatest sensitivity
when the targets are not relatively delayed, and have no Doppler
shift.
\item Conservation of Ambiguity.  
\begin{equation}
\int\!\!\infint |\chi(\tau,\nu)|^2 \; d\tau d\nu = 1
\end{equation}
This implies that, although the shape of the ambiguity surface can
change, the volume beneath that surface is preserved.  This has
tremendous ramifications for waveform design.
\item Symmetry.  
\begin{equation}
|\chi(\tau,\nu)| = |\chi(-\tau,-\nu)|
\end{equation}
\item Linear FM.  Application of a linear FM chirp distorts the
  ambiguity plane in a simple way:
\begin{equation}
\textrm{if}\;\; u(t) \rightarrow |\chi(\tau,\nu)| \;\;\; \textrm{then} \;\;\;
                u(t) \exp(\pi j \kappa t^2) \rightarrow |\chi(\tau,
                \nu + \kappa \tau)|
\end{equation}

\end{enumerate}

\subsection{Other Forms of the Ambiguity Function}

\subsubsection{Wigner Distribution}

A slightly different version of the ambiguity function is known as the
Wigner or Wigner-Ville Distribution. \cite{something}.
\begin{equation}
\chi_w(\tau,\nu) = \infint u(t + \tau/2) u^\ast(t-\tau/2) e^{2\pi j\nu t} \; dt 
\end{equation}
It is straightforward to show a simple relationship to $\chi$:
\begin{equation}
\chi_w(\tau,\nu) = e^{-\pi j \nu \tau} \chi(\tau, \nu)
\end{equation}
The Wigner Distribution has the nice property of complex symmetry, rather than
merely magnitude symmetry (property 3 above).
\begin{equation}
\chi_w(\tau,\nu) = \chi^\ast_w(-\tau,-\nu) 
\end{equation}
While $\chi_w$ is a little more awkward for computation, it may be more convenient for some proofs.
APPENDIX D elaborates on this topic.

\subsection{Fourier-Transformed form} 

Using the Fourier Transform relations found in appendix
\ref{fourier-transform}, page \pageref{fourier-transform}
we can show that
\begin{equation} \label{e:AF-def-alt}
\chi(\tau,\nu) = e^{-2\pi j \nu \tau} \infint \tilde{u}(f)\;
\tilde{u}^\ast(f-\nu) \;e^{2\pi j f \tau} \;df 
\end{equation}
If we look at the "wigner-form" we have
\begin{equation} \label{e:AF-def-alt}
\chi(\tau,\nu) = e^{-\pi j \nu \tau} \infint \tilde{u}(f + \nu/2)\;
\tilde{u}^\ast(f-\nu/2) \;e^{2\pi j f \tau} \;df  = e^{-\pi j \nu \tau} \chi_w(\tau,\nu)
\end{equation}
 


\subsection{Computation of the Ambiguity Function}

Depending upon whether one starts from \eqref{AF-def} or
\eqref{AF-def-alt} a few different approaches for computing the
Ambiguity Function are suggested.
\begin{itemize}
\item direct evaluation for a discrete set of $(\tau, \nu)$ pairs.
  The most efficient approach for just a few points. In general it
  takes a lot of computation to evaluate the AF, so this would be a
  naive approach to evaluate the entire $\tau, \nu$ plane.
\item Recognition of the the implied Fourier Transform in
  \eqref{AF-def}.  Requires recorrelation for each range $\tau$, but
  otherwise able to make use of the Fast Fourier Transform.  This is
  the basic approach used to calculate the cross ambiguity function
  for the Manastash Ridge Radar.  Sometimes referred to as a ``range
  first'' calculation.  A fast approach for computing a range-cut.
\item Recognition of the implied inverse Fourier Transform in
  \eqref{AF-def-alt}.  A completely valid approach which performs a
  correlation in the frquency domain.  A fast approach for computing a
  Doppler cut.
\end{itemize}


\subsection{Interpretation of the Ambiguity Function}

As defined, $|\chi(\tau,\nu)|$ is the magnitude of the response of the
matched filter to delayed and Doppler-shifted versions of the original
single.  It's a bit subtle, but $\chi$ has units of ``voltage''
(explain more), and thus the square, $|\chi(\tau,\nu)|^2$ (no relation
to ``chi-squared statistics,'' alas), has units of ``power'' or
``energy'' and is thus an ``observable.''

Since $|\chi|^2$ is strictly non-negative and integrable over the
$\tau, \nu$ plane, it can apparently serve as some sort of probability
density.  It represents a \textit{forward problem} \index{forward
  problem} in the following sense. 

\begin{quote}
If a (normalized) target is present at the range/Doppler coordinates
$\tau, \nu$, the output power from the undelayed, un-Doppler shifted
matched filter would be $|\chi(\tau,\nu)|^2$.
\end{quote}

Thus, if we specify a particular situation, we can predict, or
simulate what we would expect to happen.  However, with a radar, and
with most instruments, we actually wish to solve a different problem,
an \textit{inverse problem}:\index{inverse problem}  

\begin{quote}
Given a set of samples of the output of a matched filter, what
possible target positions and motions could have been responsible for
the measurements?
\end{quote}

In general, inverse problems are far more difficult to solve than
forward problesm.  Indeed it can be tremendously difficult to even
pose inverse problems in ways which permit their solution.  Also, the
most general form of solution of an inverse problem is not a set of
parameters, but rather a probability density for the desired
parameters\cite{tarantola}. 

Forward problem is a sequence of conditional probabilities:
\begin{equation}
p(voltage measured) = p(no-noise-output) * p(noise voltage)
no-noise-output-power = chi(tau,nu)*target_power(sigma,tau,nu)
\end{equation}

TO BE CONTINUED

\subsection{Ambiguity of some basic signals}

\subsubsection{Square Pulse} The ambiguity function for the basic
square pulse is readily evaluated, and given in the following formula:
\begin{equation}
|\chi(\tau,\nu)|^2 = \left\{\begin{array}{lcc}
\frac{1 - \cos(2\pi\nu(1-|\tau|))}{2\pi^2 \nu^2} &\textrm{for} & |\tau| < 1\\
0  & \textrm{for} & |\tau| > 0
\end{array}\right.
\end{equation}
For the cut $\nu = 0$ the expected autocorrelation function (ACF)
emerges, $|\chi(\tau,0)| = 1 - |\tau|$ for $|\tau| < 1$.  The
ambiguity function is plotted in \figref{square-pulse-ambig}.

\begin{figure}
\centerline{\epsfxsize=0.4\textwidth  \epsffile{ambig-square-pulse-wire.eps} \epsfxsize=0.4\textwidth \epsffile{ambig-square-pulse-cntr.eps}}
\caption{\label{f:square-pulse-ambig} The ambiguity function $|\chi|$
  for a unit duration square pulse, plotted as a surface (left) and
  with contours (right)}
\end{figure}

\subsubsection{Gaussian Pulse}

Although not used for radar applications (because of the amplitude
modulation), we can compute the ambiguity function of a gaussian
pulse. CHECK THIS!!!!
\begin{eqnarray}
u(t) &=& \sqrt{\frac{1}{\sqrt{2\pi \sigma^2}} e^{-t^2/(2\sigma^2)}} \\
|\chi(\tau,\nu)| &=& e^{-\tau^2/(8\sigma^2)} e^{-2 \pi^2 \sigma^2 \nu^2}
\end{eqnarray}
This is clearly the product of two gaussians, one in the delay
direction ($\tau$) and one in the Doppler direction ($\nu$).  The
product of the widths is $2\sigma/(2\pi\sigma) = 1/(2\pi)$, a constant.

\subsubsection{Periodic Pulse Train}

Consider a train of repeated pulses, each pulse with waveform $u_0(t)$
having ambiguity function $\chi_0(\tau,\nu)$.  We can describe the
sequence of $N$ pulses separated by interval $T$ as follows:
\begin{equation}
u(t) = \frac{1}{\sqrt{N}} \sum_{n=0}^{N-1} u_0(t - nT)
\end{equation}

