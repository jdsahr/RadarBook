\typeout{BLHABAALSDKJAFSALKFJSAFLKJASLJFLJALKFJDLASFLJK}

\chapter{Detection}

Useful operation of a radar requires the detection of known signals.
These signals may be simple pulses, pure tones, or intricately coded,
or even random.  These signals are known, or, in the case of passive
radar, there should be some independent way of deducing what they were 

If the transmitted signal is $s(t)$, then we will be interested in
discovering the amplitude $a$, delay $\tau$, and Doppler Shift $v$ in
the presence of noise $n(t)$, in a received signal.  Of course
additional signal properties are of interest, such as the angle of
arrival, angular image, but these properties are derived from spatial
information --- a vector of signals --- and we'll begin by considering
a simple scalar signal.

\section{Narrowband Signal}

We will find it very convenient to work with a narrowband
representation of radar signals.  As a practical matter the actual
radiated signal will be a real-valued signal of the following form
\begin{equation} \label{e:realmodel0}
y(t) = s(t) \cos[2 \pi f_0 t + \phi(t)]
\end{equation}
where the signal has some amplitude modulation $a(t)$ and phase
modulation $\phi(t)$ and a carrier frequency $f_0$.  An alternative
formulation would be 
\begin{equation}
y(t) = \Re\left[\tilde{s}(t) \exp(2\pi j f_0 t)\right]
\end{equation}
where we now include the phase modulation into a complex amplitude
$s(t)$.  This permits very convenient incorporation of amplitude and
phase changes, delays, and Doppler shifts:
\begin{equation}
y(t) = \Re\left[a \tilde{s}(t - \tau) \exp(2\pi j f_0 (t-\tau)) \exp(2\pi j \nu
  (t-\tau))\right]
\end{equation}
In fact, we can let the signal $\tilde{y}(t)$ be complex,
\begin{equation} \label{e:analyticmodel0}
\tilde{y}(t) = a\;\tilde{s}(t - \tau) \exp(2\pi j f_0 (t-\tau)) \exp(2\pi j \nu  (t-\tau))
\end{equation}
One problem with this signal is that the imaginary part of
$\tilde{y}(t)$ has not been specified.  The usual response is to
constrain the imaginary part of $\tilde{y}(t)$ so that the whole
signal spectrum has identically zero content for negative frequencies.
Such a signal is denoted the ``analytic signal'' in communications
literature \cite{vantrees,whalen,helstrom}.  Given a real signal
$s(t)$, the Hilbert Transform creates the appropriate imaginary part
to create an analytic signal.

\begin{factoid}
The Matlab function {\protect\texttt{hilbert(x)}} returns the analytic signal
$\tilde{x}$ given a real signal $x$.
\end{factoid}

Suppose that the spectrum of $\tilde{S}$ is entirely bounded within
$\pm f_s$, and further suppose that $f_s \ll f_0$.  Then such a
signal is said to be ``narrowband'' because the spectral width is much
less than the center frequency.  The Nyquist Rate of the signal $s$ is
$2 f_s$, the full bandwidth of the signal, including the negative
spectral content.

\begin{factoid} All radio broadcast signals are narrowband signals.
   FM channels have (full) bandwidth 200 kHz at a center frequency of
   100 MHz; AM channels have a (full) bandwidth of about 20 kHz at 1
   MHz.  In both cases the signal bandwidth is only 2 percent of the
   center frequency.
\end{factoid}

In the discussion of bandwidth above, the Doppler frequency shift
$\nu$ was not mentioned.  In general, the maximum Doppler Shift
$\nu_{max}$ is usually smaller than the radar signal bandwidth $f_s$.
Exceptions do exist, such as pure CW illuminators with very low
bandwidth.

\section{The IQ Receiver}

When we wish to receive and process narrowband signals, a particularly
convenient receiver is available which produces a signal which is very
nearly ``analytic.''  It is possible to build finite impulse response
(FIR) processors which can produce an approximation of the Hermite
Transform, but a somewhat different method is used in practice.

First, we can rewrite \eqref{analyticmodel0} as follows:
\begin{equation}
y(t) = \Re\{ \tilde{y}(t) \} = 
            \frac{1}{2}\left[\tilde{y}(t) + \tilde{y}^\ast(t)\right] 
\end{equation}
Now, let us ``down convert'' by multiplying $y(t)$ by the conjugate
carrier signal $\exp(-2\pi j f_0 t)$,
\begin{eqnarray}
e^{-2\pi j f_0 t} y(t) &=&
\frac{1}{2} \left(
  a \tilde{s}(t - \tau) e^{2\pi j \nu t} e^{- 2\pi j (f_0 + \nu) \tau}
  \; + \right.  \nonumber
\\
& & \;\;\;\;\;\; \left.a^\ast \tilde{s}^\ast(t - \tau) e^{-2\pi j \nu t} 
   e^{2\pi j(f_0 + \nu) \tau} e^{-4\pi j f_0 t} \right)
\end{eqnarray}
Note that the first term has only low frequency content, while the
second term has been shifted to high (negative) frequency.  If we
apply a low pass filter which slices off the high (negative)
frequencies, we will be left with the ``baseband signal'' $y_b(t)$,
\begin{eqnarray}
y_b(t) &=& {\cal L}\{e^{-2\pi j f_0 t} \tilde{y}(t)\} \\
       &=& \frac{1}{2}a \tilde{s}(t - \tau) e^{2\pi j \nu (t-\tau)} e^{- 2\pi
       j f_0 \tau} 
\end{eqnarray}
Thus the signal $y_b(t)$ is directly proportional to the complex
modulation, scaled by the scattering amplitude $a$ and a phase factor
$\exp(-2\pi j f_0 \tau)$, which in general can not be determined but
will usually be unimportant anyway.

Now the baseband signal $y_b(t)$ is complex valued, and it would be
reasonable to object.  Nevertheless this signal is perfectly
realizable by simply permitting $y_b(t)$ to be decomposed into real
and imaginary parts, each part being real valued.

NEED FIGURE, NEED MORE TEXT

\section{The Matched Filter}

Let us initially consider the detection of a signal which is neither
delayed nor Doppler shifted, but whose amplitude $a$ is unknown.  The
received signal will be processed by conversion to baseband, and then
passed through a filter with some impulse response $h(t)$.  The result
is a new signal $r(t)$,
\begin{eqnarray}
r(t) &=& h(t) \ast (y(t) + n(t)) \\
     &=& \int_0^\infty h^\ast(t') [y(t-t') + n(t-t')]\; dt'
\end{eqnarray}





\chapter{Introduction}

\noindent {\Huge P}hase codes present a powerful means for modifying
the ambiguity of simple pulses.  

\section{Barker Codes}

\newpage

\section{M-sequences}

There is a large class of nearly perfect binary phase codes known as
M-sequences (``maximal length'') or PN-sequences (``pseudo noise'').
In contrast to the Barker Codes, the M-sequences exist for large
length (indeed for any length $M = 2^N - 1$, for positive integer
$N$).   The M-sequences share the nice autocorrelation property of the
Barker code in the cyclic as apposed to acyclic sense.  Thus when we
compute the range sidelobes via autocorrelation we have
\begin{equation}
\chi(\tau,0) = \sum_{k=0}^{k < N} c_k c_{k-\tau}
\end{equation}
and it is understood that either the subscripts of $c$ are computed
modulo $N$ or equivalently that the code is periodic with period $N$,
$c_k = c_{k+N}$.

The periodicity of the codes implies that they are 100 percent
duty-cycle --- this is a CW modulation scheme, implying that the radar
using these waveforms must be bistatic, although McEliece has
suggested that the waveforms actually work well aperiodically.

The construction of the codes is a direct, novel application of a
topic in abstract algebra, the Galois Fields arising from prime
polynomials modulo 2.  As a practical matter the codes are extremely
easy to generate in digital logic consisting of shift registers and
EXOR gates.  As an example, consider \figref{mseq1}, which shows a
shift register implementation associated with the polynomial $x^3 + x
+ 1 = 0$.

The shift register state stores the sequence $x^n$ which has been
reduced modulo the polynomial $x^3 + x + 1$, with coefficients
evaluated modulo two.  Thus we have
\begin{equation}
\begin{array}{lclclcl}
1 \\
x \\
x^2 \\
x^3 &\equiv& x+1 \\
x^4 &      & x^2 + x \\
x^5 &      & x^3 + x^2 &\equiv& x^2 + x + 1 \\
x^6 &      &           &      & x^3 + x^2 + x &\equiv&x^2 + 1 \\
x^7 &      &           &      &               &  &x^3 + x \equiv 1
\end{array}
\end{equation}
Thus we can see that $x^7 \equiv 1$ modulo $x^3 + x + 1$, which means
(of course) that $x^{7n} \equiv 1$.  The M-sequences are formed by
examining the coefficients of any bit; if we examine the 1s place, we
have $1001011$.  Mapping $1$ to $+$ and $0$ to $-$ we achieve a
biphase code with nearly ideal cyclic autocorrelation: the zero lag
has value $N$ and all other lags have value $-1$.

A little experimentation reveals that the sequence generated by
multiplying one M-sequence by a delayed copy is itself the same
M-sequence with a third shift.  Also, one can see that the M-sequences
have exactly one more $1$ than $0$.  Assembling these facts leads to
the strong correlation properties of the M sequences.

\begin{figure}
\centerline{\epsfxsize=0.6\textwidth \epsffile{mseq-sr.eps}}
BLAH.
\caption{\label{f:mseq1} This shift register circuit generates an M
sequence based on the polynmial $x^3 + x + 1 = 0$, which can be
rearranged (modulo $2$) to be $x^3 = 1 + x$.  The m-sequence arises
from the ``ones'' place of the shift register system, $\ldots 1001011\ldots$.}
\end{figure}

It is quite fortunate that prime polynomials modulo 2 exist for all
orders of polynomials.  Thus, unlike Barker codes arbitrarily long
M-sequences can be found (although not for all lengths).

The underlying polynomials are ``prime'' which means that they have no
factors except themselves and unity.  Of course, this factorization is
to be performed with coefficients modulo 2.  Thus, the polynomials $x$
and $x + 1$ are prime, although $x^2 + 1$ is not since it is the
square of $x+1$ (modulo 2).  A beginning table of prime polynomials
can be found in \tabref{ppolys}.  This table is not exhaustive, it's
just an indication of some primes.  None of the entries are really
large enough to use; in practice primes of order 10, giving sequences
of length 1023 begin to be useful.  A prime polynomial of order 48 is
embedded in the Global Positioning System algorithms; it is evaluated
each microsecond, and its period is thus (8.9 years).  The Qualcomm
CDMA protocol uses a generalization of m-sequences called ``Gold
codes'' \cite{uppala-1998,gold-1967,gold-1968} to permit several users
to share a single channel.

\begin{table}
\caption{\label{t:ppolys} A beginning table of prime polynomials
modulo 2.  The polynomials exist for all orders, and there are many of
them.  Polynomials of any particular order need only be tested by
primes of half order or smaller; also, a complete table can be built
using ``sieve'' methods.  As in the conventional natural primes, there
is no pattern which permits easy testing or generation of large primes.}
\centerline{\begin{tabular}{rl}
order & polynomials \\ \hline
1     & $x, x+1$ \\
2     & $x^2 + x + 1$ \\
3     & $x^3 + x + 1, x^3 + x^2 + 1$ \\
4     & $x^4 + x + 1, x^4 + x^3 + 1$ \\
5     & $x^5 + x^2 + 1, x^5 + x^3 + 1, x^5 + x^3 + x^2 + x + 1, \ldots$ \\
6     & $x^6 + x + 1, x^6 + x^3 + 1, \ldots$
\end{tabular}}
\end{table}

