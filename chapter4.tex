\chapter{Phase Codes for Underspread Targets}

Phase codes present a powerful means for modifying the ambiguity of
simple pulses.   The central idea is that the ``tent'' of a long
square pulse can be collapsed with an intricate phase coding, yielding
a more compact range resolution.  

For the most part, phase codes are considered from the viewpoint of
range clutter only, neglecting the Doppler component of the ambiguity
function.  These codes are very useful, but are most effective for
targets whose Doppler content evolves much less rapidly than the radar
waveform.

\section{Barker Codes}

In 1953 Barker\cite{barker-1953} developed a simple class of codes for
automatic synchronization of communications systems.  The basic idea
was to identify code sequences consisting of $+$ and $-$ phases which
produced a very sharp ACF peak with low delay sidelobes.  The best you
can do with binary phase codes is to accept sidelobes of $0$ (for even
lags) or $\pm 1$ for odd lags.  There are a finite number of such
codes, and all of them are given in \tabref{barker}.


\begin{table}
\caption{\label{t:barker} List of all Barker codes.  Note that if $c$
  is a Barker code, then so is $-c$, as well as the time reversal of $c$.}
\begin{center}
\begin{tabular}{rl}
length & pattern \\ \hline
2      & $++$, \rule{1em}{0mm} $+-$ \\
3      & $++-$ \\
4      & $+++-$,  \rule{1em}{0mm} $++-+$ \\
5      & $+++-+$ \\
7      & $+++--+-$ \\
11     & $+++---+--+-$ \\
13     & $+++++--++-+-+$
\end{tabular}
\end{center}
\end{table}

The limited number of such codes is an interesting mathematical
curiosity, but not really a practical one as we shall see later in the
discussion of random codes.  

The proof mechanisms for the non-existence of long Barker codes is
interesting, and (somewhat surprisingly) rests upon the periodic
correlation functions of these explicitly aperiodic codes.   Among
other things, it becomes clear that long codes of length $4n+1,
4n+2,4n+3$ are easily ruled out, and we learn that the odd length
codes must have range sidelobes of the same sign.  Codes of length
$4n$, however, require considerably more effort to prove their
existence (or nonexistence), partly because of their possible
alternating sidelobe signs.

It appears that the existence or non-existence of Barker codes of
length $4n$ remains open in the mathematical sense.  However, in the
practical sense the matter is closed.  It is known that there are no
additional Barker codes of length less than 200 or so.  A Barker code
of length 64 would be useful; a Barker code of length $2^{16}$ would
not be useful, because of transmitter bandwidth and other
considerations.

There are, however, longer codes with low range sidelobes that are
useful, such as the ``almost Barker Code'', of length 28, which has 9
range sidelobes of magnitude 2, the rest being 1 or 0
\cite{gray+farley-1973} \footnote{Radio Science, v8 no. 2, pp 123-131,
  feb 1973}
\begin{center}
$++-++-+--+---+---+---++++---$
\end{center}

% \clearpage

\section{Complementary Codes}

Consider the pair of 4-baud Barker Codes $+++-$ and $++-+$.  If you
compute and plot their autocorrelation functions, you'll notice that
the pattern of sidelobes has opposite sign (except for the central
peak).  Thus, if we interrogate a (constant) target first with one
pulse, then with the second, and add the two results point by point, a
``perfect'' pulse compression is obtained with zero amplitude
sidelobes.

Although 4-baud complementary codes are marginally useful, it is easy
to create much longer codes using the following theorem:

\begin{quote}
If $A$, $B$ are a complementary code pair, then the concatenated codes
$AB, A\bar{B}$ are also complementary (where the overbar indicates
changing the sign of each bit).
\end{quote}

Using this formula it is easy to create very long codes (and you can
observe that the 4-baud codes are the extension of the two 2-baud
codes).  However codes created in this fashion are usually suboptimal
in that their individual sidelobes can be fairly large, thus requiring
more precision in cancellation to achieve the perfect composite result.

Thus, in general some sort of search is required to find good, long
complementary codes.

\section{Polyphase Codes}

You can make polyphase versions of Barker Codes, too.  These can be
much longer than the biphase codes.  Six-phase codes are used in some
aerospace applications.

\section{M-sequences}

There is a large class of nearly perfect binary phase codes known as
M-sequences (``maximal length'') or PN-sequences (``pseudo noise'').
In contrast to the Barker Codes, the M-sequences exist for large
length (indeed for any length $M = 2^N - 1$, for positive integer
$N$).   The M-sequences share the nice autocorrelation property of the
Barker code in the cyclic as apposed to acyclic sense.  Thus when we
compute the range sidelobes via autocorrelation we have
\begin{equation}
\chi(\tau,0) = \sum_{k=0}^{k < N} c_k c_{k-\tau}
\end{equation}
and it is understood that either the subscripts of $c$ are computed
modulo $N$ or equivalently that the code is periodic with period $N$,
$c_k = c_{k+N}$.

The periodicity of the codes implies that they are 100 percent
duty-cycle --- this is a CW modulation scheme, implying that the radar
using these waveforms must be bistatic, although McEliece has
suggested that the waveforms actually work well aperiodically.

The construction of the codes is a direct, novel application of a
topic in abstract algebra, the Galois Fields arising from prime
polynomials modulo 2.  As a practical matter the codes are extremely
easy to generate in digital logic consisting of shift registers and
EXOR gates.  As an example, consider \figref{mseq1}, which shows a
shift register implementation associated with the polynomial $x^3 + x
+ 1 = 0$.

The shift register state stores the sequence $x^n$ which has been
reduced modulo the polynomial $x^3 + x + 1$, with coefficients
evaluated modulo two.  Thus we have
\begin{equation}
\begin{array}{lclclcl}
1 \\
x \\
x^2 \\
x^3 &\equiv& x+1 \\
x^4 &      & x^2 + x \\
x^5 &      & x^3 + x^2 &\equiv& x^2 + x + 1 \\
x^6 &      &           &      & x^3 + x^2 + x &\equiv&x^2 + 1 \\
x^7 &      &           &      &               &  &x^3 + x \equiv 1
\end{array}
\end{equation}
Thus we can see that $x^7 \equiv 1$ modulo $x^3 + x + 1$, which means
(of course) that $x^{7n} \equiv 1$.  The M-sequences are formed by
examining the coefficients of any bit; if we examine the 1s place, we
have $1001011$.  Mapping $1$ to $+$ and $0$ to $-$ we achieve a
biphase code with nearly ideal cyclic autocorrelation: the zero lag
has value $N$ and all other lags have value $-1$.

A little experimentation reveals that the sequence generated by
multiplying one M-sequence by a delayed copy is itself the same
M-sequence with a third shift.  Also, one can see that the M-sequences
have exactly one more $1$ than $0$.  Assembling these facts leads to
the strong correlation properties of the M sequences.

\begin{figure}
\centerline{\epsfxsize=0.6\textwidth \epsffile{mseq-sr.eps}}
\caption{\label{f:mseq1} This shift register circuit generates an M
sequence based on the polynmial $x^3 + x + 1 = 0$, which can be
rearranged (modulo $2$) to be $x^3 = 1 + x$.  The m-sequence arises
from the ``ones'' place of the shift register system, $\ldots 1001011\ldots$.}
\end{figure}

It is quite fortunate that prime polynomials modulo 2 exist for all
orders of polynomials.  Thus, unlike Barker codes arbitrarily long
M-sequences can be found (although not for all lengths).

The underlying polynomials are ``prime'' which means that they have no
factors except themselves and unity.  Of course, this factorization is
to be performed with coefficients modulo 2.  Thus, the polynomials $x$
and $x + 1$ are prime, although $x^2 + 1$ is not since it is the
square of $x+1$ (modulo 2).  A beginning table of prime polynomials
can be found in \tabref{ppolys}.  This table is not exhaustive, it's
just an indication of some primes.  None of the entries are really
large enough to use; in practice primes of order 10, giving sequences
of length 1023 begin to be useful.  A prime polynomial of order 48 is
embedded in the Global Positioning System algorithms; it is evaluated
each microsecond, and its period is thus (8.9 years).  The Qualcomm
CDMA protocol uses a generalization of m-sequences called ``Gold
codes'' \cite{uppala-1998,gold-1967,gold-1968} to permit several users
to share a single channel.

\begin{table}
\caption{\label{t:ppolys} A beginning table of prime polynomials
modulo 2.  The polynomials exist for all orders, and there are many of
them.  Polynomials of any particular order need only be tested by
primes of half order or smaller; also, a complete table can be built
using ``sieve'' methods.  As in the conventional natural primes, there
is no pattern which permits easy testing or generation of large primes.}
\centerline{\begin{tabular}{rl}
order & polynomials \\ \hline
1     & $x, x+1$ \\
2     & $x^2 + x + 1$ \\
3     & $x^3 + x + 1, x^3 + x^2 + 1$ \\
4     & $x^4 + x + 1, x^4 + x^3 + 1$ \\
5     & $x^5 + x^2 + 1, x^5 + x^3 + 1, x^5 + x^3 + x^2 + x + 1, \ldots$ \\
6     & $x^6 + x + 1, x^6 + x^3 + 1, \ldots$
\end{tabular}}
\end{table}

\section{Random Codes}

\subsection{Completely Random Codes}

If we contemplate the idea that good radar codes have an
autocorrelation function which resembles a delta function, we may
recall that white random processes also have a compact autocorrelation
function.  

A random sequence of $N$ $\pm 1$ bits will yield an autocorrelation
function with a zero lag height of $N$, with the typical sidelobe at
lag $n \ne 0$ having height $1/\sqrt{N-n}$.  Although this is not
particularly good, neither is it particularly bad.  Furthermore, such
codes will generally have good performance in the whole range-Doppler
ambiguity plane.

\subsection{Exhaustive Search}

In principle, one could simply examine all codes of length $N$ to find
which perform the best.  However, this method rapidly collapses in the
face of exponential reality, and the opportunity for finding long
codes through exhaustive search is probably vanishingly small.

\subsection{Optimized Random Codes}

One can get improved random codes by creating many of them, and
picking those which yield the best ambiguity properties.  Suppose we
invent a metric $M$ for goodness of a code $x_n$ with autocorrelation
$R_{xx}(n)$, such as 
\begin{equation}
M(x) = \sum_n |R_{xx}(n)|
\end{equation}
We can then create a population of codes and pick the best performer
from them.

Something a bit stronger can be done as well.  If we consider a metric
for code performance, then the shape of the manifold over all possible
codes is almost certainly a very bumpy, irregular surface.  Although
there is little structure locally, globally there will be some
noticable trends.  For example, the waveform which is all $+$ will
have a poor ambiguity, as will the waveform of alternating $+-$
phases.  If we can create a search process which preserves
``randomness'' but includes ``nearness'' then we may be able to find
good codes without wasting time in regions where only poor codes
exist.

Techniques which automate this process are widely used, and fall under
the general rubric of ``stochastic optimization.''  What one gives up
in analytic and definitive results can be regained through performance
specifications which would be difficult or impossible to express
analytically.  Sahr and Grannan\cite{sahr+grannan-1993} used the
technique of ``simulated annealing'' to design several long binary
phase codes with interesting correlation properties which would have
been extremely difficult to find by construction or exhaustive
search.  In particular they described a ``hurricane'' code for which
Barker-like performance was desired, but only for the innermost lags,
and ``quasi-orthogonal'' code pairs which individually behaved like
Barker codes, but whose cross-correlations were small.


