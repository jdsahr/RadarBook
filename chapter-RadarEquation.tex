\chapter{The Radar Equation}

In designing radars we need a method to predict their performance, to
answer the question, ``How large an echo signal will a target
present?'' or conversely, ``How far away might we detect a particular
target?''  The context of such questions is critical to understanding
the meaning of the answer.

We will construct the radar equation several times, beginning with
simple context and adding sophistication as we proceed.

\section{Elementary Monostatic Form}

A radar consists of a transmitter, a transmitting antenna, a receiver,
and a receiving antenna.  It is often the case that the receiver and
transmitter share an antenna.

If the transmitter has power $P_T$, and radiates its power through an
isotropic antenna, then the Poynting Flux (power flux density) is 
\begin{equation}
S_{iso} = \frac{P_T}{4\pi R^2} \;\;\; \textrm{W/m}^2
\end{equation}
where $R$ is the distance to the target.  If, however, the antenna has
some gain $G_T$, then in the direction of the beam the Poynting Flux
is increased by this factor, 
\begin{equation}
S = \frac{P_T G_T}{4\pi R^2}
\end{equation}

This Poynting Flux $S$ is therefore the ``brightness'' of the
radiation which strikes the target.  The target will interact with the
radiation, absorbing some fraction and scattering the rest; let us
assume for the moment that all the radiation is scattered, and that it
is scattered isotropically.  Therefore, the power radiated from the
target back to the receiver, over the same distance $R$ is
\begin{equation} \label{e:rcs1}
S_{scat} = \frac{P_{target}}{4\pi R^2}
\end{equation}

Now, how large is $P_{target}$?  It ought to scale with the incident
power flux density, and it also ought to depend on the size of the
target.  We make the connection that the target presents an area
$\sigma$ to the illumination, so that the intercepted power is
\begin{equation}
P_{target} = \sigma S
\end{equation}

Thus, the scattered Poynting Flux now can be expressed
\begin{equation}
S_{scat} = \frac{\sigma S}{4\pi R^2} =  \frac{P_T G_T \sigma}{(4\pi)^2 R^4}
\end{equation}

Finally, the receive antenna collects the scattered signal.  The
antenna presents an effective area $A_e$ to the scattered Poynting
Flux.  We may find it more convenient to instead relate the effective
area to the gain,
\begin{equation}
A_e = G \frac{\lambda^2}{4\pi}
\end{equation}
When the transmitter and receiver share an antenna, then $G_T = G_R =
G$.

Putting this all together, we have what is often called ``The Radar
Equation:''
\begin{equation} \label{e:radareq0}
P_R = \frac{P_T G^2 \sigma \lambda^2}{(4\pi)^3 R^4}
\end{equation}

It is common to express the radar equation in dB; note that $4\pi$ is
very nearly 11 dB, and that $R$ is in meters (not km, or any other
unit).

\subsection{Scattering Cross Section}

Previously we defined the scattering cross section $\sigma$ as the
area which accounts for the fraction of power intercepted by the
target (see \eqref{rcs1}).  This definition is not bad for starters,
but it has some major limitations in descriptive power.

For the radar problem, we don't really care how much power the target
intercepts, what we \textbf{really} care about is how much is
scattered towards our receiver.

A very nice (and general) definition for the bistatic scattering cross
section is given by \cite{tsang-kong-shin}:

\begin{equation}
\sigma(\hat\imath, \hat o) = 4\pi \lim_{r\rightarrow\infty} r^2 
\frac{\vec{E}_s\cdot\vec{E}_s^\ast}{\vec{E}_i\cdot\vec{E}_i^\ast}
\end{equation}


\subsection{Noise}

The received power $P_r$ in \eqref{radareq0} tells us how large the
scattered signal is, but it does not tell us whether that signal is
detectable.  Many signals compete for the attention of a radar,
including noise, clutter (echoes from targets we do not want to see),
interference (signals from other sources which accidentally interfere
with our receiver), and jamming (signals which others present to
interfere with our radar).

Let us consider noise first.  Noise is an unavoidable aspect of all
physical systems, which arises from finite temperature effects.  
Targets are embedded in a setting which has a finite temperature, and
thus the background radiates noise power into the antennas.  In
addition, the circuitry which manipulates the signals is itself at
finite temperature, and thus contributes noise power to the total
signal.

Noise must also be specified in its temporal and spectral content.  It
can be impulsive (``shot noise'') or constant; it can be gaussian or
non gaussian (in amplitude statistics); it can be ``white'' (a flat
power spectrum) or ``colored'' (a non-constant power spectrum).

For our immediate purposes we'll assume that the noise is constant,
gaussian, and white.  In this case the noise can be described with a
simple power spectral density,
\begin{equation}
\frac{dP_N(f)}{df} = k_B T_{sys} \;\;\;\; \textrm{W/Hz}
\end{equation}
where $k_B = 1.38\times 10^{-23}$ W$/^\circ$K/Hz, and $T_{sys}$ is the
effective temperature of the receiving system

Thus the total noise power accompanying the signal in a receiver with
bandwidth $B$ (Hz) is
\begin{equation}
P_N = k_B T_{sys} B
\end{equation}

We can now modify the radar equation to create a version which
describes the detectability of a scatterer by comparing the scattered
power to the noise power.

\begin{equation}
\textrm{SNR} = \frac{P_R}{P_N} = \frac{P_T G^2 \sigma
\lambda^2}{(4\pi)^3 k_B T_{sys} B}
\end{equation}

Roughly speaking, a target is just detectable when the SNR = 1,
although we will need to refine this statement considerably for it to
be of use.  In aerospace applications an SNR > 15 dB is often
required, whereas for some other applications an SNR considerably
smaller than unity will suffice.

By reducing the bandwidth or the system temperature we can increase
the sensitivity. Special techniques are required to reduce the system
temperature below about 500 $^\circ K$.  However, some targets exist
in a very noisy environment so that the system performances is
dominated by external noise.  Targets and transmitter waveforms often
create strong lower bounds on practical bandwidth, so that the
bandwidth cannot be arbitrarily lowered without banishing the desired
signal.

\subsection{Clutter}

Clutter is a kind of noise-like signal which interferes with
detection, but is due to the receipt of unwanted transmitter echoes
from uninteresting targets.  The word ``clutter'' is used in two
slightly different senses.  In the aerospace community, ``clutter''
refers specifically to echoes from other specific targets, such as
trees, buildings, birds, rain, which ionospheric scientists would call
``ground clutter.''  In the ionospheric community ``clutter'' or
``range clutter'' refers to confusing signals at a particular range
which are due to scatterers at a different range --- in the aerospace
community this would be known as ``ambiguity.''  At any rate, clutter
is any noiselike signal which goes away if you turn the transmitter
off.  

We could assign a scattering cross section $\sigma_c$ to clutter
sources (as well as a specific range).  Since both the scattered
signal and the clutter power depends upon the transmitter power, the
Signal to Clutter Ration (SCR) is independent of the transmitter
power.  To put it differently, you cannot overcome clutter by
increasing the transmitter power.

Ground clutter is associated with targets that are not moving, or at
least not moving very fast.  Such clutter can often be substantially
removed with appropriate Doppler processing, with a particularly
simple example being ``moving target indicator'' or MTI radar.

Ground clutter is typically ``near'' the transmitter.  In ionospheric
radars most of the ground clutter is gone after about 80 km, at which
point a ``pure'' signal remains.  However we must recal that the delay
between a target at 75 km and at 90 km is only 100 $\mu s$; if the 75
km ground clutter is large, the receiver must recover recover its
sensitivity in time to ``see'' the weaker, desirable target.

Finally, one can sometimes address ground clutter problems by careful
design of antenna patterns so that the ground clutter is not
illuminated in the first place.  Such techniques are usually
restricted to short wavelengths.

\subsection{Jamming and Interference}

It may be that some other party intentionally or accidently radiates a
signal which desensitizes the radar.  There are a variety of
techniques to address this:
\begin{enumerate}
\item locate the interferor and reduce its emissions. 
\item increase the transmitter power.  advantage: simple and
effective.  disadvantage: consumes more power, radiates larger signal.
\item adaptive beam forming to locate and null the interference.
advantage: sophisticated response, provides location of interferor
disadvantage: requires more complex antenna and receiver, much more
computation, creates ``blind spots'' where you can never see.
\item move to a new location which does not suffer interference.
advantage: effective; disadvantage: lose original field of view
\end{enumerate}
In military applications ``jamming'' is also known as ``electronic
counter measures'' or ECM, while counter measures to jamming are
called ``electronic counter counter measures'' or ECCM.  There is an
enormous literature on these topics, and whole engineering journals
devoted to the subject.



\section{Bistatic Radar Equation}

In the previous development we assumed that the transmitter and
receiver shared an antenna at a single site.  This is not necessary,
although it is often the most convenient and efficient arrangment.
However, in the late twentieth century a number of technological
advances have reduced some of the challenges to operating transmitters
separately from receivers.  These ``bistatic'' radars have some
interesting operational features which should increase their
prevalence.
\begin{itemize}
\item bistatic radar transmitters can use very high duty cycle
waveforms, which permit efficient, low peak power transmitters.
\item no T/R switch is needed.
\item receivers may be forward located with low visual impact
\item a target may be viewed from several angles, increasing the
quality of information gathered.
\item receivers may be able to operate ``passively,'' taking advantage
of ``transmitters of opportunity.''
\end{itemize}

For an scatterer in a bistatic system there are now two distances in
the formula: $R_T$ is the distance from the target to the transmitter;
$R_R$ is the distance from the target to the receiver, so that the
bistatic radar equation becomes
\begin{equation}
P_R = \frac{P_T G_T G_R \, \sigma \lambda^2}{(4\pi)^3 R_T^2 R_R^2}
\end{equation}
In the equation above no structure is shown in the scattering cross
section, but in we need to account for the generally complex structure
of the cross section,
\begin{equation}
\sigma = \sigma(\lambda, \hat{\imath}, \hat{o})
\end{equation}
Here $\lambda$ represents the frequency dependence of the target
scattering, $\hat{\imath}$ is a unit vector pointing in the direction
of the illuminating wave, and $\hat{o}$ is a unit vector pointing in
the direction of the scattered wave.  This expression is necessarily
limited to the far field.

In fact, the expression for $\sigma$ should be further expanded to
account for the relative scatter of the two polarization states that
the waves may have (thus $\sigma$ is a tensor of rank $2\times 2$).
If the target has high enough velocity, then relativistic corrections
would complicate the expression further.  We'll have a bit more to say
about scattering cross section: it is a large topic.

% \hline

\section{The Radar Equation (2)}

Mathematical formulas which describe the strength or detectability of
targets as a function of range, transmitter power, antenna gains,
etc., are known as ``the radar equation.''  In order to complete these
expressions we'll have to introduce the concept of ``radar scattering
cross section'' (RCS).  We will not comment extensively on the RCS, an
enormous topic.

To start we will consider the case of a simple point target which lies
at a suitably large distance so that the radar waveform may be treated
as a plane wave, and also that the angular size of the target be much
smaller than the antenna beam (these assumptions may all be relaxed.).

Consider a transmitter which produces $P_t$ power, and delivers this
power to an isotropic antenna.  The power density (Poynting Flux)
found at a distance $R$ from the antenna is thus
\begin{displaymath}
S_t = \frac{P_t}{4\pi R^2}
\end{displaymath}

If the antenna has gain $G_t$, then in the direction of that gain the
power density is
\begin{equation}
S_t = \frac{P_t G_t}{4\pi R^2}
\end{equation}

A point-like target intercepts a fraction of the incident power
$\sigma S_t$ and re-radiates it (assume isotropic scatter for the moment).
Then the power density returned to the transmitter over the same
distance $R$ is
\begin{equation}
S_r = \frac{\sigma S_t}{4 \pi R^2} = \frac{\sigma G_t P_t}{(4\pi)^2R^4}
\end{equation}
Therefore the received power $P_r$ is simply the scattered power
density $S_r$ times the effective area of the receiving antenna $A_r$:
\begin{equation}
P_r = \frac{\sigma A_r G_t P_t}{(4\pi)^2R^4}
\end{equation}
There is a simple relation between the effective area of an antenna
and its gain,
\begin{equation}
A = \frac{G \lambda^2}{4\pi}
\end{equation}
where $\lambda$ is the wavelength of the radiation.  Thus the
expression for the received power may be written in terms of the
antenna gains or areas
\begin{equation}
P_r = \frac{\sigma G_r G_t \lambda^2 P_t}{(4\pi)^3 R^4}
    = \frac{\sigma A_r A_t P_t}{4\pi  \lambda^2 R^4}
\end{equation}
For many radars, the transmitter and receiver share a single antenna,
in which case the transmitting and receiving gains (or areas) are
identical.

Considering the equations above it is plain that the fraction of
intercepted power $\sigma$ has units of area.  The parameter $\sigma$
is known as the radar cross section, and in the usage above it has the
additional implication of being the ``backscatter cross section.''

\subsection{features of the radar equation}

The simplicity of the radar equation makes it very easy to analyze,
and there are important implications for radar systems in even this
most basic form.

\begin{itemize}
\item \textbf{Inverse fourth power range dependence.}  As a target of
constant size $\sigma$ changes range, the scattered power varies
rapidly.  A target just detectable (SNR $\approx$ 0 dB) at range $R$ is
easily detectable at range $R/2$ (SNR $\approx$ 12 dB), while being
undetectable at range $2R$ (SNR $\approx -12$ dB).
\item \textbf{Doubly effective antenna.}  For a monostatic radar in
which the same antenna is used for transmission and reception, the
gain appears \textit{twice}, thus a 20 dB antenna contributes 40 dB to
the overall signal strength.
\item \textbf{High dynamic range of required of receivers.}  A target
tracked from 10 km to 100 km has a signal strength change of 10,000.
When coupled with ``clutter'' from targets which are not interesting
but otherwise large (nearby trees, buildings, mountains, etc.), or
electronic countermeasures (ECM) such as jamming, receivers with
dynamic range of 100 dB are desirable.  Receivers with instantaneous
dynamic range of 140 dB exist.
\item \textbf{short wavelengths desirable.}  Assuming that engineering
considerations limit the physical size ($A$) of antennas, the signal
strength increases for shorter wavelengths (when the antenna area is
held constant).
\item \textbf{high power transmitters are desirable.}  The scattered
power is clearly proportional to the transmitter power.  At suitably
high power, the transmitted signal may actually modify the target ---
induced heating, for example.
\end{itemize}

It's important to realize that all of the items above can be followed,
by ``yes, but ...'' arguments.  For example, the argument about
preferring short wavelengths fails if the target scattering cross
section $\sigma \propto \lambda^{-k}$ for $k > 2$.  Other engineering
considerations have yet to be considered, such as signal-to-noise
ratio and receiver bandwidth.  In the case of ``Thomson scatter''
radar studies of the ionosphere, the Doppler spectrum of the scattered
signal contains useful information only if the radio wavelength is
considerably larger than the ``Debye Length'' (a centimeter or so)
\cite{nicholson-1983}.  The wide dynamic range required of
(especially) military radars inspires the design of receivers with
features that are vastly different from ``high quality'' radio
receivers.

The lesson is that it is important to consider the whole radar
\textit{system} to evaluate the performance.  Radar systems frequently
present very unusual constraints which may be very cleverly addressed
if one's engineering judgment is not clouded by conventional
electronic systems.

\subsection{Signal to Noise Ratio}

Expressions for the scattered power do not really indicate the
usefulness of the information presented to radars, especially for
``weak'' signals.  In the absence of something to compare to, the
concept of ``weak'' is not defined.  In the real world, noise provides
a background of confusing signals.

The spectral density of blackbody radiation depends upon the
temperature and the waelength, as described by the Planck Radiation
Law:
\begin{equation}
N(\lambda)\,d\lambda = \frac{2h\nu^3}{c^2}\frac{1}{\exp(h\nu/k_B T) -
1}\, d\lambda
\end{equation}
where $h = 6.63\times 10^{-34}$ J/Hz (Planck's Constant) and $k_B =
1.38\times 10^{-23}$ J (Boltzmann's Constant), $T$ is the temperature
in $^\circ$K, and $\nu$ is the frequency in Hz.

At frequencies $\nu \ll k_B T/h$, the power spectral density is
essentially constant.  For $T = 300 ^\circ$K, $k_B T/h = 6.24\times
10^{12}$ Hz, which is in the infrared region.  In this low frequency
limit, we have
\begin{equation}
N(\nu) \, d\nu = k_B T \, d\nu
\end{equation}
The quantity $k_B T$ is abbreviated $N_0$, the ``one sided noise
spectral density'', and at $T = $ 300 $^\circ$K $N_0 = 4.14\times
10^{-21}$ W/Hz.

The noise power entering an ideal receiver with bandwidth $B$ exposed
to an environment at temperature $T$ is $N_0 = k_B T B$.  Suppose a
transmitter emits a pulse of length $t_p$, which means that the echo
from a point target will also have length $t_p$.  The amount of energy
scattered into the receiver during this time is simply $P_r t_p$, and
the amount of noise entering the receiver during this time is just
$N_0 t_p$.  Dividing the signal energy by the noise energy, we arrive
at the expression for the SNR of a single pulse:
\begin{equation}
\textrm{SNR}_1 = \frac{P_r t_p}{N_0 t_p} =
	\frac{P_t G_t G_r \lambda^2 \sigma}{(4\pi)^3 R^4 N_0 f_b}
\end{equation}
Since 
