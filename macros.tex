% macros.tex.
%
% $Id$
%
% These are several LaTeX commands and environment modifications which
% allow improved typesetting of Cornell University dissertations.
% 
% Questions to John Sahr johns@alfven.spp.cornell.edu
%

\typeout{JDS MACROS}

\newcommand{\citey}[1]{\cite{#1}}

\newcommand{\sn}[2]{{#1} \times 10^{#2}}  % scientific notation a x 10^b
\newcommand{\cmto}[1]{{\rm cm}^{#1}}      % simple power macro for cm units
\newcommand{\tens}[1]{\stackrel{\rule{1ex}{0.5mm}}{#1}}
					  % like \vec{} for tensors
\newcommand{\nablaperp}{\nabla_{\!\perp}} % a better \nable_\perp
\newcommand{\pr}{\prime}                  % abbreviation for \prime
\newcommand{\ppr}{{\prime\prime}}         % abbreviation for \prime\prime

\newcommand{\llangle}{\left\langle}       % big left langle
\newcommand{\rrangle}{\right\rangle}      % big right rangle

\newcommand{\Ci}{\textrm{Ci}}
\newcommand{\Si}{\textrm{Si}}

\newcommand{\sinc}{\textrm{sinc}}
\newcommand{\cosech}{\textrm{cosech}}

\newcommand{\iint}{\int\!\!\!\int}
\newcommand{\curl}{\nabla\times}

\newcommand{\noop}[1]{}                   % handy for ``commenting out''
\newtheorem{definition}{definition}       % definition environment
\newtheorem{example}{example}             % example environment

\newcounter{deffc}
\newenvironment{deff}[1]{\noindent\stepcounter{deffc}\textit{def \arabic{chapter}.\arabic{deffc}}: \textbf{#1}\rule{0mm}{2em} ---}{\\}

\newlength{\defaultskip}
\setlength{\defaultskip}{\baselineskip}

% My labelling scheme works as follows:
%
%  figure   references all begin with ``f:''
%  table    references all begin with ``t:''
%  text     references all begin with ``p:''
%  equation references all begin with ``e:''

\newcommand{\eqref}[1]{(\ref{e:#1})}          % formatted equation
					      % reference, i.e. (3.31)
\newcommand{\figref}[1]{Fig.~\ref{f:#1}}      % formatted figure
					      % reference, i.e. Fig.~3.31
\newcommand{\tabref}[1]{Table~\ref{t:#1}}     % formatted table
					      % reference, i.e. Table~3.31
\newcommand{\coderef}[1]{Code Ex.~\ref{c:#1}} % formatted code example
					      % reference, i.e. Code Ex.~3.31

% the following function may be needed for my thesis
% \newtheorem{definition}{Definition}[chapter] % environment for
%					      % definitions 

% tcaption{} is the same as caption, except that you get a single
% spaced caption instead of regular text spacing.  tfigure{}
% (ttable{}) is the same as figure{} (table{}), forcing the [tb]
% placement, and  making sure the lof gets made right.  This should
% be used for both tcaption{} and ttable{} (below).

\newcommand{\tcaption}[1]{\caption{#1\SS}}

\newenvironment{tfigure}{\begin{figure}[tb]}{\addtocontents{lof}{\protect\addvspace{\lotlofextraspace}}\end{figure}} 

% ttable{} the considerations are the same as for tfigure{}.

\newenvironment{ttable}{\begin{table}[tb]}{\addtocontents{lot}{\protect\addvspace{\lotlofextraspace}}\end{table}}

% yacc{} environment with \item[] sets code for yacc somewhat nicely

\newcommand{\yitem}[1]{\item[{{\bf #1} \hfill}]}

\newenvironment{yacc}%
{\SS\begin{list}{}{\itemsep=1ex \leftmargin=1.25in \labelwidth=1.0in \parsep=0.0pt \labelsep=.13in}}{\end{list}\DS}

% from Kopka and Daly, page 192, for typesetting chemical formulas
\newlength{\fntxvi} \newlength{\fntxvii}
\newcommand{\chemical}[1]
{{\fontencoding{OMS}\fontfamily{cmsy}\selectfont
  \fntxvi\the\fontdimen16\font
  \fntxvii\the\fontdimen17\font
  $\mathrm{#1}$
  \fontencoding{OMS}\fontfamily{cmsys}\selectfont
  \fontdimen16\font=\fntxvi \fontdimen17\font=\fntxvii}}

% end of macros.tex
